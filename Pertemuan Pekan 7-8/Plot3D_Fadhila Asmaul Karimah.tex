\documentclass{article}

\usepackage{eumat}

\begin{document}
\begin{eulernotebook}
\eulerheading{Menggambar Plot 3D dengan EMT}
\begin{eulercomment}
Modul ini adalah pengenalan plot 3D di Euler. Kita memerlukan plot 3D
untuk memvisualisasikan fungsi dua variabel.

Euler menggambar fungsi tersebut menggunakan algoritma pengurutan
untuk menyembunyikan bagian di latar belakang. Secara umum Euler
menggunakan proyeksi sentral. Defaultnya adalah dari kuadran x-y
positif menuju titik asal x=y=z=0, tetapi sudut=0° dilihat dari arah
sumbu y. Sudut pandang dan ketinggian dapat diubah.

Euler bisa memplot

- permukaan dengan garis bayangan dan level atau rentang level,\\
- titik-titik langit,\\
- kurva parametrik,\\
- permukaan implisit.

Plot 3D suatu fungsi menggunakan plot3d. Cara termudah adalah dengan
memplot ekspresi dalam x dan y. Parameter r mengatur rentang plot
sekitar (0,0).
\end{eulercomment}
\begin{eulerprompt}
>aspect(1.5); plot3d("x^2+sin(y)",-5,5,0,6*pi):
\end{eulerprompt}
\eulerimg{17}{images/Plot3D_Fadhila Asmaul Karimah-001.png}
\begin{eulerprompt}
>plot3d("x^2+x*sin(y)",-5,5,0,6*pi):
\end{eulerprompt}
\eulerimg{17}{images/Plot3D_Fadhila Asmaul Karimah-002.png}
\begin{eulercomment}
Silakan lakukan modifikasi agar gambar "talang bergelombang" tersebut
tidak lurus melainkan melengkung/melingkar, baik melingkar secara
mendatar maupun melingkar turun/naik (seperti papan peluncur pada
kolam renang. Temukan rumusnya.
\end{eulercomment}
\begin{eulerprompt}
>aspect(1); plot3d("8x^2-y^2",-6,8,0,10*pi):
\end{eulerprompt}
\eulerimg{27}{images/Plot3D_Fadhila Asmaul Karimah-003.png}
\begin{eulerprompt}
>reset();
\end{eulerprompt}
\eulerheading{Fungsi Dua Variabel}
\begin{eulercomment}
Untuk grafik suatu fungsi, gunakan

- ekspresi sederhana dalam x dan y,\\
- nama fungsi dari dua variabel,\\
- atau matriks data.

Standarnya adalah kisi-kisi kawat berisi dengan warna berbeda di kedua
sisi. Perhatikan bahwa jumlah interval kisi default adalah 10, tetapi
plot menggunakan jumlah default persegi panjang 40x40 untuk membuat
permukaannya. Ini dapat diubah.

- n=40, n=[40,40]: jumlah garis kisi di setiap arah\\
- grid=10, grid=[10,10]: jumlah garis grid di setiap arah

Kami menggunakan default n=40 dan grid=10.
\end{eulercomment}
\begin{eulerprompt}
>plot3d("x^2+y^2"): 
\end{eulerprompt}
\eulerimg{27}{images/Plot3D_Fadhila Asmaul Karimah-004.png}
\begin{eulercomment}
Interaksi pengguna dimungkinkan dengan parameter \textgreater{} pengguna. Pengguna
dapat menekan tombol berikut.

- kiri, kanan, atas, bawah: memutar sudut pandang\\
- +,-: memperbesar atau memperkecil\\
- a: menghasilkan anaglyph (lihat di bawah)\\
- l : tombol nyalakan sumber cahaya (lihat dibawah)\\
- spasi: reset ke default\\
- kembali: akhiri interaksi\\
x40 untuk membuat permukaannya. Ini dapat diubah.

- n=40, n=[40,40]: jumlah garis kisi di setiap arah\\
- grid=10, grid=[10,10]: jumlah garis grid di setiap arah

Kami menggunakan default n=40 dan grid=10.
\end{eulercomment}
\begin{eulerprompt}
>plot3d("exp(-x^2+y^2)",>user, ...
>  title="Turn with the vector keys (press return to finish)"):
\end{eulerprompt}
\eulerimg{27}{images/Plot3D_Fadhila Asmaul Karimah-005.png}
\begin{eulercomment}
Rentang plot untuk fungsi dapat ditentukan dengan

- a,b: rentang x\\
- c,d: rentang y\\
- r: persegi simetris di sekitar (0,0).\\
- n: jumlah subinterval untuk plot

Ada beberapa parameter untuk membuat skala fungsi atau mengubah
tampilan grafik.

fscale: membuat skala ke nilai fungsi (defaultnya adalah \textless{}fscale).\\
scale: angka atau vektor 1x2 untuk membuat skala ke arah x dan y.\\
frame: jenis bingkai (default 1).
\end{eulercomment}
\begin{eulerprompt}
>plot3d("exp(-(x^2+y^2)/5)",r=8,n=60,fscale=5,scale=1.2,frame=3,>user):
\end{eulerprompt}
\eulerimg{27}{images/Plot3D_Fadhila Asmaul Karimah-006.png}
\begin{eulercomment}
Tampilan dapat diubah dengan berbagai cara yang berbeda.

- distance: jarak pandang ke plot.\\
- zoom: nilai zoom.\\
- angle: sudut terhadap sumbu y negatif dalam radian.\\
- height: tinggi tampilan dalam radian.

Nilai default dapat diperiksa atau diubah dengan fungsi view(). Ini
mengembalikan parameter-parameter dalam urutan di atas.
\end{eulercomment}
\begin{eulerprompt}
>view
\end{eulerprompt}
\begin{euleroutput}
  [5,  2.6,  2,  0.4]
\end{euleroutput}
\begin{eulercomment}
Jarak yang lebih dekat memerlukan zoom yang lebih sedikit. Efeknya
lebih mirip lensa sudut lebar.

Pada contoh berikut, angle=0 dan height=0 dilihat dari sumbu y
negatif. Label sumbu untuk y disembunyikan dalam kasus ini.
\end{eulercomment}
\begin{eulerprompt}
>plot3d("x^2+y",distance=3,zoom=1,angle=pi/2,height=0):
\end{eulerprompt}
\eulerimg{27}{images/Plot3D_Fadhila Asmaul Karimah-007.png}
\begin{eulercomment}
Plot selalu terlihat ke tengah kubus plot. Anda dapat memindahkan
pusat dengan parameter pusat.
\end{eulercomment}
\begin{eulerprompt}
>plot3d("x^4+y^2",a=0,b=1,c=-1,d=1,angle=-20°,height=20°, ...
>  center=[0.4,0,0],zoom=5):
\end{eulerprompt}
\eulerimg{27}{images/Plot3D_Fadhila Asmaul Karimah-008.png}
\begin{eulercomment}
Plot tersebut diubah skala untuk muat ke dalam kubus satuan saat
ditampilkan. Jadi, tidak perlu mengubah jarak atau zoom tergantung
pada ukuran plot. Label-label merujuk pada ukuran yang sebenarnya,
namun.

Jika Anda mematikan ini dengan scale=false, Anda perlu memastikan
bahwa plot masih muat ke dalam jendela plot dengan mengubah jarak
pandang atau zoom, dan memindahkan pusatnya.
\end{eulercomment}
\begin{eulerprompt}
>plot3d("5*exp(-x^2-y^2)",r=2,<fscale,<scale,distance=13,height=50°, ...
>  center=[0,0,-2],frame=3):
\end{eulerprompt}
\eulerimg{27}{images/Plot3D_Fadhila Asmaul Karimah-009.png}
\begin{eulercomment}
Tersedia juga grafik polar. Parameter polar=true menggambar grafik
polar. Fungsi tetap harus menjadi fungsi dari x dan y. Parameter
"fscale" mengubah skala fungsi dengan skala sendiri. Selain itu,
fungsi akan disesuaikan dengan ukuran kubus.
\end{eulercomment}
\begin{eulerprompt}
>plot3d("1/(x^2+y^2+1)",r=5,>polar, ...
>fscale=2,>hue,n=100,zoom=4,>contour,color=blue):
\end{eulerprompt}
\eulerimg{27}{images/Plot3D_Fadhila Asmaul Karimah-010.png}
\begin{eulerprompt}
>function f(r) := exp(-r/2)*cos(r); ...
>plot3d("f(x^2+y^2)",>polar,scale=[1,1,0.4],r=pi,frame=3,zoom=4):
\end{eulerprompt}
\eulerimg{27}{images/Plot3D_Fadhila Asmaul Karimah-011.png}
\begin{eulercomment}
Parameter rotate memutar fungsi dalam sumbu x sekitar sumbu x.

- rotate=1: Menggunakan sumbu x\\
- rotate=2: Menggunakan sumbu z
\end{eulercomment}
\begin{eulerprompt}
>plot3d("x^2+1",a=-1,b=1,rotate=true,grid=5):
\end{eulerprompt}
\eulerimg{27}{images/Plot3D_Fadhila Asmaul Karimah-012.png}
\begin{eulerprompt}
>plot3d("x^2+1",a=-1,b=1,rotate=2,grid=5):
\end{eulerprompt}
\eulerimg{27}{images/Plot3D_Fadhila Asmaul Karimah-013.png}
\begin{eulerprompt}
>plot3d("sqrt(25-x^2)",a=0,b=5,rotate=1):
\end{eulerprompt}
\eulerimg{27}{images/Plot3D_Fadhila Asmaul Karimah-014.png}
\begin{eulerprompt}
>plot3d("x*sin(x)",a=0,b=6pi,rotate=2):
\end{eulerprompt}
\eulerimg{27}{images/Plot3D_Fadhila Asmaul Karimah-015.png}
\begin{eulercomment}
Ini adalah sebuah plot dengan tiga fungsi.
\end{eulercomment}
\begin{eulerprompt}
>plot3d("x","x^2+y^2","y",r=2,zoom=3.5,frame=3):
\end{eulerprompt}
\eulerimg{27}{images/Plot3D_Fadhila Asmaul Karimah-016.png}
\eulerheading{Plot Kontur}
\begin{eulercomment}
Untuk plot ini, Euler menambahkan garis-garis kisi. Sebagai gantinya,
kita bisa menggunakan garis-garis tingkat dan satu warna atau spektrum
warna. Euler dapat menggambar tinggi fungsi pada plot dengan shading.
Dalam semua plot 3D, Euler dapat menghasilkan anaglif merah/cyan.

- \textgreater{}hue: Mengaktifkan shading ringan alih-alih kawat.\\
- \textgreater{}contour: Menampilkan garis kontur otomatis pada plot.\\
- level=... (or levels): Sebuah vektor nilai untuk garis kontur.

Nilai defaultnya adalah level="auto", yang menghitung beberapa garis
kontur secara otomatis. Seperti yang Anda lihat pada plot,
tingkat-tingkat tersebut sebenarnya adalah rentang tingkat.

Gaya default dapat diubah. Untuk plot kontur berikutnya, kita
menggunakan kisi yang lebih halus dengan 100x100 titik, memperbesar
fungsi dan plot, dan mengubah sudut pandang yang berbeda.
\end{eulercomment}
\begin{eulerprompt}
>plot3d("exp(-x^2-y^2)",r=2,n=100,level="thin", ...
> >contour,>spectral,fscale=1,scale=1.1,angle=45°,height=20°):
\end{eulerprompt}
\eulerimg{27}{images/Plot3D_Fadhila Asmaul Karimah-017.png}
\begin{eulerprompt}
>plot3d("exp(x*y)",angle=100°,>contour,color=green):
\end{eulerprompt}
\eulerimg{27}{images/Plot3D_Fadhila Asmaul Karimah-018.png}
\begin{eulercomment}
Pengaturan awal mengggunakan warna abu-abu. Namun, berbagai pilihan
warna spektrum juga tersedia.

- \textgreater{}spectral: Menggunakan skema spektral bawaan\\
- color=...: Menggunakan warna khusus atau skema spektral

Pada plot berikut, kita menggunakan skema spektral bawaan dan
meningkatkan jumlah titik untuk mendapatkan tampilan yang sangat
halus.
\end{eulercomment}
\begin{eulerprompt}
>plot3d("x^2+y^2",>spectral,>contour,n=100):
\end{eulerprompt}
\eulerimg{27}{images/Plot3D_Fadhila Asmaul Karimah-019.png}
\begin{eulercomment}
Daripada garis level otomatis, kita juga dapat mengatur nilai-nilai
garis level. Ini akan menghasilkan garis level yang tipis daripada
rentang level.
\end{eulercomment}
\begin{eulerprompt}
>plot3d("x^2-y^2",0,5,0,5,level=-1:0.1:1,color=redgreen):
\end{eulerprompt}
\eulerimg{27}{images/Plot3D_Fadhila Asmaul Karimah-020.png}
\begin{eulercomment}
Dalam plot berikut, kami menggunakan dua rentang level yang sangat
luas mulai dari -0,1 hingga 1, dan dari 0,9 hingga 1. Ini dimasukkan
sebagai matriks dengan batas level sebagai kolom.

Selain itu, kami melapisi grid dengan 10 interval di setiap arah.
\end{eulercomment}
\begin{eulerprompt}
>plot3d("x^2+y^3",level=[-0.1,0.9;0,1], ...
>  >spectral,angle=30°,grid=10,contourcolor=gray):
\end{eulerprompt}
\eulerimg{27}{images/Plot3D_Fadhila Asmaul Karimah-021.png}
\begin{eulercomment}
Dalam contoh berikut, kami menggambar himpunan, di mana

\end{eulercomment}
\begin{eulerformula}
\[
f(x,y) = x^y-y^x = 0
\]
\end{eulerformula}
\begin{eulercomment}
Kami menggunakan satu garis tipis untuk garis tingkat.
\end{eulercomment}
\begin{eulerprompt}
>plot3d("x^y-y^x",level=0,a=0,b=6,c=0,d=6,contourcolor=red,n=100):
\end{eulerprompt}
\eulerimg{27}{images/Plot3D_Fadhila Asmaul Karimah-023.png}
\begin{eulercomment}
Ini adalah mungkin untuk menampilkan sebuah bidang kontur di bawah
plot. Warna dan jarak dari plot dapat ditentukan.
\end{eulercomment}
\begin{eulerprompt}
>plot3d("x^2+y^4",>cp,cpcolor=green,cpdelta=0.2):
\end{eulerprompt}
\eulerimg{27}{images/Plot3D_Fadhila Asmaul Karimah-024.png}
\begin{eulercomment}
Berikut beberapa gaya lainnya. Kami selalu mematikan bingkai, dan
menggunakan berbagai skema warna untuk plot dan grid.




\end{eulercomment}
\begin{eulerprompt}
>figure(2,2); ...
>expr="y^3-x^2"; ...
>figure(1);  ...
>  plot3d(expr,<frame,>cp,cpcolor=spectral); ...
>figure(2);  ...
>  plot3d(expr,<frame,>spectral,grid=10,cp=2); ...
>figure(3);  ...
>  plot3d(expr,<frame,>contour,color=gray,nc=5,cp=3,cpcolor=greenred); ...
>figure(4);  ...
>  plot3d(expr,<frame,>hue,grid=10,>transparent,>cp,cpcolor=gray); ...
>figure(0):
\end{eulerprompt}
\eulerimg{27}{images/Plot3D_Fadhila Asmaul Karimah-025.png}
\begin{eulercomment}
Ada beberapa skema spektral lainnya, diberi nomor dari 1 hingga 9.
Tetapi Anda juga dapat menggunakan warna=nilai, di mana nilai

- spectral: untuk rentang dari biru hingga merah\\
- white: untuk rentang yang lebih lemah\\
- kuningbiru, unguhijau, birukuning, hijaumerah\\
- birukuning, hijaupurple, kuningbiru, merahhijau
\end{eulercomment}
\begin{eulerprompt}
>figure(3,3); ...
>for i=1:9;  ...
>  figure(i); plot3d("x^2+y^2",spectral=i,>contour,>cp,<frame,zoom=4);  ...
>end; ...
>figure(0):
\end{eulerprompt}
\eulerimg{27}{images/Plot3D_Fadhila Asmaul Karimah-026.png}
\begin{eulercomment}
Sumber cahaya dapat diubah dengan tombol "l" dan kunci kuror selama
interaksi pengguna. Ini juga dapat diatur dengan parameter.

light: arah cahaya\\
amb: cahaya ambien antara 0 dan 1

Perlu diperhatikan bahwa program tidak membedakan antara sisi plot.
Tidak ada bayangan. Untuk ini, Anda memerlukan Povray.
\end{eulercomment}
\begin{eulerprompt}
>plot3d("-x^2-y^2", ...
>  hue=true,light=[0,1,1],amb=0,user=true, ...
>  title="Press l and cursor keys (return to exit)"):
\end{eulerprompt}
\eulerimg{27}{images/Plot3D_Fadhila Asmaul Karimah-027.png}
\begin{eulercomment}
Parameter warna mengubah warna permukaan. Warna garis level juga dapat
diubah.
\end{eulercomment}
\begin{eulerprompt}
>plot3d("-x^2-y^2",color=rgb(0.2,0.2,0),hue=true,frame=false, ...
>  zoom=3,contourcolor=red,level=-2:0.1:1,dl=0.01):
\end{eulerprompt}
\eulerimg{27}{images/Plot3D_Fadhila Asmaul Karimah-028.png}
\begin{eulercomment}
Warna 0 memberikan efek pelangi yang istimewa.
\end{eulercomment}
\begin{eulerprompt}
>plot3d("x^2/(x^2+y^2+1)",color=0,hue=true,grid=10):
\end{eulerprompt}
\eulerimg{27}{images/Plot3D_Fadhila Asmaul Karimah-029.png}
\begin{eulercomment}
Permukaannya juga bisa transparan.
\end{eulercomment}
\begin{eulerprompt}
>plot3d("x^2+y^2",>transparent,grid=10,wirecolor=red):
\end{eulerprompt}
\eulerimg{27}{images/Plot3D_Fadhila Asmaul Karimah-030.png}
\eulerheading{Plot Implisit}
\begin{eulercomment}
Terdapat juga plot implisit dalam tiga dimensi. Euler menghasilkan
potongan melalui objek-objek tersebut. Fitur-fitur dari plot3d
mencakup plot implisit. Plot ini menampilkan himpunan nol dari suatu
fungsi dalam tiga variabel.\\
Solusi dari

\end{eulercomment}
\begin{eulerformula}
\[
f(x, y, z) = 0
\]
\end{eulerformula}
\begin{eulercomment}
dapat divisualisasikan dalam potongan sejajar dengan bidang x-y,
bidang x-z, dan bidang y-z.

implicit=1: potongan sejajar dengan bidang y-z\\
implicit=2: potongan sejajar dengan bidang x-z\\
implicit=4: potongan sejajar dengan bidang x-y

Tambahkan nilai-nilai ini, jika Anda ingin. Dalam contoh ini, kami
memplot


\end{eulercomment}
\begin{eulerformula}
\[
M = \{ (x,y,z) : x^2+y^3+zy=1 \}
\]
\end{eulerformula}
\begin{eulerprompt}
>plot3d("x^2+y^3+z*y-1",r=5,implicit=3):
\end{eulerprompt}
\eulerimg{27}{images/Plot3D_Fadhila Asmaul Karimah-033.png}
\begin{eulerprompt}
>c=1; d=1;
>plot3d("((x^2+y^2-c^2)^2+(z^2-1)^2)*((y^2+z^2-c^2)^2+(x^2-1)^2)*((z^2+x^2-c^2)^2+(y^2-1)^2)-d",r=2,<frame,>implicit,>user): 
\end{eulerprompt}
\eulerimg{27}{images/Plot3D_Fadhila Asmaul Karimah-034.png}
\begin{eulerprompt}
>plot3d("x^2+y^2+4*x*z+z^3",>implicit,r=2,zoom=2.5):
\end{eulerprompt}
\eulerimg{27}{images/Plot3D_Fadhila Asmaul Karimah-035.png}
\eulerheading{Plotting Data 3D}
\begin{eulercomment}
Sama seperti plot2d, plot3d menerima data. Untuk objek 3D, Anda perlu
menyediakan matriks nilai x, y, dan z, atau tiga fungsi atau ekspresi
fx(x, y), fy(x, y), fz(x, y).

\end{eulercomment}
\begin{eulerformula}
\[
\gamma(t,s) = (x(t,s),y(t,s),z(t,s))
\]
\end{eulerformula}
\begin{eulercomment}
Karena x, y, z adalah matriks, kita mengasumsikan bahwa (t, s)
berjalan melalui grid persegi. Sebagai hasilnya, Anda dapat membuat
gambar-gambar persegi panjang dalam ruang.

Anda dapat menggunakan bahasa matriks Euler untuk menghasilkan
koordinat dengan efektif.

Dalam contoh berikut, kita menggunakan vektor nilai t dan vektor kolom
nilai s untuk memparametrikan permukaan bola. Dalam gambaran, kita
dapat menandai wilayah-wilayah, dalam kasus kita, wilayah polar.
\end{eulercomment}
\begin{eulerprompt}
>t=linspace(0,2pi,180); s=linspace(-pi/2,pi/2,90)'; ...
>x=cos(s)*cos(t); y=cos(s)*sin(t); z=sin(s); ...
>plot3d(x,y,z,>hue, ...
>color=blue,<frame,grid=[10,20], ...
>values=s,contourcolor=red,level=[90°-24°;90°-22°], ...
>scale=1.4,height=50°):
\end{eulerprompt}
\eulerimg{27}{images/Plot3D_Fadhila Asmaul Karimah-037.png}
\begin{eulercomment}
Ini adalah contoh, yang merupakan grafik dari sebuah fungsi.
\end{eulercomment}
\begin{eulerprompt}
>t=-1:0.1:1; s=(-1:0.1:1)'; plot3d(t,s,t*s,grid=10):
\end{eulerprompt}
\eulerimg{27}{images/Plot3D_Fadhila Asmaul Karimah-038.png}
\begin{eulercomment}
Namun, kita dapat membuat berbagai jenis permukaan. Berikut ini adalah
permukaan yang sama sebagai fungsi

\end{eulercomment}
\begin{eulerformula}
\[
x = y \, z
\]
\end{eulerformula}
\begin{eulerprompt}
>plot3d(t*s,t,s,angle=180°,grid=10):
\end{eulerprompt}
\eulerimg{27}{images/Plot3D_Fadhila Asmaul Karimah-040.png}
\begin{eulercomment}
Dengan lebih banyak usaha, kita dapat menghasilkan banyak permukaan.

Pada contoh berikut, kita membuat tampilan berbayang dari sebuah bola
yang distorsi. Koordinat biasa untuk bola tersebut adalah

\end{eulercomment}
\begin{eulerformula}
\[
\gamma(t,s) = (\cos(t)\cos(s),\sin(t)\sin(s),\cos(s))
\]
\end{eulerformula}
\begin{eulercomment}
dengan

\end{eulercomment}
\begin{eulerformula}
\[
0 \le t \le 2\pi, \quad \frac{-\pi}{2} \le s \le \frac{\pi}{2}.
\]
\end{eulerformula}
\begin{eulercomment}
Kita mengdistorsi ini dengan faktor

\end{eulercomment}
\begin{eulerformula}
\[
d(t,s) = \frac{\cos(4t)+\cos(8s)}{4}.
\]
\end{eulerformula}
\begin{eulerprompt}
>t=linspace(0,2pi,320); s=linspace(-pi/2,pi/2,160)'; ...
>d=1+0.2*(cos(4*t)+cos(8*s)); ...
>plot3d(cos(t)*cos(s)*d,sin(t)*cos(s)*d,sin(s)*d,hue=1, ...
>  light=[1,0,1],frame=0,zoom=5):
\end{eulerprompt}
\eulerimg{27}{images/Plot3D_Fadhila Asmaul Karimah-044.png}
\begin{eulercomment}
Tentu saja, awan titik juga mungkin. Untuk menggambarkan data titik
dalam ruang, kita memerlukan tiga vektor untuk koordinat titik-titik
tersebut.

Gaya-gaya tersebut sama seperti dalam plot2d dengan points=true;
\end{eulercomment}
\begin{eulerprompt}
>n=500;  ...
>  plot3d(normal(1,n),normal(1,n),normal(1,n),points=true,style="."):
\end{eulerprompt}
\eulerimg{27}{images/Plot3D_Fadhila Asmaul Karimah-045.png}
\begin{eulercomment}
Ini juga memungkinkan untuk menggambar kurva dalam tiga dimensi (3D).
Dalam hal ini, lebih mudah untuk menghitung sebelumnya titik-titik
dari kurva tersebut. Untuk kurva-kurva dalam bidang, kita menggunakan
urutan koordinat dan parameter wire=true.
\end{eulercomment}
\begin{eulerprompt}
>t=linspace(0,8pi,500); ...
>plot3d(sin(t),cos(t),t/10,>wire,zoom=3):
\end{eulerprompt}
\eulerimg{27}{images/Plot3D_Fadhila Asmaul Karimah-046.png}
\begin{eulerprompt}
>t=linspace(0,4pi,1000); plot3d(cos(t),sin(t),t/2pi,>wire, ...
>linewidth=3,wirecolor=blue):
\end{eulerprompt}
\eulerimg{27}{images/Plot3D_Fadhila Asmaul Karimah-047.png}
\begin{eulerprompt}
>X=cumsum(normal(3,100)); ...
> plot3d(X[1],X[2],X[3],>anaglyph,>wire):
\end{eulerprompt}
\eulerimg{27}{images/Plot3D_Fadhila Asmaul Karimah-048.png}
\begin{eulercomment}
EMT juga dapat membuat plot dalam mode anaglyph. Untuk melihat plot
tersebut, Anda memerlukan kacamata merah/biru.
\end{eulercomment}
\begin{eulerprompt}
> plot3d("x^2+y^3",>anaglyph,>contour,angle=30°):
\end{eulerprompt}
\eulerimg{27}{images/Plot3D_Fadhila Asmaul Karimah-049.png}
\begin{eulercomment}
Seringkali, skema warna spektral digunakan untuk grafik ini. Ini
menekankan tinggi fungsi tersebut.
\end{eulercomment}
\begin{eulerprompt}
>plot3d("x^2*y^3-y",>spectral,>contour,zoom=3.2):
\end{eulerprompt}
\eulerimg{27}{images/Plot3D_Fadhila Asmaul Karimah-050.png}
\begin{eulercomment}
Euler juga dapat menggambar permukaan-parameterkan ketika
parameter-parameter tersebut adalah nilai-nilai x-, y-, dan z- dari
gambar grid berbentuk persegi panjang di dalam ruang.

Untuk demonstrasi berikutnya, kami menyiapkan parameter u dan v, dan
menghasilkan koordinat ruang dari kedua parameter tersebut.

\end{eulercomment}
\begin{eulerprompt}
>u=linspace(-1,1,10); v=linspace(0,2*pi,50)'; ...
>X=(3+u*cos(v/2))*cos(v); Y=(3+u*cos(v/2))*sin(v); Z=u*sin(v/2); ...
>plot3d(X,Y,Z,>anaglyph,<frame,>wire,scale=2.3):
\end{eulerprompt}
\eulerimg{27}{images/Plot3D_Fadhila Asmaul Karimah-051.png}
\begin{eulercomment}
Berikut contoh yang lebih rumit, yang megah dengan kacamata
merah/cyan.
\end{eulercomment}
\begin{eulerprompt}
>u:=linspace(-pi,pi,160); v:=linspace(-pi,pi,400)';  ...
>x:=(4*(1+.25*sin(3*v))+cos(u))*cos(2*v); ...
>y:=(4*(1+.25*sin(3*v))+cos(u))*sin(2*v); ...
> z=sin(u)+2*cos(3*v); ...
>plot3d(x,y,z,frame=0,scale=1.5,hue=1,light=[1,0,-1],zoom=2.8,>anaglyph):
\end{eulerprompt}
\eulerimg{27}{images/Plot3D_Fadhila Asmaul Karimah-052.png}
\eulerheading{Plot Statistik}
\begin{eulercomment}
Grafik batang juga mungkin. Untuk ini, kita harus memberikan:

- x: vektor baris dengan n+1 elemen\\
- y: vektor kolom dengan n+1 elemen\\
- z: matriks nxn dari nilai-nilai.

z bisa lebih besar, tetapi hanya nilai-nilai nxn yang akan digunakan.

Dalam contoh ini, kita pertama-tama menghitung nilai-nilai. Kemudian
kita menyesuaikan x dan y, sehingga vektor-vektor tersebut berpusat
pada nilai yang digunakan.
\end{eulercomment}
\begin{eulerprompt}
>x=-1:0.1:1; y=x'; z=x^2+y^2; ...
>xa=(x|1.1)-0.05; ya=(y_1.1)-0.05; ...
>plot3d(xa,ya,z,bar=true):
\end{eulerprompt}
\eulerimg{27}{images/Plot3D_Fadhila Asmaul Karimah-053.png}
\begin{eulercomment}
Mungkin untuk membagi plot permukaan menjadi dua atau lebih bagian.

\end{eulercomment}
\begin{eulerprompt}
>x=-1:0.1:1; y=x'; z=x+y; d=zeros(size(x)); ...
>plot3d(x,y,z,disconnect=2:2:20):
\end{eulerprompt}
\eulerimg{27}{images/Plot3D_Fadhila Asmaul Karimah-054.png}
\begin{eulercomment}
Jika Anda memuat atau menghasilkan matriks data M dari sebuah file dan
perlu membuat plotnya dalam 3D, Anda dapat melakukan penskalaan pada
matriks tersebut menjadi rentang [-1,1] dengan menggunakan perintah
"scale(M)", atau melakukan penskalaan dengan menggunakan "zscale". Ini
dapat dikombinasikan dengan faktor-faktor penskalaan individu yang
diterapkan secara tambahan.
\end{eulercomment}
\begin{eulerprompt}
>i=1:20; j=i'; ...
>plot3d(i*j^2+100*normal(20,20),>zscale,scale=[1,1,1.5],angle=-40°,zoom=1.8):
\end{eulerprompt}
\eulerimg{27}{images/Plot3D_Fadhila Asmaul Karimah-055.png}
\begin{eulerprompt}
>Z=intrandom(5,100,6); v=zeros(5,6); ...
>loop 1 to 5; v[#]=getmultiplicities(1:6,Z[#]); end; ...
>columnsplot3d(v',scols=1:5,ccols=[1:5]):
\end{eulerprompt}
\eulerimg{27}{images/Plot3D_Fadhila Asmaul Karimah-056.png}
\eulerheading{Permukaan Benda Putar}
\begin{eulerprompt}
>plot2d("(x^2+y^2-1)^3-x^2*y^3",r=1.3, ...
>style="#",color=red,<outline, ...
>level=[-2;0],n=100):
\end{eulerprompt}
\eulerimg{27}{images/Plot3D_Fadhila Asmaul Karimah-057.png}
\begin{eulerprompt}
>ekspresi &= (x^2+y^2-1)^3-x^2*y^3; $ekspresi
\end{eulerprompt}
\begin{eulerformula}
\[
\left(y^2+x^2-1\right)^3-x^2\,y^3
\]
\end{eulerformula}
\begin{eulercomment}
Kami ingin memutar kurva hati sekitar sumbu y. Berikut adalah ekspresi
yang mendefinisikan bentuk hati:

\end{eulercomment}
\begin{eulerformula}
\[
f(x,y)=(x^2+y^2-1)^3-x^2.y^3.
\]
\end{eulerformula}
\begin{eulercomment}
Selanjutnya kita menetapkan

\end{eulercomment}
\begin{eulerformula}
\[
x=r.cos(a),\quad y=r.sin(a).
\]
\end{eulerformula}
\begin{eulerprompt}
>function fr(r,a) &= ekspresi with [x=r*cos(a),y=r*sin(a)] | trigreduce; $fr(r,a)
\end{eulerprompt}
\begin{eulerformula}
\[
\left(r^2-1\right)^3+\frac{\left(\sin \left(5\,a\right)-\sin \left(  3\,a\right)-2\,\sin a\right)\,r^5}{16}
\]
\end{eulerformula}
\begin{eulercomment}
Ini memungkinkan untuk mendefinisikan fungsi numerik, yang memecahkan
untuk r, jika a diberikan. Dengan fungsi itu, kita dapat menggambar
hati yang berputar sebagai permukaan parametrik.
\end{eulercomment}
\begin{eulerprompt}
>function map f(a) := bisect("fr",0,2;a); ...
>t=linspace(-pi/2,pi/2,100); r=f(t);  ...
>s=linspace(pi,2pi,100)'; ...
>plot3d(r*cos(t)*sin(s),r*cos(t)*cos(s),r*sin(t), ...
>>hue,<frame,color=red,zoom=4,amb=0,max=0.7,grid=12,height=50°):
\end{eulerprompt}
\eulerimg{27}{images/Plot3D_Fadhila Asmaul Karimah-062.png}
\begin{eulercomment}
Berikut adalah plot 3D dari gambar di atas yang diputar sekitar sumbu
z. Kami mendefinisikan fungsi yang menggambarkan objek tersebut.
\end{eulercomment}
\begin{eulerprompt}
>function f(x,y,z) ...
\end{eulerprompt}
\begin{eulerudf}
  r=x^2+y^2;
  return (r+z^2-1)^3-r*z^3;
   endfunction
\end{eulerudf}
\begin{eulerprompt}
>plot3d("f(x,y,z)", ...
>xmin=0,xmax=1.2,ymin=-1.2,ymax=1.2,zmin=-1.2,zmax=1.4, ...
>implicit=1,angle=-30°,zoom=2.5,n=[10,100,60],>anaglyph):
\end{eulerprompt}
\eulerimg{27}{images/Plot3D_Fadhila Asmaul Karimah-063.png}
\eulerheading{Plot 3D Khusus}
\begin{eulercomment}
Fungsi plot3d bagus untuk dimiliki, tetapi tidak memenuhi semua
kebutuhan. Selain rutinitas yang lebih dasar, mungkin Anda bisa
mendapatkan plot bingkai dari objek apa pun yang Anda sukai.

Meskipun Euler bukan program 3D, itu dapat menggabungkan beberapa
objek dasar. Kami mencoba untuk memvisualisasikan sebuah parabola dan
tangennya.
\end{eulercomment}
\begin{eulerprompt}
>function myplot ...
\end{eulerprompt}
\begin{eulerudf}
    y=-1:0.01:1; x=(-1:0.01:1)';
    plot3d(x,y,0.2*(x-0.1)/2,<scale,<frame,>hue, ..
      hues=0.5,>contour,color=orange);
    h=holding(1);
    plot3d(x,y,(x^2+y^2)/2,<scale,<frame,>contour,>hue);
    holding(h);
  endfunction
\end{eulerudf}
\begin{eulercomment}
Sekarang framedplot() menyediakan bingkai, dan mengatur tampilan.
\end{eulercomment}
\begin{eulerprompt}
>framedplot("myplot",[-1,1,-1,1,0,1],height=0,angle=-30°, ...
>  center=[0,0,-0.7],zoom=3):
\end{eulerprompt}
\eulerimg{27}{images/Plot3D_Fadhila Asmaul Karimah-064.png}
\begin{eulercomment}
Dengan cara yang sama, Anda dapat menggambar bidang kontur secara
manual. Perhatikan bahwa plot3d() mengatur jendela ke fullwindow()
secara default, tetapi plotcontourplane() mengasumsikan hal tersebut.
\end{eulercomment}
\begin{eulerprompt}
>x=-1:0.02:1.1; y=x'; z=x^2-y^4;
>function myplot (x,y,z) ...
\end{eulerprompt}
\begin{eulerudf}
    zoom(2);
    wi=fullwindow();
    plotcontourplane(x,y,z,level="auto",<scale);
    plot3d(x,y,z,>hue,<scale,>add,color=white,level="thin");
    window(wi);
    reset();
  endfunction
\end{eulerudf}
\begin{eulerprompt}
>myplot(x,y,z):
\end{eulerprompt}
\eulerimg{27}{images/Plot3D_Fadhila Asmaul Karimah-065.png}
\eulerheading{Animasi}
\begin{eulercomment}
Euler dapat menggunakan bingkai (frames) untuk pra-menghitung animasi.

Salah satu fungsi yang menggunakan teknik ini adalah fungsi rotate.
Ini dapat mengubah sudut pandang dan menggambar ulang plot 3D. Fungsi
tersebut memanggil addpage() untuk setiap plot baru. Akhirnya, ia
menganimasikan plot-plot tersebut.

Silakan pelajari sumber kode fungsi rotate untuk melihat lebih banyak
detailnya.
\end{eulercomment}
\begin{eulerprompt}
>function testplot () := plot3d("x^2+y^3"); ...
>rotate("testplot"); testplot():
\end{eulerprompt}
\eulerimg{27}{images/Plot3D_Fadhila Asmaul Karimah-066.png}
\eulerheading{Menggambar Povray}
\begin{eulercomment}
Dengan bantuan berkas Euler povray.e, Euler dapat menghasilkan berkas
Povray. Hasilnya sangat bagus untuk dilihat.

Anda perlu menginstal Povray (32bit atau 64bit) dari
http://www.povray.org/,\\
dan letakkan sub-direktori "bin" dari Povray ke dalam path lingkungan,
atau atur variabel "defaultpovray" dengan path lengkap yang menunjuk
ke "pvengine.exe".

Antarmuka Povray Euler menghasilkan berkas Povray di direktori rumah
pengguna, dan memanggil Povray untuk menguraikan berkas-berkas ini.
Nama berkas default adalah current.pov, dan direktori default adalah
eulerhome(), biasanya c:\textbackslash{}Users\textbackslash{}Username\textbackslash{}Euler. Povray menghasilkan
berkas PNG, yang dapat dimuat oleh Euler ke dalam buku catatan. Untuk
membersihkan berkas-berkas ini, gunakan povclear().

Fungsi pov3d berada dalam semangat yang sama dengan plot3d. Ini dapat
menghasilkan grafik dari fungsi f(x,y), atau permukaan dengan
koordinat X,Y,Z dalam matriks, termasuk garis-garis level opsional.
Fungsi ini akan memulai raytracer secara otomatis, dan memuat adegan
ke dalam buku catatan Euler.

Selain pov3d(), ada banyak fungsi lain yang menghasilkan objek Povray.
Fungsi-fungsi ini mengembalikan string yang berisi kode Povray untuk
objek-objek tersebut. Untuk menggunakan fungsi-fungsi ini, mulai
berkas Povray dengan povstart(). Kemudian gunakan writeln(...) untuk
menulis objek-objek ke berkas adegan. Akhirnya, akhiri berkas dengan
povend(). Secara default, raytracer akan mulai, dan PNG akan
dimasukkan ke dalam buku catatan Euler.

Fungsi objek memiliki parameter bernama "look", yang memerlukan string
dengan kode Povray untuk tekstur dan penyelesaian objek tersebut.
Fungsi povlook() dapat digunakan untuk menghasilkan string ini. Ini
memiliki parameter untuk warna, transparansi, Phong Shading, dll.

Perlu diingat bahwa alam semesta Povray memiliki sistem koordinat yang
berbeda. Antarmuka ini menerjemahkan semua koordinat ke sistem Povray.
Jadi Anda dapat terus berpikir dalam sistem koordinat Euler dengan z
mengarah secara vertikal ke atas, dan sumbu x,y,z sesuai dengan tangan
kanan.

Anda perlu memuat berkas povray.
\end{eulercomment}
\begin{eulerprompt}
>load povray;
\end{eulerprompt}
\begin{eulercomment}
Pastikan direktori bin Povray ada dalam path. Jika tidak, edit
variabel berikut agar berisi path ke eksekusi povray.
\end{eulercomment}
\begin{eulerprompt}
>defaultpovray="C:\(\backslash\)Program Files\(\backslash\)POV-Ray\(\backslash\)v3.7\(\backslash\)bin\(\backslash\)pvengine.exe"
\end{eulerprompt}
\begin{euleroutput}
  C:\(\backslash\)Program Files\(\backslash\)POV-Ray\(\backslash\)v3.7\(\backslash\)bin\(\backslash\)pvengine.exe
\end{euleroutput}
\begin{eulercomment}
Untuk kesan pertama, kami menggambar fungsi sederhana. Perintah
berikut menghasilkan file povray di direktori pengguna Anda, dan
menjalankan Povray untuk pelacakan sinar file ini.

Jika Anda memulai perintah berikut, GUI Povray seharusnya terbuka,
menjalankan file, dan menutup secara otomatis. Karena alasan keamanan,
Anda akan ditanyai apakah Anda ingin mengizinkan file exe ini untuk
berjalan. Anda dapat menekan batal untuk menghentikan pertanyaan lebih
lanjut. Anda mungkin harus menekan OK di jendela Povray untuk mengakui
dialog awal Povray.
\end{eulercomment}
\begin{eulerprompt}
>plot3d("x^2+y^2",zoom=2):
\end{eulerprompt}
\eulerimg{27}{images/Plot3D_Fadhila Asmaul Karimah-067.png}
\begin{eulerprompt}
>pov3d("x^2+y^2",zoom=3);
\end{eulerprompt}
\eulerimg{27}{images/Plot3D_Fadhila Asmaul Karimah-068.png}
\begin{eulercomment}
Kita dapat membuat fungsi tersebut transparan dan menambahkan yang
lainnya. Kita juga dapat menambahkan garis level pada plot fungsi.
\end{eulercomment}
\begin{eulerprompt}
>pov3d("x^2+y^3",axiscolor=red,angle=-45°,>anaglyph, ...
>  look=povlook(cyan,0.2),level=-1:0.5:1,zoom=3.8);
\end{eulerprompt}
\eulerimg{27}{images/Plot3D_Fadhila Asmaul Karimah-069.png}
\begin{eulercomment}
Kadang-kadang perlu untuk mencegah penskalaan fungsi, dan penskalaan
fungsi secara manual.

Kita menggambar himpunan titik-titik dalam bidang kompleks, di mana
hasil kali jarak ke 1 dan -1 sama dengan 1.
\end{eulercomment}
\begin{eulerprompt}
>pov3d("((x-1)^2+y^2)*((x+1)^2+y^2)/40",r=2, ...
>  angle=-120°,level=1/40,dlevel=0.005,light=[-1,1,1],height=10°,n=50, ...
>  <fscale,zoom=3.8);
\end{eulerprompt}
\eulerimg{27}{images/Plot3D_Fadhila Asmaul Karimah-070.png}
\eulerheading{Plotting dengan Koordinat}
\begin{eulercomment}
Daripada menggunakan fungsi, kita dapat melakukan plotting dengan
koordinat. Seperti pada plot3d, kita memerlukan tiga matriks untuk
mendefinisikan objek tersebut.

Pada contoh ini, kita memutar sebuah fungsi sekitar sumbu z.
\end{eulercomment}
\begin{eulerprompt}
>function f(x) := x^3-x+1; ...
>x=-1:0.01:1; t=linspace(0,2pi,50)'; ...
>Z=x; X=cos(t)*f(x); Y=sin(t)*f(x); ...
>pov3d(X,Y,Z,angle=40°,look=povlook(green,0.1),height=20°,axis=0,zoom=4,light=[10,-5,5]);
\end{eulerprompt}
\eulerimg{27}{images/Plot3D_Fadhila Asmaul Karimah-071.png}
\begin{eulercomment}
Dalam contoh berikut, kita menggambar gelombang teredam. Kami
menghasilkan gelombang tersebut dengan bahasa matriks Euler.

Kami juga menunjukkan bagaimana objek tambahan dapat ditambahkan ke
dalam adegan pov3d. Untuk pembuatan objek, lihat contoh-contoh
berikut. Perhatikan bahwa plot3d akan menyesuaikan skala plot sehingga
cocok dalam kubus satuan.
\end{eulercomment}
\begin{eulerprompt}
>r=linspace(0,1,80); phi=linspace(0,2pi,80)'; ...
>x=r*cos(phi); y=r*sin(phi); z=exp(-5*r)*cos(8*pi*r)/3;  ...
>pov3d(x,y,z,zoom=5,axis=0,height=30°,add=povsphere([0,0,0.5],0.1,povlook(red)), ...
>  w=500,h=300);
\end{eulerprompt}
\eulerimg{16}{images/Plot3D_Fadhila Asmaul Karimah-072.png}
\begin{eulercomment}
Dengan metode shading canggih dari Povray, sangat sedikit titik dapat
menghasilkan permukaan yang sangat halus. Hanya di batas-batas dan
dalam bayangan trik ini mungkin menjadi jelas.

Untuk ini, kita perlu menambahkan vektor normal di setiap titik
matriks.
\end{eulercomment}
\begin{eulerprompt}
>Z &= x^2*y^3
\end{eulerprompt}
\begin{euleroutput}
  
                                   2  3
                                  x  y
  
\end{euleroutput}
\begin{eulercomment}
Persamaan dari permukaannya adalah [x, y, Z]. Kami menghitung dua
turunan terhadap x dan y dari ini dan mengambil hasil perkalian silang
sebagai normalnya.
\end{eulercomment}
\begin{eulerprompt}
>dx &= diff([x,y,Z],x); dy &= diff([x,y,Z],y);
\end{eulerprompt}
\begin{eulercomment}
Kami mendefinisikan normal sebagai hasil perkalian silang dari
turunan-turunan ini, dan mendefinisikan fungsi koordinat.
\end{eulercomment}
\begin{eulerprompt}
>N &= crossproduct(dx,dy); NX &= N[1]; NY &= N[2]; NZ &= N[3]; N,
\end{eulerprompt}
\begin{euleroutput}
  
                                 3       2  2
                         [- 2 x y , - 3 x  y , 1]
  
\end{euleroutput}
\begin{eulercomment}
Kami hanya menggunakan 25 poin.
\end{eulercomment}
\begin{eulerprompt}
>x=-1:0.5:1; y=x';
>pov3d(x,y,Z(x,y),angle=10°, ...
>  xv=NX(x,y),yv=NY(x,y),zv=NZ(x,y),<shadow);
\end{eulerprompt}
\eulerimg{27}{images/Plot3D_Fadhila Asmaul Karimah-073.png}
\begin{eulercomment}
Berikut adalah simpul Trefoil yang dibuat oleh A. Busser dalam Povray.
Terdapat versi yang diperbaiki dari ini dalam contoh-contoh.

See: Examples\textbackslash{}Trefoil Knot \textbar{} Trefoil Knot

Untuk tampilan yang bagus dengan tidak terlalu banyak titik, kami
menambahkan vektor normal di sini. Kami menggunakan Maxima untuk
menghitung vektor normal untuk kami. Pertama, tiga fungsi untuk
koordinat sebagai ekspresi simbolik.
\end{eulercomment}
\begin{eulerprompt}
>X &= ((4+sin(3*y))+cos(x))*cos(2*y); ...
>Y &= ((4+sin(3*y))+cos(x))*sin(2*y); ...
>Z &= sin(x)+2*cos(3*y);
\end{eulerprompt}
\begin{eulercomment}
Kemudian turunan dua vektor terhadap x dan y.
\end{eulercomment}
\begin{eulerprompt}
>dx &= diff([X,Y,Z],x); dy &= diff([X,Y,Z],y);
\end{eulerprompt}
\begin{eulercomment}
Sekarang yang normal, yang merupakan hasil perkalian silang dari kedua
turunan tersebut.
\end{eulercomment}
\begin{eulerprompt}
>dn &= crossproduct(dx,dy);
\end{eulerprompt}
\begin{eulercomment}
Sekarang kita mengevaluasi semua ini secara numerik.
\end{eulercomment}
\begin{eulerprompt}
>x:=linspace(-%pi,%pi,40); y:=linspace(-%pi,%pi,100)';
\end{eulerprompt}
\begin{eulercomment}
Vektor normal adalah hasil evaluasi dari ekspresi simbolis dn[i] untuk
i=1,2,3. Syntax untuk ini adalah \&"expression"(parameter). Ini
merupakan alternatif dari metode pada contoh sebelumnya, di mana kita
mendefinisikan ekspresi simbolis NX, NY, NZ terlebih dahulu.
\end{eulercomment}
\begin{eulerprompt}
>pov3d(X(x,y),Y(x,y),Z(x,y),>anaglyph,axis=0,zoom=5,w=450,h=350, ...
>  <shadow,look=povlook(blue), ...
>  xv=&"dn[1]"(x,y), yv=&"dn[2]"(x,y), zv=&"dn[3]"(x,y));
\end{eulerprompt}
\eulerimg{21}{images/Plot3D_Fadhila Asmaul Karimah-074.png}
\begin{eulercomment}
Kita juga dapat membuat grid dalam 3D.
\end{eulercomment}
\begin{eulerprompt}
>povstart(zoom=4); ...
>x=-1:0.5:1; r=1-(x+1)^2/6; ...
>t=(0°:30°:360°)'; y=r*cos(t); z=r*sin(t); ...
>writeln(povgrid(x,y,z,d=0.02,dballs=0.05)); ...
>povend();
\end{eulerprompt}
\eulerimg{27}{images/Plot3D_Fadhila Asmaul Karimah-075.png}
\begin{eulercomment}
Dengan povgrid(), kurva-kurva menjadi mungkin.
\end{eulercomment}
\begin{eulerprompt}
>povstart(center=[0,0,1],zoom=3.6); ...
>t=linspace(0,2,1000); r=exp(-t); ...
>x=cos(2*pi*10*t)*r; y=sin(2*pi*10*t)*r; z=t; ...
>writeln(povgrid(x,y,z,povlook(red))); ...
>writeAxis(0,2,axis=3); ...
>povend();
\end{eulerprompt}
\eulerimg{27}{images/Plot3D_Fadhila Asmaul Karimah-076.png}
\eulerheading{Objek Povray}
\begin{eulercomment}
Di atas, kami menggunakan pov3d untuk memplot permukaan. Antarmuka
povray dalam Euler juga dapat menghasilkan objek Povray. Objek-objek
ini disimpan sebagai string dalam Euler, dan perlu ditulis ke file
Povray.

Kami memulai output dengan povstart().
\end{eulercomment}
\begin{eulerprompt}
>povstart(zoom=4);
\end{eulerprompt}
\begin{eulercomment}
Pertama, kita mendefinisikan tiga silinder dan menyimpannya dalam
bentuk string dalam Euler.

Fungsi-fungsi seperti povx() hanya mengembalikan vektor [1,0,0], yang
dapat digunakan sebagai penggantinya.
\end{eulercomment}
\begin{eulerprompt}
>c1=povcylinder(-povx,povx,1,povlook(red)); ...
>c2=povcylinder(-povy,povy,1,povlook(yellow)); ...
>c3=povcylinder(-povz,povz,1,povlook(blue)); ...
\end{eulerprompt}
\begin{eulercomment}
Kalimat-kalimat tersebut berisi kode Povray, yang pada saat itu tidak
perlu kita pahami.
\end{eulercomment}
\begin{eulerprompt}
>c2
\end{eulerprompt}
\begin{euleroutput}
  cylinder \{ <0,0,-1>, <0,0,1>, 1
   texture \{ pigment \{ color rgb <0.941176,0.941176,0.392157> \}  \} 
   finish \{ ambient 0.2 \} 
   \}
\end{euleroutput}
\begin{eulercomment}
Seperti yang Anda lihat, kami menambahkan tekstur pada objek dalam
tiga warna yang berbeda.

Hal ini dilakukan dengan menggunakan povlook(), yang mengembalikan
string dengan kode Povray yang relevan. Kami dapat menggunakan warna
Euler default, atau mendefinisikan warna sendiri. Kami juga dapat
menambahkan transparansi, atau mengubah cahaya ambien.
\end{eulercomment}
\begin{eulerprompt}
>povlook(rgb(0.1,0.2,0.3),0.1,0.5)
\end{eulerprompt}
\begin{euleroutput}
   texture \{ pigment \{ color rgbf <0.101961,0.2,0.301961,0.1> \}  \} 
   finish \{ ambient 0.5 \} 
  
\end{euleroutput}
\begin{eulercomment}
Sekarang kita mendefinisikan sebuah objek persimpangan, dan menulis
hasilnya ke dalam file.
\end{eulercomment}
\begin{eulerprompt}
>writeln(povintersection([c1,c2,c3]));
\end{eulerprompt}
\begin{eulercomment}
Persimpangan tiga silinder sulit untuk dibayangkan, jika Anda belum
pernah melihatnya sebelumnya.
\end{eulercomment}
\begin{eulerprompt}
>povend;
\end{eulerprompt}
\eulerimg{27}{images/Plot3D_Fadhila Asmaul Karimah-077.png}
\begin{eulercomment}
Berikut ini adalah fungsi-fungsi yang menghasilkan fraktal secara
rekursif.

Fungsi pertama menunjukkan bagaimana Euler mengatasi objek Povray
sederhana. Fungsi povbox() mengembalikan sebuah string yang berisi
koordinat kotak, tekstur, dan penyelesaian.
\end{eulercomment}
\begin{eulerprompt}
>function onebox(x,y,z,d) := povbox([x,y,z],[x+d,y+d,z+d],povlook());
>function fractal (x,y,z,h,n) ...
\end{eulerprompt}
\begin{eulerudf}
   if n==1 then writeln(onebox(x,y,z,h));
   else
     h=h/3;
     fractal(x,y,z,h,n-1);
     fractal(x+2*h,y,z,h,n-1);
     fractal(x,y+2*h,z,h,n-1);
     fractal(x,y,z+2*h,h,n-1);
     fractal(x+2*h,y+2*h,z,h,n-1);
     fractal(x+2*h,y,z+2*h,h,n-1);
     fractal(x,y+2*h,z+2*h,h,n-1);
     fractal(x+2*h,y+2*h,z+2*h,h,n-1);
     fractal(x+h,y+h,z+h,h,n-1);
   endif;
  endfunction
\end{eulerudf}
\begin{eulerprompt}
>povstart(fade=10,<shadow);
>fractal(-1,-1,-1,2,4);
>povend();
\end{eulerprompt}
\eulerimg{27}{images/Plot3D_Fadhila Asmaul Karimah-078.png}
\begin{eulercomment}
Perbedaan memungkinkan pemisahan satu objek dari yang lain. Seperti
perpotongan, itu merupakan bagian dari objek CSG dalam Povray.
\end{eulercomment}
\begin{eulerprompt}
>povstart(light=[5,-5,5],fade=10);
\end{eulerprompt}
\begin{eulercomment}
Untuk demonstrasi ini, kita mendefinisikan sebuah objek dalam Povray,
daripada menggunakan sebuah string dalam Euler. Definisi tersebut
langsung ditulis ke dalam file.

Koordinat sebuah kotak dengan nilai -1 hanya berarti [-1,-1,-1].
\end{eulercomment}
\begin{eulerprompt}
>povdefine("mycube",povbox(-1,1));
\end{eulerprompt}
\begin{eulercomment}
Kita dapat menggunakan objek ini dalam povobject(), yang mengembalikan
sebuah string seperti biasanya.
\end{eulercomment}
\begin{eulerprompt}
>c1=povobject("mycube",povlook(red));
\end{eulerprompt}
\begin{eulercomment}
Kami menghasilkan sebuah kubus kedua, lalu memutarnya dan
membesarkannya sedikit.
\end{eulercomment}
\begin{eulerprompt}
>c2=povobject("mycube",povlook(yellow),translate=[1,1,1], ...
>  rotate=xrotate(10°)+yrotate(10°), scale=1.2);
\end{eulerprompt}
\begin{eulercomment}
Kemudian kita mengambil perbedaan dari kedua objek tersebut.
\end{eulercomment}
\begin{eulerprompt}
>writeln(povdifference(c1,c2));
\end{eulerprompt}
\begin{eulercomment}
Sekarang tambahkan tiga sumbu.
\end{eulercomment}
\begin{eulerprompt}
>writeAxis(-1.2,1.2,axis=1); ...
>writeAxis(-1.2,1.2,axis=2); ...
>writeAxis(-1.2,1.2,axis=4); ...
>povend();
\end{eulerprompt}
\eulerimg{27}{images/Plot3D_Fadhila Asmaul Karimah-079.png}
\eulerheading{Fungsi Implisit}
\begin{eulercomment}
Povray dapat menggambar himpunan di mana f(x, y, z) = 0, seperti
parameter implisit dalam plot3d. Hasilnya terlihat jauh lebih baik,
namun sintaks untuk fungsi ini agak berbeda. Anda tidak dapat
menggunakan keluaran dari Maxima atau ekspresi Euler.

\end{eulercomment}
\begin{eulerformula}
\[
((x^2+y^2-c^2)^2+(z^2-1)^2)*((y^2+z^2-c^2)^2+(x^2-1)^2)*((z^2+x^2-c^2)^2+(y^2-1)^2)=d
\]
\end{eulerformula}
\begin{eulerprompt}
>povstart(angle=70°,height=50°,zoom=4);
>writeln(povsurface("(pow(pow(x,2)+pow(y,2)-pow(c,2),2)+pow(pow(z,2)-1,2))(pow(pow(y,2)+pow(z,2)-pow(c,2),2)+pow(pow(x,2)-1,2))(pow(pow(z,2)+pow(x,2)-pow(c,2),2)+pow(pow(y,2)-1,2))-d",povlook(red))); ...
>povend();
\end{eulerprompt}
\begin{euleroutput}
  Error : Povray error!
  
  Error generated by error() command
  
  povray:
      error("Povray error!");
  Try "trace errors" to inspect local variables after errors.
  povend:
      povray(file,w,h,aspect,exit); 
\end{euleroutput}
\begin{eulerprompt}
>povstart(angle=25°,height=10°); 
>writeln(povsurface("pow(x,2)+pow(y,2)*pow(z,2)-1",povlook(blue),povbox(-2,2,"")));
>povend();
\end{eulerprompt}
\eulerimg{27}{images/Plot3D_Fadhila Asmaul Karimah-081.png}
\begin{eulerprompt}
>povstart(angle=70°,height=50°,zoom=4);
\end{eulerprompt}
\begin{eulercomment}
Buatlah permukaan implisit. Perhatikan sintaks yang berbeda dalam
ekspresi ini.
\end{eulercomment}
\begin{eulerprompt}
>writeln(povsurface("pow(x,2)*y-pow(y,3)-pow(z,2)",povlook(green))); ...
>writeAxes(); ...
>povend();
\end{eulerprompt}
\eulerimg{27}{images/Plot3D_Fadhila Asmaul Karimah-082.png}
\eulerheading{Objek Jaringan}
\begin{eulercomment}
Dalam contoh ini, kami akan menunjukkan bagaimana cara membuat objek
jala, dan menggambarnya dengan informasi tambahan.

Kami ingin memaksimalkan xy dengan kondisi x+y=1 dan menunjukkan
sentuhan tangensial dari garis level.
\end{eulercomment}
\begin{eulerprompt}
>povstart(angle=-10°,center=[0.5,0.5,0.5],zoom=7);
\end{eulerprompt}
\begin{eulercomment}
Kita tidak dapat menyimpan objek dalam bentuk string seperti
sebelumnya, karena terlalu besar. Jadi, kita mendefinisikan objek
dalam sebuah file Povray menggunakan #declare. Fungsi povtriangle()
melakukan ini secara otomatis. Ini dapat menerima vektor normal
seperti pov3d().

Berikut ini mendefinisikan objek mesh, dan langsung menulisnya ke
dalam file.
\end{eulercomment}
\begin{eulerprompt}
>x=0:0.02:1; y=x'; z=x*y; vx=-y; vy=-x; vz=1;
>mesh=povtriangles(x,y,z,"",vx,vy,vz);
\end{eulerprompt}
\begin{eulercomment}
Sekarang kita akan mendefinisikan dua cakram, yang akan dipotong
dengan permukaan.
\end{eulercomment}
\begin{eulerprompt}
>cl=povdisc([0.5,0.5,0],[1,1,0],2); ...
>ll=povdisc([0,0,1/4],[0,0,1],2);
\end{eulerprompt}
\begin{eulercomment}
Ketik permukaannya dikurangi dua cakram.
\end{eulercomment}
\begin{eulerprompt}
>writeln(povdifference(mesh,povunion([cl,ll]),povlook(green)));
\end{eulerprompt}
\begin{eulercomment}
Tulis dua perpotongan.
\end{eulercomment}
\begin{eulerprompt}
>writeln(povintersection([mesh,cl],povlook(red))); ...
>writeln(povintersection([mesh,ll],povlook(gray)));
\end{eulerprompt}
\begin{eulercomment}
Tulis sebuah titik maksimum.
\end{eulercomment}
\begin{eulerprompt}
>writeln(povpoint([1/2,1/2,1/4],povlook(gray),size=2*defaultpointsize));
\end{eulerprompt}
\begin{eulercomment}
Tambahkan sumbu dan selesaikan
\end{eulercomment}
\begin{eulerprompt}
>writeAxes(0,1,0,1,0,1,d=0.015); ...
>povend();
\end{eulerprompt}
\eulerimg{27}{images/Plot3D_Fadhila Asmaul Karimah-083.png}
\eulerheading{Anaglif dalam Povray}
\begin{eulercomment}
Untuk menghasilkan anaglif untuk kacamata merah/cyan, Povray harus
dijalankan dua kali dari posisi kamera yang berbeda. Ini menghasilkan
dua file Povray dan dua file PNG, yang dimuat dengan fungsi
loadanaglyph().

Tentu saja, Anda memerlukan kacamata merah/cyan untuk melihat
contoh-contoh berikut dengan benar.

Fungsi pov3d() memiliki sakelar sederhana untuk menghasilkan anaglif.
\end{eulercomment}
\begin{eulerprompt}
>pov3d("-exp(-x^2-y^2)/2",r=2,height=45°,>anaglyph, ...
>  center=[0,0,0.5],zoom=3.5);
\end{eulerprompt}
\eulerimg{27}{images/Plot3D_Fadhila Asmaul Karimah-084.png}
\begin{eulercomment}
Jika Anda membuat sebuah adegan dengan objek-objek, Anda perlu
menempatkan pembuatan adegan tersebut ke dalam sebuah fungsi, dan
menjalankannya dua kali dengan nilai yang berbeda untuk parameter
anaglyph.
\end{eulercomment}
\begin{eulerprompt}
>function myscene ...
\end{eulerprompt}
\begin{eulerudf}
    s=povsphere(povc,1);
    cl=povcylinder(-povz,povz,0.5);
    clx=povobject(cl,rotate=xrotate(90°));
    cly=povobject(cl,rotate=yrotate(90°));
    c=povbox([-1,-1,0],1);
    un=povunion([cl,clx,cly,c]);
    obj=povdifference(s,un,povlook(red));
    writeln(obj);
    writeAxes();
  endfunction
\end{eulerudf}
\begin{eulercomment}
Fungsi povanaglyph() melakukan semua ini. Parameter-parameternya mirip
dengan yang ada dalam povstart() dan povend() yang digabungkan.
\end{eulercomment}
\begin{eulerprompt}
>povanaglyph("myscene",zoom=4.5);
\end{eulerprompt}
\eulerimg{27}{images/Plot3D_Fadhila Asmaul Karimah-085.png}
\eulerheading{* Mendefinisikan Objek Sendiri}
\begin{eulercomment}
Antarmuka povray pada Euler berisi banyak objek. Namun, Anda tidak
terbatas pada objek-objek tersebut. Anda dapat membuat objek sendiri,
yang menggabungkan objek-objek lain atau benar-benar objek baru.

Kami akan mendemonstrasikan sebuah torus. Perintah Povray untuk ini
adalah "torus". Jadi, kami akan mengembalikan sebuah string dengan
perintah ini beserta parameter-parameternya. Perhatikan bahwa torus
selalu berada di pusat asal.
\end{eulercomment}
\begin{eulerprompt}
>function povdonat (r1,r2,look="") ...
\end{eulerprompt}
\begin{eulerudf}
    return "torus \{"+r1+","+r2+look+"\}";
  endfunction
\end{eulerudf}
\begin{eulercomment}
Ini adalah torus pertama kami.
\end{eulercomment}
\begin{eulerprompt}
>t1=povdonat(0.8,0.2)
\end{eulerprompt}
\begin{euleroutput}
  torus \{0.8,0.2\}
\end{euleroutput}
\begin{eulercomment}
Mari kita gunakan objek ini untuk membuat torus kedua, yang telah
diterjemahkan dan diputar.
\end{eulercomment}
\begin{eulerprompt}
>t2=povobject(t1,rotate=xrotate(90°),translate=[0.8,0,0])
\end{eulerprompt}
\begin{euleroutput}
  object \{ torus \{0.8,0.2\}
   rotate 90 *x 
   translate <0.8,0,0>
   \}
\end{euleroutput}
\begin{eulercomment}
Sekarang kita menempatkan objek-objek ini ke dalam sebuah adegan.
Untuk tampilannya, kita menggunakan Phong Shading.
\end{eulercomment}
\begin{eulerprompt}
>povstart(center=[0.4,0,0],angle=0°,zoom=3.8,aspect=1.5); ...
>writeln(povobject(t1,povlook(green,phong=1))); ...
>writeln(povobject(t2,povlook(green,phong=1))); ...
\end{eulerprompt}
\begin{eulerttcomment}
 >povend();
\end{eulerttcomment}
\begin{eulercomment}
memanggil program Povray. Namun, dalam kasus kesalahan, program ini
tidak menampilkan pesan kesalahan. Oleh karena itu, Anda sebaiknya
menggunakan

\end{eulercomment}
\begin{eulerttcomment}
 >povend(<exit);
\end{eulerttcomment}
\begin{eulercomment}

jika ada yang tidak berfungsi. Ini akan membuat jendela Povray tetap
terbuka.
\end{eulercomment}
\begin{eulerprompt}
>povend(h=320,w=480);
\end{eulerprompt}
\eulerimg{18}{images/Plot3D_Fadhila Asmaul Karimah-086.png}
\begin{eulercomment}
Berikut contoh yang lebih rinci. Kami menyelesaikan

\end{eulercomment}
\begin{eulerformula}
\[
Ax \le b, \quad x \ge 0, \quad c.x \to \text{Max.}
\]
\end{eulerformula}
\begin{eulercomment}
dan menunjukkan titik-titik yang layak dan optimum dalam plot 3D.
\end{eulercomment}
\begin{eulerprompt}
>A=[10,8,4;5,6,8;6,3,2;9,5,6];
>b=[10,10,10,10]';
>c=[1,1,1];
\end{eulerprompt}
\begin{eulercomment}
Pertama, mari kita periksa, apakah contoh ini memiliki solusi sama
sekali.
\end{eulercomment}
\begin{eulerprompt}
>x=simplex(A,b,c,>max,>check)'
\end{eulerprompt}
\begin{euleroutput}
  [0,  1,  0.5]
\end{euleroutput}
\begin{eulercomment}
Ya, itu sudah ada.

Selanjutnya, kita akan mendefinisikan dua objek. Yang pertama adalah
bidang

\end{eulercomment}
\begin{eulerformula}
\[
a \cdot x \le b
\]
\end{eulerformula}
\begin{eulerprompt}
>function oneplane (a,b,look="") ...
\end{eulerprompt}
\begin{eulerudf}
    return povplane(a,b,look)
  endfunction
\end{eulerudf}
\begin{eulercomment}
Kemudian kita mendefinisikan irisan dari semua ruang setengah dan
sebuah kubus.
\end{eulercomment}
\begin{eulerprompt}
>function adm (A, b, r, look="") ...
\end{eulerprompt}
\begin{eulerudf}
    ol=[];
    loop 1 to rows(A); ol=ol|oneplane(A[#],b[#]); end;
    ol=ol|povbox([0,0,0],[r,r,r]);
    return povintersection(ol,look);
  endfunction
\end{eulerudf}
\begin{eulercomment}
Sekarang kita dapat menggambar adegan tersebut.
\end{eulercomment}
\begin{eulerprompt}
>povstart(angle=120°,center=[0.5,0.5,0.5],zoom=3.5); ...
>writeln(adm(A,b,2,povlook(green,0.4))); ...
>writeAxes(0,1.3,0,1.6,0,1.5); ...
\end{eulerprompt}
\begin{eulercomment}
Berikut adalah lingkaran di sekitar titik optimum.
\end{eulercomment}
\begin{eulerprompt}
>writeln(povintersection([povsphere(x,0.5),povplane(c,c.x')], ...
>  povlook(red,0.9)));
\end{eulerprompt}
\begin{eulercomment}
Dan kesalahan ke arah optimal.
\end{eulercomment}
\begin{eulerprompt}
>writeln(povarrow(x,c*0.5,povlook(red)));
\end{eulerprompt}
\begin{eulercomment}
Kami menambahkan teks ke layar. Teks hanyalah objek 3D. Kita perlu
menempatkan dan memutarnya sesuai dengan pandangan kita.
\end{eulercomment}
\begin{eulerprompt}
>writeln(povtext("Linear Problem",[0,0.2,1.3],size=0.05,rotate=5°)); ...
>povend();
\end{eulerprompt}
\eulerimg{27}{images/Plot3D_Fadhila Asmaul Karimah-089.png}
\eulerheading{Lebih Banyak Contoh}
\begin{eulercomment}
Anda dapat menemukan beberapa contoh Povray di euler di file berikut.

See: Examples/Dandelin Spheres\\
See: Examples/Donat Math\\
See: Examples/Trefoil Knot\\
See: Examples/Optimization by Affine Scaling

\begin{eulercomment}
\eulerheading{Contoh Soal}
\begin{eulercomment}
1. Sketsakan plot 3D untuk fungsi berikut

\end{eulercomment}
\begin{eulerformula}
\[
z= \sqrt{6-x^2-y^2}
\]
\end{eulerformula}
\begin{eulerprompt}
>plot3d("exp((6-x^2-y^2)^1/2)",r=5,n=30,fscale=2,scale=1.2,frame=3):
\end{eulerprompt}
\eulerimg{27}{images/Plot3D_Fadhila Asmaul Karimah-091.png}
\begin{eulercomment}
2. Sketsakan plot kontur untuk fungsi berikut

\end{eulercomment}
\begin{eulerformula}
\[
z=-10\sqrt|xy|
\]
\end{eulerformula}
\begin{eulerprompt}
>plot3d("exp(-10*abs(x*y)^1/2)",r=2,n=100,level="thin", ...
> >contour,>spectral,angle=30°,height=20°):
\end{eulerprompt}
\eulerimg{27}{images/Plot3D_Fadhila Asmaul Karimah-093.png}
\begin{eulercomment}
3. Sketsakan plot 3D berikut

\end{eulercomment}
\begin{eulerformula}
\[
\frac{9x^2-36y^2}{10}
\]
\end{eulerformula}
\begin{eulerprompt}
>load povray;
>defaultpovray="C:\(\backslash\)Program Files\(\backslash\)POV-Ray\(\backslash\)v3.7\(\backslash\)bin\(\backslash\)pvengine.exe"
\end{eulerprompt}
\begin{euleroutput}
  C:\(\backslash\)Program Files\(\backslash\)POV-Ray\(\backslash\)v3.7\(\backslash\)bin\(\backslash\)pvengine.exe
\end{euleroutput}
\begin{eulerprompt}
>pov3d("(9*x^2-36*y^2)/10",zoom=3);
\end{eulerprompt}
\eulerimg{27}{images/Plot3D_Fadhila Asmaul Karimah-095.png}
\begin{eulercomment}
4. Sketsakan peta kontur untuk fungsi berikut

\end{eulercomment}
\begin{eulerformula}
\[
z=\frac{y}{1+x^2+y^2}
\]
\end{eulerformula}
\begin{eulerprompt}
>plot3d("exp(y/(1+x^2+y^2))",r=4,n=100,level="thin", ...
> >contour,>spectral,angle=45°,height=30°):
\end{eulerprompt}
\eulerimg{27}{images/Plot3D_Fadhila Asmaul Karimah-097.png}
\end{eulernotebook}
\end{document}
