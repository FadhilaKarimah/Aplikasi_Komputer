\documentclass{report}
\usepackage[utf8]{inputenc}
\usepackage{eumat}
\usepackage[Conny]{fncychap}
\usepackage[bahasa]{babel}

% Rename Contents
\addto\captionsenglish{\renewcommand{\contentsname}{\vspace{-0.5cm} \textbf{Daftar Isi} \vspace{-2cm}}}

\begin{document}

% Cover Page
\begin{titlepage}
    \begin{center}
        \vspace*{0,2cm}

        \Huge
        \textbf{TUGAS APLIKASI KOMPUTER}
        
        \vspace{1cm}
        
        \LARGE
        BAB LaTeX dan Markdown  
        
        \vspace{1cm}
        
        \includegraphics[width=0.5\textwidth]{images/Logo UNY.png}

        \vspace{1cm}
        
        \textbf{Fadhila Asma'ul Karimah}\\
        22305144031\\
        Matematika B 2022
        
        \vspace{2cm}
        
        \Large
        \textbf{PRODI MATEMATIKA}\\
        \textbf{DEPARTEMEN PENDIDIKAN MATEMATIKA}\\
        \textbf{FAKULTAS MATEMATIKA DAN ILMU PENGETAHUAN ALAM}
        \textbf{UNIVERSITAS NEGERI YOGYAKARTA}\\
        \textbf{2023/2024}
        
    \end{center}
\end{titlepage}

\newpage
\tableofcontents
\chapter{Pekan 2: Pengenalan Software Euler Math Toolbox}
\input{Pekan 2/EMT00-FisrtSteps_Fadhila Asmaul Karimah}
\newpage
\chapter{Pekan 3-4: Penggunaan Software EMT untuk Aljabar}
\input{Pekan 3-4/EMT4aljabar_Fadhila Asma'ul Karimah}
\newpage
\chapter{Pekan 5-6: Penggunaan Software EMT untuk Plot 2D}
\input{Pekan 5-6/EMT4Plot2D_Fadhila Asmaul Karimah}
\newpage
\chapter{Pekan 7-8: Penggunaan Software EMT untuk Plot 3D}
\documentclass{article}

\usepackage{eumat}

\begin{document}
\begin{eulernotebook}
\eulerheading{Menggambar Plot 3D dengan EMT}
\begin{eulercomment}
Modul ini adalah pengenalan plot 3D di Euler. Kita memerlukan plot 3D
untuk memvisualisasikan fungsi dua variabel.

Euler menggambar fungsi tersebut menggunakan algoritma pengurutan
untuk menyembunyikan bagian di latar belakang. Secara umum Euler
menggunakan proyeksi sentral. Defaultnya adalah dari kuadran x-y
positif menuju titik asal x=y=z=0, tetapi sudut=0° dilihat dari arah
sumbu y. Sudut pandang dan ketinggian dapat diubah.

Euler bisa memplot

- permukaan dengan garis bayangan dan level atau rentang level,\\
- titik-titik langit,\\
- kurva parametrik,\\
- permukaan implisit.

Plot 3D suatu fungsi menggunakan plot3d. Cara termudah adalah dengan
memplot ekspresi dalam x dan y. Parameter r mengatur rentang plot
sekitar (0,0).
\end{eulercomment}
\begin{eulerprompt}
>aspect(1.5); plot3d("x^2+sin(y)",-5,5,0,6*pi):
\end{eulerprompt}
\eulerimg{17}{images/Plot3D_Fadhila Asmaul Karimah-001.png}
\begin{eulerprompt}
>plot3d("x^2+x*sin(y)",-5,5,0,6*pi):
\end{eulerprompt}
\eulerimg{17}{images/Plot3D_Fadhila Asmaul Karimah-002.png}
\begin{eulercomment}
Silakan lakukan modifikasi agar gambar "talang bergelombang" tersebut
tidak lurus melainkan melengkung/melingkar, baik melingkar secara
mendatar maupun melingkar turun/naik (seperti papan peluncur pada
kolam renang. Temukan rumusnya.
\end{eulercomment}
\begin{eulerprompt}
>aspect(1); plot3d("8x^2-y^2",-6,8,0,10*pi):
\end{eulerprompt}
\eulerimg{27}{images/Plot3D_Fadhila Asmaul Karimah-003.png}
\begin{eulerprompt}
>reset();
\end{eulerprompt}
\eulerheading{Fungsi Dua Variabel}
\begin{eulercomment}
Untuk grafik suatu fungsi, gunakan

- ekspresi sederhana dalam x dan y,\\
- nama fungsi dari dua variabel,\\
- atau matriks data.

Standarnya adalah kisi-kisi kawat berisi dengan warna berbeda di kedua
sisi. Perhatikan bahwa jumlah interval kisi default adalah 10, tetapi
plot menggunakan jumlah default persegi panjang 40x40 untuk membuat
permukaannya. Ini dapat diubah.

- n=40, n=[40,40]: jumlah garis kisi di setiap arah\\
- grid=10, grid=[10,10]: jumlah garis grid di setiap arah

Kami menggunakan default n=40 dan grid=10.
\end{eulercomment}
\begin{eulerprompt}
>plot3d("x^2+y^2"): 
\end{eulerprompt}
\eulerimg{27}{images/Plot3D_Fadhila Asmaul Karimah-004.png}
\begin{eulercomment}
Interaksi pengguna dimungkinkan dengan parameter \textgreater{} pengguna. Pengguna
dapat menekan tombol berikut.

- kiri, kanan, atas, bawah: memutar sudut pandang\\
- +,-: memperbesar atau memperkecil\\
- a: menghasilkan anaglyph (lihat di bawah)\\
- l : tombol nyalakan sumber cahaya (lihat dibawah)\\
- spasi: reset ke default\\
- kembali: akhiri interaksi\\
x40 untuk membuat permukaannya. Ini dapat diubah.

- n=40, n=[40,40]: jumlah garis kisi di setiap arah\\
- grid=10, grid=[10,10]: jumlah garis grid di setiap arah

Kami menggunakan default n=40 dan grid=10.
\end{eulercomment}
\begin{eulerprompt}
>plot3d("exp(-x^2+y^2)",>user, ...
>  title="Turn with the vector keys (press return to finish)"):
\end{eulerprompt}
\eulerimg{27}{images/Plot3D_Fadhila Asmaul Karimah-005.png}
\begin{eulercomment}
Rentang plot untuk fungsi dapat ditentukan dengan

- a,b: rentang x\\
- c,d: rentang y\\
- r: persegi simetris di sekitar (0,0).\\
- n: jumlah subinterval untuk plot

Ada beberapa parameter untuk membuat skala fungsi atau mengubah
tampilan grafik.

fscale: membuat skala ke nilai fungsi (defaultnya adalah \textless{}fscale).\\
scale: angka atau vektor 1x2 untuk membuat skala ke arah x dan y.\\
frame: jenis bingkai (default 1).
\end{eulercomment}
\begin{eulerprompt}
>plot3d("exp(-(x^2+y^2)/5)",r=8,n=60,fscale=5,scale=1.2,frame=3,>user):
\end{eulerprompt}
\eulerimg{27}{images/Plot3D_Fadhila Asmaul Karimah-006.png}
\begin{eulercomment}
Tampilan dapat diubah dengan berbagai cara yang berbeda.

- distance: jarak pandang ke plot.\\
- zoom: nilai zoom.\\
- angle: sudut terhadap sumbu y negatif dalam radian.\\
- height: tinggi tampilan dalam radian.

Nilai default dapat diperiksa atau diubah dengan fungsi view(). Ini
mengembalikan parameter-parameter dalam urutan di atas.
\end{eulercomment}
\begin{eulerprompt}
>view
\end{eulerprompt}
\begin{euleroutput}
  [5,  2.6,  2,  0.4]
\end{euleroutput}
\begin{eulercomment}
Jarak yang lebih dekat memerlukan zoom yang lebih sedikit. Efeknya
lebih mirip lensa sudut lebar.

Pada contoh berikut, angle=0 dan height=0 dilihat dari sumbu y
negatif. Label sumbu untuk y disembunyikan dalam kasus ini.
\end{eulercomment}
\begin{eulerprompt}
>plot3d("x^2+y",distance=3,zoom=1,angle=pi/2,height=0):
\end{eulerprompt}
\eulerimg{27}{images/Plot3D_Fadhila Asmaul Karimah-007.png}
\begin{eulercomment}
Plot selalu terlihat ke tengah kubus plot. Anda dapat memindahkan
pusat dengan parameter pusat.
\end{eulercomment}
\begin{eulerprompt}
>plot3d("x^4+y^2",a=0,b=1,c=-1,d=1,angle=-20°,height=20°, ...
>  center=[0.4,0,0],zoom=5):
\end{eulerprompt}
\eulerimg{27}{images/Plot3D_Fadhila Asmaul Karimah-008.png}
\begin{eulercomment}
Plot tersebut diubah skala untuk muat ke dalam kubus satuan saat
ditampilkan. Jadi, tidak perlu mengubah jarak atau zoom tergantung
pada ukuran plot. Label-label merujuk pada ukuran yang sebenarnya,
namun.

Jika Anda mematikan ini dengan scale=false, Anda perlu memastikan
bahwa plot masih muat ke dalam jendela plot dengan mengubah jarak
pandang atau zoom, dan memindahkan pusatnya.
\end{eulercomment}
\begin{eulerprompt}
>plot3d("5*exp(-x^2-y^2)",r=2,<fscale,<scale,distance=13,height=50°, ...
>  center=[0,0,-2],frame=3):
\end{eulerprompt}
\eulerimg{27}{images/Plot3D_Fadhila Asmaul Karimah-009.png}
\begin{eulercomment}
Tersedia juga grafik polar. Parameter polar=true menggambar grafik
polar. Fungsi tetap harus menjadi fungsi dari x dan y. Parameter
"fscale" mengubah skala fungsi dengan skala sendiri. Selain itu,
fungsi akan disesuaikan dengan ukuran kubus.
\end{eulercomment}
\begin{eulerprompt}
>plot3d("1/(x^2+y^2+1)",r=5,>polar, ...
>fscale=2,>hue,n=100,zoom=4,>contour,color=blue):
\end{eulerprompt}
\eulerimg{27}{images/Plot3D_Fadhila Asmaul Karimah-010.png}
\begin{eulerprompt}
>function f(r) := exp(-r/2)*cos(r); ...
>plot3d("f(x^2+y^2)",>polar,scale=[1,1,0.4],r=pi,frame=3,zoom=4):
\end{eulerprompt}
\eulerimg{27}{images/Plot3D_Fadhila Asmaul Karimah-011.png}
\begin{eulercomment}
Parameter rotate memutar fungsi dalam sumbu x sekitar sumbu x.

- rotate=1: Menggunakan sumbu x\\
- rotate=2: Menggunakan sumbu z
\end{eulercomment}
\begin{eulerprompt}
>plot3d("x^2+1",a=-1,b=1,rotate=true,grid=5):
\end{eulerprompt}
\eulerimg{27}{images/Plot3D_Fadhila Asmaul Karimah-012.png}
\begin{eulerprompt}
>plot3d("x^2+1",a=-1,b=1,rotate=2,grid=5):
\end{eulerprompt}
\eulerimg{27}{images/Plot3D_Fadhila Asmaul Karimah-013.png}
\begin{eulerprompt}
>plot3d("sqrt(25-x^2)",a=0,b=5,rotate=1):
\end{eulerprompt}
\eulerimg{27}{images/Plot3D_Fadhila Asmaul Karimah-014.png}
\begin{eulerprompt}
>plot3d("x*sin(x)",a=0,b=6pi,rotate=2):
\end{eulerprompt}
\eulerimg{27}{images/Plot3D_Fadhila Asmaul Karimah-015.png}
\begin{eulercomment}
Ini adalah sebuah plot dengan tiga fungsi.
\end{eulercomment}
\begin{eulerprompt}
>plot3d("x","x^2+y^2","y",r=2,zoom=3.5,frame=3):
\end{eulerprompt}
\eulerimg{27}{images/Plot3D_Fadhila Asmaul Karimah-016.png}
\eulerheading{Plot Kontur}
\begin{eulercomment}
Untuk plot ini, Euler menambahkan garis-garis kisi. Sebagai gantinya,
kita bisa menggunakan garis-garis tingkat dan satu warna atau spektrum
warna. Euler dapat menggambar tinggi fungsi pada plot dengan shading.
Dalam semua plot 3D, Euler dapat menghasilkan anaglif merah/cyan.

- \textgreater{}hue: Mengaktifkan shading ringan alih-alih kawat.\\
- \textgreater{}contour: Menampilkan garis kontur otomatis pada plot.\\
- level=... (or levels): Sebuah vektor nilai untuk garis kontur.

Nilai defaultnya adalah level="auto", yang menghitung beberapa garis
kontur secara otomatis. Seperti yang Anda lihat pada plot,
tingkat-tingkat tersebut sebenarnya adalah rentang tingkat.

Gaya default dapat diubah. Untuk plot kontur berikutnya, kita
menggunakan kisi yang lebih halus dengan 100x100 titik, memperbesar
fungsi dan plot, dan mengubah sudut pandang yang berbeda.
\end{eulercomment}
\begin{eulerprompt}
>plot3d("exp(-x^2-y^2)",r=2,n=100,level="thin", ...
> >contour,>spectral,fscale=1,scale=1.1,angle=45°,height=20°):
\end{eulerprompt}
\eulerimg{27}{images/Plot3D_Fadhila Asmaul Karimah-017.png}
\begin{eulerprompt}
>plot3d("exp(x*y)",angle=100°,>contour,color=green):
\end{eulerprompt}
\eulerimg{27}{images/Plot3D_Fadhila Asmaul Karimah-018.png}
\begin{eulercomment}
Pengaturan awal mengggunakan warna abu-abu. Namun, berbagai pilihan
warna spektrum juga tersedia.

- \textgreater{}spectral: Menggunakan skema spektral bawaan\\
- color=...: Menggunakan warna khusus atau skema spektral

Pada plot berikut, kita menggunakan skema spektral bawaan dan
meningkatkan jumlah titik untuk mendapatkan tampilan yang sangat
halus.
\end{eulercomment}
\begin{eulerprompt}
>plot3d("x^2+y^2",>spectral,>contour,n=100):
\end{eulerprompt}
\eulerimg{27}{images/Plot3D_Fadhila Asmaul Karimah-019.png}
\begin{eulercomment}
Daripada garis level otomatis, kita juga dapat mengatur nilai-nilai
garis level. Ini akan menghasilkan garis level yang tipis daripada
rentang level.
\end{eulercomment}
\begin{eulerprompt}
>plot3d("x^2-y^2",0,5,0,5,level=-1:0.1:1,color=redgreen):
\end{eulerprompt}
\eulerimg{27}{images/Plot3D_Fadhila Asmaul Karimah-020.png}
\begin{eulercomment}
Dalam plot berikut, kami menggunakan dua rentang level yang sangat
luas mulai dari -0,1 hingga 1, dan dari 0,9 hingga 1. Ini dimasukkan
sebagai matriks dengan batas level sebagai kolom.

Selain itu, kami melapisi grid dengan 10 interval di setiap arah.
\end{eulercomment}
\begin{eulerprompt}
>plot3d("x^2+y^3",level=[-0.1,0.9;0,1], ...
>  >spectral,angle=30°,grid=10,contourcolor=gray):
\end{eulerprompt}
\eulerimg{27}{images/Plot3D_Fadhila Asmaul Karimah-021.png}
\begin{eulercomment}
Dalam contoh berikut, kami menggambar himpunan, di mana

\end{eulercomment}
\begin{eulerformula}
\[
f(x,y) = x^y-y^x = 0
\]
\end{eulerformula}
\begin{eulercomment}
Kami menggunakan satu garis tipis untuk garis tingkat.
\end{eulercomment}
\begin{eulerprompt}
>plot3d("x^y-y^x",level=0,a=0,b=6,c=0,d=6,contourcolor=red,n=100):
\end{eulerprompt}
\eulerimg{27}{images/Plot3D_Fadhila Asmaul Karimah-023.png}
\begin{eulercomment}
Ini adalah mungkin untuk menampilkan sebuah bidang kontur di bawah
plot. Warna dan jarak dari plot dapat ditentukan.
\end{eulercomment}
\begin{eulerprompt}
>plot3d("x^2+y^4",>cp,cpcolor=green,cpdelta=0.2):
\end{eulerprompt}
\eulerimg{27}{images/Plot3D_Fadhila Asmaul Karimah-024.png}
\begin{eulercomment}
Berikut beberapa gaya lainnya. Kami selalu mematikan bingkai, dan
menggunakan berbagai skema warna untuk plot dan grid.




\end{eulercomment}
\begin{eulerprompt}
>figure(2,2); ...
>expr="y^3-x^2"; ...
>figure(1);  ...
>  plot3d(expr,<frame,>cp,cpcolor=spectral); ...
>figure(2);  ...
>  plot3d(expr,<frame,>spectral,grid=10,cp=2); ...
>figure(3);  ...
>  plot3d(expr,<frame,>contour,color=gray,nc=5,cp=3,cpcolor=greenred); ...
>figure(4);  ...
>  plot3d(expr,<frame,>hue,grid=10,>transparent,>cp,cpcolor=gray); ...
>figure(0):
\end{eulerprompt}
\eulerimg{27}{images/Plot3D_Fadhila Asmaul Karimah-025.png}
\begin{eulercomment}
Ada beberapa skema spektral lainnya, diberi nomor dari 1 hingga 9.
Tetapi Anda juga dapat menggunakan warna=nilai, di mana nilai

- spectral: untuk rentang dari biru hingga merah\\
- white: untuk rentang yang lebih lemah\\
- kuningbiru, unguhijau, birukuning, hijaumerah\\
- birukuning, hijaupurple, kuningbiru, merahhijau
\end{eulercomment}
\begin{eulerprompt}
>figure(3,3); ...
>for i=1:9;  ...
>  figure(i); plot3d("x^2+y^2",spectral=i,>contour,>cp,<frame,zoom=4);  ...
>end; ...
>figure(0):
\end{eulerprompt}
\eulerimg{27}{images/Plot3D_Fadhila Asmaul Karimah-026.png}
\begin{eulercomment}
Sumber cahaya dapat diubah dengan tombol "l" dan kunci kuror selama
interaksi pengguna. Ini juga dapat diatur dengan parameter.

light: arah cahaya\\
amb: cahaya ambien antara 0 dan 1

Perlu diperhatikan bahwa program tidak membedakan antara sisi plot.
Tidak ada bayangan. Untuk ini, Anda memerlukan Povray.
\end{eulercomment}
\begin{eulerprompt}
>plot3d("-x^2-y^2", ...
>  hue=true,light=[0,1,1],amb=0,user=true, ...
>  title="Press l and cursor keys (return to exit)"):
\end{eulerprompt}
\eulerimg{27}{images/Plot3D_Fadhila Asmaul Karimah-027.png}
\begin{eulercomment}
Parameter warna mengubah warna permukaan. Warna garis level juga dapat
diubah.
\end{eulercomment}
\begin{eulerprompt}
>plot3d("-x^2-y^2",color=rgb(0.2,0.2,0),hue=true,frame=false, ...
>  zoom=3,contourcolor=red,level=-2:0.1:1,dl=0.01):
\end{eulerprompt}
\eulerimg{27}{images/Plot3D_Fadhila Asmaul Karimah-028.png}
\begin{eulercomment}
Warna 0 memberikan efek pelangi yang istimewa.
\end{eulercomment}
\begin{eulerprompt}
>plot3d("x^2/(x^2+y^2+1)",color=0,hue=true,grid=10):
\end{eulerprompt}
\eulerimg{27}{images/Plot3D_Fadhila Asmaul Karimah-029.png}
\begin{eulercomment}
Permukaannya juga bisa transparan.
\end{eulercomment}
\begin{eulerprompt}
>plot3d("x^2+y^2",>transparent,grid=10,wirecolor=red):
\end{eulerprompt}
\eulerimg{27}{images/Plot3D_Fadhila Asmaul Karimah-030.png}
\eulerheading{Plot Implisit}
\begin{eulercomment}
Terdapat juga plot implisit dalam tiga dimensi. Euler menghasilkan
potongan melalui objek-objek tersebut. Fitur-fitur dari plot3d
mencakup plot implisit. Plot ini menampilkan himpunan nol dari suatu
fungsi dalam tiga variabel.\\
Solusi dari

\end{eulercomment}
\begin{eulerformula}
\[
f(x, y, z) = 0
\]
\end{eulerformula}
\begin{eulercomment}
dapat divisualisasikan dalam potongan sejajar dengan bidang x-y,
bidang x-z, dan bidang y-z.

implicit=1: potongan sejajar dengan bidang y-z\\
implicit=2: potongan sejajar dengan bidang x-z\\
implicit=4: potongan sejajar dengan bidang x-y

Tambahkan nilai-nilai ini, jika Anda ingin. Dalam contoh ini, kami
memplot


\end{eulercomment}
\begin{eulerformula}
\[
M = \{ (x,y,z) : x^2+y^3+zy=1 \}
\]
\end{eulerformula}
\begin{eulerprompt}
>plot3d("x^2+y^3+z*y-1",r=5,implicit=3):
\end{eulerprompt}
\eulerimg{27}{images/Plot3D_Fadhila Asmaul Karimah-033.png}
\begin{eulerprompt}
>c=1; d=1;
>plot3d("((x^2+y^2-c^2)^2+(z^2-1)^2)*((y^2+z^2-c^2)^2+(x^2-1)^2)*((z^2+x^2-c^2)^2+(y^2-1)^2)-d",r=2,<frame,>implicit,>user): 
\end{eulerprompt}
\eulerimg{27}{images/Plot3D_Fadhila Asmaul Karimah-034.png}
\begin{eulerprompt}
>plot3d("x^2+y^2+4*x*z+z^3",>implicit,r=2,zoom=2.5):
\end{eulerprompt}
\eulerimg{27}{images/Plot3D_Fadhila Asmaul Karimah-035.png}
\eulerheading{Plotting Data 3D}
\begin{eulercomment}
Sama seperti plot2d, plot3d menerima data. Untuk objek 3D, Anda perlu
menyediakan matriks nilai x, y, dan z, atau tiga fungsi atau ekspresi
fx(x, y), fy(x, y), fz(x, y).

\end{eulercomment}
\begin{eulerformula}
\[
\gamma(t,s) = (x(t,s),y(t,s),z(t,s))
\]
\end{eulerformula}
\begin{eulercomment}
Karena x, y, z adalah matriks, kita mengasumsikan bahwa (t, s)
berjalan melalui grid persegi. Sebagai hasilnya, Anda dapat membuat
gambar-gambar persegi panjang dalam ruang.

Anda dapat menggunakan bahasa matriks Euler untuk menghasilkan
koordinat dengan efektif.

Dalam contoh berikut, kita menggunakan vektor nilai t dan vektor kolom
nilai s untuk memparametrikan permukaan bola. Dalam gambaran, kita
dapat menandai wilayah-wilayah, dalam kasus kita, wilayah polar.
\end{eulercomment}
\begin{eulerprompt}
>t=linspace(0,2pi,180); s=linspace(-pi/2,pi/2,90)'; ...
>x=cos(s)*cos(t); y=cos(s)*sin(t); z=sin(s); ...
>plot3d(x,y,z,>hue, ...
>color=blue,<frame,grid=[10,20], ...
>values=s,contourcolor=red,level=[90°-24°;90°-22°], ...
>scale=1.4,height=50°):
\end{eulerprompt}
\eulerimg{27}{images/Plot3D_Fadhila Asmaul Karimah-037.png}
\begin{eulercomment}
Ini adalah contoh, yang merupakan grafik dari sebuah fungsi.
\end{eulercomment}
\begin{eulerprompt}
>t=-1:0.1:1; s=(-1:0.1:1)'; plot3d(t,s,t*s,grid=10):
\end{eulerprompt}
\eulerimg{27}{images/Plot3D_Fadhila Asmaul Karimah-038.png}
\begin{eulercomment}
Namun, kita dapat membuat berbagai jenis permukaan. Berikut ini adalah
permukaan yang sama sebagai fungsi

\end{eulercomment}
\begin{eulerformula}
\[
x = y \, z
\]
\end{eulerformula}
\begin{eulerprompt}
>plot3d(t*s,t,s,angle=180°,grid=10):
\end{eulerprompt}
\eulerimg{27}{images/Plot3D_Fadhila Asmaul Karimah-040.png}
\begin{eulercomment}
Dengan lebih banyak usaha, kita dapat menghasilkan banyak permukaan.

Pada contoh berikut, kita membuat tampilan berbayang dari sebuah bola
yang distorsi. Koordinat biasa untuk bola tersebut adalah

\end{eulercomment}
\begin{eulerformula}
\[
\gamma(t,s) = (\cos(t)\cos(s),\sin(t)\sin(s),\cos(s))
\]
\end{eulerformula}
\begin{eulercomment}
dengan

\end{eulercomment}
\begin{eulerformula}
\[
0 \le t \le 2\pi, \quad \frac{-\pi}{2} \le s \le \frac{\pi}{2}.
\]
\end{eulerformula}
\begin{eulercomment}
Kita mengdistorsi ini dengan faktor

\end{eulercomment}
\begin{eulerformula}
\[
d(t,s) = \frac{\cos(4t)+\cos(8s)}{4}.
\]
\end{eulerformula}
\begin{eulerprompt}
>t=linspace(0,2pi,320); s=linspace(-pi/2,pi/2,160)'; ...
>d=1+0.2*(cos(4*t)+cos(8*s)); ...
>plot3d(cos(t)*cos(s)*d,sin(t)*cos(s)*d,sin(s)*d,hue=1, ...
>  light=[1,0,1],frame=0,zoom=5):
\end{eulerprompt}
\eulerimg{27}{images/Plot3D_Fadhila Asmaul Karimah-044.png}
\begin{eulercomment}
Tentu saja, awan titik juga mungkin. Untuk menggambarkan data titik
dalam ruang, kita memerlukan tiga vektor untuk koordinat titik-titik
tersebut.

Gaya-gaya tersebut sama seperti dalam plot2d dengan points=true;
\end{eulercomment}
\begin{eulerprompt}
>n=500;  ...
>  plot3d(normal(1,n),normal(1,n),normal(1,n),points=true,style="."):
\end{eulerprompt}
\eulerimg{27}{images/Plot3D_Fadhila Asmaul Karimah-045.png}
\begin{eulercomment}
Ini juga memungkinkan untuk menggambar kurva dalam tiga dimensi (3D).
Dalam hal ini, lebih mudah untuk menghitung sebelumnya titik-titik
dari kurva tersebut. Untuk kurva-kurva dalam bidang, kita menggunakan
urutan koordinat dan parameter wire=true.
\end{eulercomment}
\begin{eulerprompt}
>t=linspace(0,8pi,500); ...
>plot3d(sin(t),cos(t),t/10,>wire,zoom=3):
\end{eulerprompt}
\eulerimg{27}{images/Plot3D_Fadhila Asmaul Karimah-046.png}
\begin{eulerprompt}
>t=linspace(0,4pi,1000); plot3d(cos(t),sin(t),t/2pi,>wire, ...
>linewidth=3,wirecolor=blue):
\end{eulerprompt}
\eulerimg{27}{images/Plot3D_Fadhila Asmaul Karimah-047.png}
\begin{eulerprompt}
>X=cumsum(normal(3,100)); ...
> plot3d(X[1],X[2],X[3],>anaglyph,>wire):
\end{eulerprompt}
\eulerimg{27}{images/Plot3D_Fadhila Asmaul Karimah-048.png}
\begin{eulercomment}
EMT juga dapat membuat plot dalam mode anaglyph. Untuk melihat plot
tersebut, Anda memerlukan kacamata merah/biru.
\end{eulercomment}
\begin{eulerprompt}
> plot3d("x^2+y^3",>anaglyph,>contour,angle=30°):
\end{eulerprompt}
\eulerimg{27}{images/Plot3D_Fadhila Asmaul Karimah-049.png}
\begin{eulercomment}
Seringkali, skema warna spektral digunakan untuk grafik ini. Ini
menekankan tinggi fungsi tersebut.
\end{eulercomment}
\begin{eulerprompt}
>plot3d("x^2*y^3-y",>spectral,>contour,zoom=3.2):
\end{eulerprompt}
\eulerimg{27}{images/Plot3D_Fadhila Asmaul Karimah-050.png}
\begin{eulercomment}
Euler juga dapat menggambar permukaan-parameterkan ketika
parameter-parameter tersebut adalah nilai-nilai x-, y-, dan z- dari
gambar grid berbentuk persegi panjang di dalam ruang.

Untuk demonstrasi berikutnya, kami menyiapkan parameter u dan v, dan
menghasilkan koordinat ruang dari kedua parameter tersebut.

\end{eulercomment}
\begin{eulerprompt}
>u=linspace(-1,1,10); v=linspace(0,2*pi,50)'; ...
>X=(3+u*cos(v/2))*cos(v); Y=(3+u*cos(v/2))*sin(v); Z=u*sin(v/2); ...
>plot3d(X,Y,Z,>anaglyph,<frame,>wire,scale=2.3):
\end{eulerprompt}
\eulerimg{27}{images/Plot3D_Fadhila Asmaul Karimah-051.png}
\begin{eulercomment}
Berikut contoh yang lebih rumit, yang megah dengan kacamata
merah/cyan.
\end{eulercomment}
\begin{eulerprompt}
>u:=linspace(-pi,pi,160); v:=linspace(-pi,pi,400)';  ...
>x:=(4*(1+.25*sin(3*v))+cos(u))*cos(2*v); ...
>y:=(4*(1+.25*sin(3*v))+cos(u))*sin(2*v); ...
> z=sin(u)+2*cos(3*v); ...
>plot3d(x,y,z,frame=0,scale=1.5,hue=1,light=[1,0,-1],zoom=2.8,>anaglyph):
\end{eulerprompt}
\eulerimg{27}{images/Plot3D_Fadhila Asmaul Karimah-052.png}
\eulerheading{Plot Statistik}
\begin{eulercomment}
Grafik batang juga mungkin. Untuk ini, kita harus memberikan:

- x: vektor baris dengan n+1 elemen\\
- y: vektor kolom dengan n+1 elemen\\
- z: matriks nxn dari nilai-nilai.

z bisa lebih besar, tetapi hanya nilai-nilai nxn yang akan digunakan.

Dalam contoh ini, kita pertama-tama menghitung nilai-nilai. Kemudian
kita menyesuaikan x dan y, sehingga vektor-vektor tersebut berpusat
pada nilai yang digunakan.
\end{eulercomment}
\begin{eulerprompt}
>x=-1:0.1:1; y=x'; z=x^2+y^2; ...
>xa=(x|1.1)-0.05; ya=(y_1.1)-0.05; ...
>plot3d(xa,ya,z,bar=true):
\end{eulerprompt}
\eulerimg{27}{images/Plot3D_Fadhila Asmaul Karimah-053.png}
\begin{eulercomment}
Mungkin untuk membagi plot permukaan menjadi dua atau lebih bagian.

\end{eulercomment}
\begin{eulerprompt}
>x=-1:0.1:1; y=x'; z=x+y; d=zeros(size(x)); ...
>plot3d(x,y,z,disconnect=2:2:20):
\end{eulerprompt}
\eulerimg{27}{images/Plot3D_Fadhila Asmaul Karimah-054.png}
\begin{eulercomment}
Jika Anda memuat atau menghasilkan matriks data M dari sebuah file dan
perlu membuat plotnya dalam 3D, Anda dapat melakukan penskalaan pada
matriks tersebut menjadi rentang [-1,1] dengan menggunakan perintah
"scale(M)", atau melakukan penskalaan dengan menggunakan "zscale". Ini
dapat dikombinasikan dengan faktor-faktor penskalaan individu yang
diterapkan secara tambahan.
\end{eulercomment}
\begin{eulerprompt}
>i=1:20; j=i'; ...
>plot3d(i*j^2+100*normal(20,20),>zscale,scale=[1,1,1.5],angle=-40°,zoom=1.8):
\end{eulerprompt}
\eulerimg{27}{images/Plot3D_Fadhila Asmaul Karimah-055.png}
\begin{eulerprompt}
>Z=intrandom(5,100,6); v=zeros(5,6); ...
>loop 1 to 5; v[#]=getmultiplicities(1:6,Z[#]); end; ...
>columnsplot3d(v',scols=1:5,ccols=[1:5]):
\end{eulerprompt}
\eulerimg{27}{images/Plot3D_Fadhila Asmaul Karimah-056.png}
\eulerheading{Permukaan Benda Putar}
\begin{eulerprompt}
>plot2d("(x^2+y^2-1)^3-x^2*y^3",r=1.3, ...
>style="#",color=red,<outline, ...
>level=[-2;0],n=100):
\end{eulerprompt}
\eulerimg{27}{images/Plot3D_Fadhila Asmaul Karimah-057.png}
\begin{eulerprompt}
>ekspresi &= (x^2+y^2-1)^3-x^2*y^3; $ekspresi
\end{eulerprompt}
\begin{eulerformula}
\[
\left(y^2+x^2-1\right)^3-x^2\,y^3
\]
\end{eulerformula}
\begin{eulercomment}
Kami ingin memutar kurva hati sekitar sumbu y. Berikut adalah ekspresi
yang mendefinisikan bentuk hati:

\end{eulercomment}
\begin{eulerformula}
\[
f(x,y)=(x^2+y^2-1)^3-x^2.y^3.
\]
\end{eulerformula}
\begin{eulercomment}
Selanjutnya kita menetapkan

\end{eulercomment}
\begin{eulerformula}
\[
x=r.cos(a),\quad y=r.sin(a).
\]
\end{eulerformula}
\begin{eulerprompt}
>function fr(r,a) &= ekspresi with [x=r*cos(a),y=r*sin(a)] | trigreduce; $fr(r,a)
\end{eulerprompt}
\begin{eulerformula}
\[
\left(r^2-1\right)^3+\frac{\left(\sin \left(5\,a\right)-\sin \left(  3\,a\right)-2\,\sin a\right)\,r^5}{16}
\]
\end{eulerformula}
\begin{eulercomment}
Ini memungkinkan untuk mendefinisikan fungsi numerik, yang memecahkan
untuk r, jika a diberikan. Dengan fungsi itu, kita dapat menggambar
hati yang berputar sebagai permukaan parametrik.
\end{eulercomment}
\begin{eulerprompt}
>function map f(a) := bisect("fr",0,2;a); ...
>t=linspace(-pi/2,pi/2,100); r=f(t);  ...
>s=linspace(pi,2pi,100)'; ...
>plot3d(r*cos(t)*sin(s),r*cos(t)*cos(s),r*sin(t), ...
>>hue,<frame,color=red,zoom=4,amb=0,max=0.7,grid=12,height=50°):
\end{eulerprompt}
\eulerimg{27}{images/Plot3D_Fadhila Asmaul Karimah-062.png}
\begin{eulercomment}
Berikut adalah plot 3D dari gambar di atas yang diputar sekitar sumbu
z. Kami mendefinisikan fungsi yang menggambarkan objek tersebut.
\end{eulercomment}
\begin{eulerprompt}
>function f(x,y,z) ...
\end{eulerprompt}
\begin{eulerudf}
  r=x^2+y^2;
  return (r+z^2-1)^3-r*z^3;
   endfunction
\end{eulerudf}
\begin{eulerprompt}
>plot3d("f(x,y,z)", ...
>xmin=0,xmax=1.2,ymin=-1.2,ymax=1.2,zmin=-1.2,zmax=1.4, ...
>implicit=1,angle=-30°,zoom=2.5,n=[10,100,60],>anaglyph):
\end{eulerprompt}
\eulerimg{27}{images/Plot3D_Fadhila Asmaul Karimah-063.png}
\eulerheading{Plot 3D Khusus}
\begin{eulercomment}
Fungsi plot3d bagus untuk dimiliki, tetapi tidak memenuhi semua
kebutuhan. Selain rutinitas yang lebih dasar, mungkin Anda bisa
mendapatkan plot bingkai dari objek apa pun yang Anda sukai.

Meskipun Euler bukan program 3D, itu dapat menggabungkan beberapa
objek dasar. Kami mencoba untuk memvisualisasikan sebuah parabola dan
tangennya.
\end{eulercomment}
\begin{eulerprompt}
>function myplot ...
\end{eulerprompt}
\begin{eulerudf}
    y=-1:0.01:1; x=(-1:0.01:1)';
    plot3d(x,y,0.2*(x-0.1)/2,<scale,<frame,>hue, ..
      hues=0.5,>contour,color=orange);
    h=holding(1);
    plot3d(x,y,(x^2+y^2)/2,<scale,<frame,>contour,>hue);
    holding(h);
  endfunction
\end{eulerudf}
\begin{eulercomment}
Sekarang framedplot() menyediakan bingkai, dan mengatur tampilan.
\end{eulercomment}
\begin{eulerprompt}
>framedplot("myplot",[-1,1,-1,1,0,1],height=0,angle=-30°, ...
>  center=[0,0,-0.7],zoom=3):
\end{eulerprompt}
\eulerimg{27}{images/Plot3D_Fadhila Asmaul Karimah-064.png}
\begin{eulercomment}
Dengan cara yang sama, Anda dapat menggambar bidang kontur secara
manual. Perhatikan bahwa plot3d() mengatur jendela ke fullwindow()
secara default, tetapi plotcontourplane() mengasumsikan hal tersebut.
\end{eulercomment}
\begin{eulerprompt}
>x=-1:0.02:1.1; y=x'; z=x^2-y^4;
>function myplot (x,y,z) ...
\end{eulerprompt}
\begin{eulerudf}
    zoom(2);
    wi=fullwindow();
    plotcontourplane(x,y,z,level="auto",<scale);
    plot3d(x,y,z,>hue,<scale,>add,color=white,level="thin");
    window(wi);
    reset();
  endfunction
\end{eulerudf}
\begin{eulerprompt}
>myplot(x,y,z):
\end{eulerprompt}
\eulerimg{27}{images/Plot3D_Fadhila Asmaul Karimah-065.png}
\eulerheading{Animasi}
\begin{eulercomment}
Euler dapat menggunakan bingkai (frames) untuk pra-menghitung animasi.

Salah satu fungsi yang menggunakan teknik ini adalah fungsi rotate.
Ini dapat mengubah sudut pandang dan menggambar ulang plot 3D. Fungsi
tersebut memanggil addpage() untuk setiap plot baru. Akhirnya, ia
menganimasikan plot-plot tersebut.

Silakan pelajari sumber kode fungsi rotate untuk melihat lebih banyak
detailnya.
\end{eulercomment}
\begin{eulerprompt}
>function testplot () := plot3d("x^2+y^3"); ...
>rotate("testplot"); testplot():
\end{eulerprompt}
\eulerimg{27}{images/Plot3D_Fadhila Asmaul Karimah-066.png}
\eulerheading{Menggambar Povray}
\begin{eulercomment}
Dengan bantuan berkas Euler povray.e, Euler dapat menghasilkan berkas
Povray. Hasilnya sangat bagus untuk dilihat.

Anda perlu menginstal Povray (32bit atau 64bit) dari
http://www.povray.org/,\\
dan letakkan sub-direktori "bin" dari Povray ke dalam path lingkungan,
atau atur variabel "defaultpovray" dengan path lengkap yang menunjuk
ke "pvengine.exe".

Antarmuka Povray Euler menghasilkan berkas Povray di direktori rumah
pengguna, dan memanggil Povray untuk menguraikan berkas-berkas ini.
Nama berkas default adalah current.pov, dan direktori default adalah
eulerhome(), biasanya c:\textbackslash{}Users\textbackslash{}Username\textbackslash{}Euler. Povray menghasilkan
berkas PNG, yang dapat dimuat oleh Euler ke dalam buku catatan. Untuk
membersihkan berkas-berkas ini, gunakan povclear().

Fungsi pov3d berada dalam semangat yang sama dengan plot3d. Ini dapat
menghasilkan grafik dari fungsi f(x,y), atau permukaan dengan
koordinat X,Y,Z dalam matriks, termasuk garis-garis level opsional.
Fungsi ini akan memulai raytracer secara otomatis, dan memuat adegan
ke dalam buku catatan Euler.

Selain pov3d(), ada banyak fungsi lain yang menghasilkan objek Povray.
Fungsi-fungsi ini mengembalikan string yang berisi kode Povray untuk
objek-objek tersebut. Untuk menggunakan fungsi-fungsi ini, mulai
berkas Povray dengan povstart(). Kemudian gunakan writeln(...) untuk
menulis objek-objek ke berkas adegan. Akhirnya, akhiri berkas dengan
povend(). Secara default, raytracer akan mulai, dan PNG akan
dimasukkan ke dalam buku catatan Euler.

Fungsi objek memiliki parameter bernama "look", yang memerlukan string
dengan kode Povray untuk tekstur dan penyelesaian objek tersebut.
Fungsi povlook() dapat digunakan untuk menghasilkan string ini. Ini
memiliki parameter untuk warna, transparansi, Phong Shading, dll.

Perlu diingat bahwa alam semesta Povray memiliki sistem koordinat yang
berbeda. Antarmuka ini menerjemahkan semua koordinat ke sistem Povray.
Jadi Anda dapat terus berpikir dalam sistem koordinat Euler dengan z
mengarah secara vertikal ke atas, dan sumbu x,y,z sesuai dengan tangan
kanan.

Anda perlu memuat berkas povray.
\end{eulercomment}
\begin{eulerprompt}
>load povray;
\end{eulerprompt}
\begin{eulercomment}
Pastikan direktori bin Povray ada dalam path. Jika tidak, edit
variabel berikut agar berisi path ke eksekusi povray.
\end{eulercomment}
\begin{eulerprompt}
>defaultpovray="C:\(\backslash\)Program Files\(\backslash\)POV-Ray\(\backslash\)v3.7\(\backslash\)bin\(\backslash\)pvengine.exe"
\end{eulerprompt}
\begin{euleroutput}
  C:\(\backslash\)Program Files\(\backslash\)POV-Ray\(\backslash\)v3.7\(\backslash\)bin\(\backslash\)pvengine.exe
\end{euleroutput}
\begin{eulercomment}
Untuk kesan pertama, kami menggambar fungsi sederhana. Perintah
berikut menghasilkan file povray di direktori pengguna Anda, dan
menjalankan Povray untuk pelacakan sinar file ini.

Jika Anda memulai perintah berikut, GUI Povray seharusnya terbuka,
menjalankan file, dan menutup secara otomatis. Karena alasan keamanan,
Anda akan ditanyai apakah Anda ingin mengizinkan file exe ini untuk
berjalan. Anda dapat menekan batal untuk menghentikan pertanyaan lebih
lanjut. Anda mungkin harus menekan OK di jendela Povray untuk mengakui
dialog awal Povray.
\end{eulercomment}
\begin{eulerprompt}
>plot3d("x^2+y^2",zoom=2):
\end{eulerprompt}
\eulerimg{27}{images/Plot3D_Fadhila Asmaul Karimah-067.png}
\begin{eulerprompt}
>pov3d("x^2+y^2",zoom=3);
\end{eulerprompt}
\eulerimg{27}{images/Plot3D_Fadhila Asmaul Karimah-068.png}
\begin{eulercomment}
Kita dapat membuat fungsi tersebut transparan dan menambahkan yang
lainnya. Kita juga dapat menambahkan garis level pada plot fungsi.
\end{eulercomment}
\begin{eulerprompt}
>pov3d("x^2+y^3",axiscolor=red,angle=-45°,>anaglyph, ...
>  look=povlook(cyan,0.2),level=-1:0.5:1,zoom=3.8);
\end{eulerprompt}
\eulerimg{27}{images/Plot3D_Fadhila Asmaul Karimah-069.png}
\begin{eulercomment}
Kadang-kadang perlu untuk mencegah penskalaan fungsi, dan penskalaan
fungsi secara manual.

Kita menggambar himpunan titik-titik dalam bidang kompleks, di mana
hasil kali jarak ke 1 dan -1 sama dengan 1.
\end{eulercomment}
\begin{eulerprompt}
>pov3d("((x-1)^2+y^2)*((x+1)^2+y^2)/40",r=2, ...
>  angle=-120°,level=1/40,dlevel=0.005,light=[-1,1,1],height=10°,n=50, ...
>  <fscale,zoom=3.8);
\end{eulerprompt}
\eulerimg{27}{images/Plot3D_Fadhila Asmaul Karimah-070.png}
\eulerheading{Plotting dengan Koordinat}
\begin{eulercomment}
Daripada menggunakan fungsi, kita dapat melakukan plotting dengan
koordinat. Seperti pada plot3d, kita memerlukan tiga matriks untuk
mendefinisikan objek tersebut.

Pada contoh ini, kita memutar sebuah fungsi sekitar sumbu z.
\end{eulercomment}
\begin{eulerprompt}
>function f(x) := x^3-x+1; ...
>x=-1:0.01:1; t=linspace(0,2pi,50)'; ...
>Z=x; X=cos(t)*f(x); Y=sin(t)*f(x); ...
>pov3d(X,Y,Z,angle=40°,look=povlook(green,0.1),height=20°,axis=0,zoom=4,light=[10,-5,5]);
\end{eulerprompt}
\eulerimg{27}{images/Plot3D_Fadhila Asmaul Karimah-071.png}
\begin{eulercomment}
Dalam contoh berikut, kita menggambar gelombang teredam. Kami
menghasilkan gelombang tersebut dengan bahasa matriks Euler.

Kami juga menunjukkan bagaimana objek tambahan dapat ditambahkan ke
dalam adegan pov3d. Untuk pembuatan objek, lihat contoh-contoh
berikut. Perhatikan bahwa plot3d akan menyesuaikan skala plot sehingga
cocok dalam kubus satuan.
\end{eulercomment}
\begin{eulerprompt}
>r=linspace(0,1,80); phi=linspace(0,2pi,80)'; ...
>x=r*cos(phi); y=r*sin(phi); z=exp(-5*r)*cos(8*pi*r)/3;  ...
>pov3d(x,y,z,zoom=5,axis=0,height=30°,add=povsphere([0,0,0.5],0.1,povlook(red)), ...
>  w=500,h=300);
\end{eulerprompt}
\eulerimg{16}{images/Plot3D_Fadhila Asmaul Karimah-072.png}
\begin{eulercomment}
Dengan metode shading canggih dari Povray, sangat sedikit titik dapat
menghasilkan permukaan yang sangat halus. Hanya di batas-batas dan
dalam bayangan trik ini mungkin menjadi jelas.

Untuk ini, kita perlu menambahkan vektor normal di setiap titik
matriks.
\end{eulercomment}
\begin{eulerprompt}
>Z &= x^2*y^3
\end{eulerprompt}
\begin{euleroutput}
  
                                   2  3
                                  x  y
  
\end{euleroutput}
\begin{eulercomment}
Persamaan dari permukaannya adalah [x, y, Z]. Kami menghitung dua
turunan terhadap x dan y dari ini dan mengambil hasil perkalian silang
sebagai normalnya.
\end{eulercomment}
\begin{eulerprompt}
>dx &= diff([x,y,Z],x); dy &= diff([x,y,Z],y);
\end{eulerprompt}
\begin{eulercomment}
Kami mendefinisikan normal sebagai hasil perkalian silang dari
turunan-turunan ini, dan mendefinisikan fungsi koordinat.
\end{eulercomment}
\begin{eulerprompt}
>N &= crossproduct(dx,dy); NX &= N[1]; NY &= N[2]; NZ &= N[3]; N,
\end{eulerprompt}
\begin{euleroutput}
  
                                 3       2  2
                         [- 2 x y , - 3 x  y , 1]
  
\end{euleroutput}
\begin{eulercomment}
Kami hanya menggunakan 25 poin.
\end{eulercomment}
\begin{eulerprompt}
>x=-1:0.5:1; y=x';
>pov3d(x,y,Z(x,y),angle=10°, ...
>  xv=NX(x,y),yv=NY(x,y),zv=NZ(x,y),<shadow);
\end{eulerprompt}
\eulerimg{27}{images/Plot3D_Fadhila Asmaul Karimah-073.png}
\begin{eulercomment}
Berikut adalah simpul Trefoil yang dibuat oleh A. Busser dalam Povray.
Terdapat versi yang diperbaiki dari ini dalam contoh-contoh.

See: Examples\textbackslash{}Trefoil Knot \textbar{} Trefoil Knot

Untuk tampilan yang bagus dengan tidak terlalu banyak titik, kami
menambahkan vektor normal di sini. Kami menggunakan Maxima untuk
menghitung vektor normal untuk kami. Pertama, tiga fungsi untuk
koordinat sebagai ekspresi simbolik.
\end{eulercomment}
\begin{eulerprompt}
>X &= ((4+sin(3*y))+cos(x))*cos(2*y); ...
>Y &= ((4+sin(3*y))+cos(x))*sin(2*y); ...
>Z &= sin(x)+2*cos(3*y);
\end{eulerprompt}
\begin{eulercomment}
Kemudian turunan dua vektor terhadap x dan y.
\end{eulercomment}
\begin{eulerprompt}
>dx &= diff([X,Y,Z],x); dy &= diff([X,Y,Z],y);
\end{eulerprompt}
\begin{eulercomment}
Sekarang yang normal, yang merupakan hasil perkalian silang dari kedua
turunan tersebut.
\end{eulercomment}
\begin{eulerprompt}
>dn &= crossproduct(dx,dy);
\end{eulerprompt}
\begin{eulercomment}
Sekarang kita mengevaluasi semua ini secara numerik.
\end{eulercomment}
\begin{eulerprompt}
>x:=linspace(-%pi,%pi,40); y:=linspace(-%pi,%pi,100)';
\end{eulerprompt}
\begin{eulercomment}
Vektor normal adalah hasil evaluasi dari ekspresi simbolis dn[i] untuk
i=1,2,3. Syntax untuk ini adalah \&"expression"(parameter). Ini
merupakan alternatif dari metode pada contoh sebelumnya, di mana kita
mendefinisikan ekspresi simbolis NX, NY, NZ terlebih dahulu.
\end{eulercomment}
\begin{eulerprompt}
>pov3d(X(x,y),Y(x,y),Z(x,y),>anaglyph,axis=0,zoom=5,w=450,h=350, ...
>  <shadow,look=povlook(blue), ...
>  xv=&"dn[1]"(x,y), yv=&"dn[2]"(x,y), zv=&"dn[3]"(x,y));
\end{eulerprompt}
\eulerimg{21}{images/Plot3D_Fadhila Asmaul Karimah-074.png}
\begin{eulercomment}
Kita juga dapat membuat grid dalam 3D.
\end{eulercomment}
\begin{eulerprompt}
>povstart(zoom=4); ...
>x=-1:0.5:1; r=1-(x+1)^2/6; ...
>t=(0°:30°:360°)'; y=r*cos(t); z=r*sin(t); ...
>writeln(povgrid(x,y,z,d=0.02,dballs=0.05)); ...
>povend();
\end{eulerprompt}
\eulerimg{27}{images/Plot3D_Fadhila Asmaul Karimah-075.png}
\begin{eulercomment}
Dengan povgrid(), kurva-kurva menjadi mungkin.
\end{eulercomment}
\begin{eulerprompt}
>povstart(center=[0,0,1],zoom=3.6); ...
>t=linspace(0,2,1000); r=exp(-t); ...
>x=cos(2*pi*10*t)*r; y=sin(2*pi*10*t)*r; z=t; ...
>writeln(povgrid(x,y,z,povlook(red))); ...
>writeAxis(0,2,axis=3); ...
>povend();
\end{eulerprompt}
\eulerimg{27}{images/Plot3D_Fadhila Asmaul Karimah-076.png}
\eulerheading{Objek Povray}
\begin{eulercomment}
Di atas, kami menggunakan pov3d untuk memplot permukaan. Antarmuka
povray dalam Euler juga dapat menghasilkan objek Povray. Objek-objek
ini disimpan sebagai string dalam Euler, dan perlu ditulis ke file
Povray.

Kami memulai output dengan povstart().
\end{eulercomment}
\begin{eulerprompt}
>povstart(zoom=4);
\end{eulerprompt}
\begin{eulercomment}
Pertama, kita mendefinisikan tiga silinder dan menyimpannya dalam
bentuk string dalam Euler.

Fungsi-fungsi seperti povx() hanya mengembalikan vektor [1,0,0], yang
dapat digunakan sebagai penggantinya.
\end{eulercomment}
\begin{eulerprompt}
>c1=povcylinder(-povx,povx,1,povlook(red)); ...
>c2=povcylinder(-povy,povy,1,povlook(yellow)); ...
>c3=povcylinder(-povz,povz,1,povlook(blue)); ...
\end{eulerprompt}
\begin{eulercomment}
Kalimat-kalimat tersebut berisi kode Povray, yang pada saat itu tidak
perlu kita pahami.
\end{eulercomment}
\begin{eulerprompt}
>c2
\end{eulerprompt}
\begin{euleroutput}
  cylinder \{ <0,0,-1>, <0,0,1>, 1
   texture \{ pigment \{ color rgb <0.941176,0.941176,0.392157> \}  \} 
   finish \{ ambient 0.2 \} 
   \}
\end{euleroutput}
\begin{eulercomment}
Seperti yang Anda lihat, kami menambahkan tekstur pada objek dalam
tiga warna yang berbeda.

Hal ini dilakukan dengan menggunakan povlook(), yang mengembalikan
string dengan kode Povray yang relevan. Kami dapat menggunakan warna
Euler default, atau mendefinisikan warna sendiri. Kami juga dapat
menambahkan transparansi, atau mengubah cahaya ambien.
\end{eulercomment}
\begin{eulerprompt}
>povlook(rgb(0.1,0.2,0.3),0.1,0.5)
\end{eulerprompt}
\begin{euleroutput}
   texture \{ pigment \{ color rgbf <0.101961,0.2,0.301961,0.1> \}  \} 
   finish \{ ambient 0.5 \} 
  
\end{euleroutput}
\begin{eulercomment}
Sekarang kita mendefinisikan sebuah objek persimpangan, dan menulis
hasilnya ke dalam file.
\end{eulercomment}
\begin{eulerprompt}
>writeln(povintersection([c1,c2,c3]));
\end{eulerprompt}
\begin{eulercomment}
Persimpangan tiga silinder sulit untuk dibayangkan, jika Anda belum
pernah melihatnya sebelumnya.
\end{eulercomment}
\begin{eulerprompt}
>povend;
\end{eulerprompt}
\eulerimg{27}{images/Plot3D_Fadhila Asmaul Karimah-077.png}
\begin{eulercomment}
Berikut ini adalah fungsi-fungsi yang menghasilkan fraktal secara
rekursif.

Fungsi pertama menunjukkan bagaimana Euler mengatasi objek Povray
sederhana. Fungsi povbox() mengembalikan sebuah string yang berisi
koordinat kotak, tekstur, dan penyelesaian.
\end{eulercomment}
\begin{eulerprompt}
>function onebox(x,y,z,d) := povbox([x,y,z],[x+d,y+d,z+d],povlook());
>function fractal (x,y,z,h,n) ...
\end{eulerprompt}
\begin{eulerudf}
   if n==1 then writeln(onebox(x,y,z,h));
   else
     h=h/3;
     fractal(x,y,z,h,n-1);
     fractal(x+2*h,y,z,h,n-1);
     fractal(x,y+2*h,z,h,n-1);
     fractal(x,y,z+2*h,h,n-1);
     fractal(x+2*h,y+2*h,z,h,n-1);
     fractal(x+2*h,y,z+2*h,h,n-1);
     fractal(x,y+2*h,z+2*h,h,n-1);
     fractal(x+2*h,y+2*h,z+2*h,h,n-1);
     fractal(x+h,y+h,z+h,h,n-1);
   endif;
  endfunction
\end{eulerudf}
\begin{eulerprompt}
>povstart(fade=10,<shadow);
>fractal(-1,-1,-1,2,4);
>povend();
\end{eulerprompt}
\eulerimg{27}{images/Plot3D_Fadhila Asmaul Karimah-078.png}
\begin{eulercomment}
Perbedaan memungkinkan pemisahan satu objek dari yang lain. Seperti
perpotongan, itu merupakan bagian dari objek CSG dalam Povray.
\end{eulercomment}
\begin{eulerprompt}
>povstart(light=[5,-5,5],fade=10);
\end{eulerprompt}
\begin{eulercomment}
Untuk demonstrasi ini, kita mendefinisikan sebuah objek dalam Povray,
daripada menggunakan sebuah string dalam Euler. Definisi tersebut
langsung ditulis ke dalam file.

Koordinat sebuah kotak dengan nilai -1 hanya berarti [-1,-1,-1].
\end{eulercomment}
\begin{eulerprompt}
>povdefine("mycube",povbox(-1,1));
\end{eulerprompt}
\begin{eulercomment}
Kita dapat menggunakan objek ini dalam povobject(), yang mengembalikan
sebuah string seperti biasanya.
\end{eulercomment}
\begin{eulerprompt}
>c1=povobject("mycube",povlook(red));
\end{eulerprompt}
\begin{eulercomment}
Kami menghasilkan sebuah kubus kedua, lalu memutarnya dan
membesarkannya sedikit.
\end{eulercomment}
\begin{eulerprompt}
>c2=povobject("mycube",povlook(yellow),translate=[1,1,1], ...
>  rotate=xrotate(10°)+yrotate(10°), scale=1.2);
\end{eulerprompt}
\begin{eulercomment}
Kemudian kita mengambil perbedaan dari kedua objek tersebut.
\end{eulercomment}
\begin{eulerprompt}
>writeln(povdifference(c1,c2));
\end{eulerprompt}
\begin{eulercomment}
Sekarang tambahkan tiga sumbu.
\end{eulercomment}
\begin{eulerprompt}
>writeAxis(-1.2,1.2,axis=1); ...
>writeAxis(-1.2,1.2,axis=2); ...
>writeAxis(-1.2,1.2,axis=4); ...
>povend();
\end{eulerprompt}
\eulerimg{27}{images/Plot3D_Fadhila Asmaul Karimah-079.png}
\eulerheading{Fungsi Implisit}
\begin{eulercomment}
Povray dapat menggambar himpunan di mana f(x, y, z) = 0, seperti
parameter implisit dalam plot3d. Hasilnya terlihat jauh lebih baik,
namun sintaks untuk fungsi ini agak berbeda. Anda tidak dapat
menggunakan keluaran dari Maxima atau ekspresi Euler.

\end{eulercomment}
\begin{eulerformula}
\[
((x^2+y^2-c^2)^2+(z^2-1)^2)*((y^2+z^2-c^2)^2+(x^2-1)^2)*((z^2+x^2-c^2)^2+(y^2-1)^2)=d
\]
\end{eulerformula}
\begin{eulerprompt}
>povstart(angle=70°,height=50°,zoom=4);
>writeln(povsurface("(pow(pow(x,2)+pow(y,2)-pow(c,2),2)+pow(pow(z,2)-1,2))(pow(pow(y,2)+pow(z,2)-pow(c,2),2)+pow(pow(x,2)-1,2))(pow(pow(z,2)+pow(x,2)-pow(c,2),2)+pow(pow(y,2)-1,2))-d",povlook(red))); ...
>povend();
\end{eulerprompt}
\begin{euleroutput}
  Error : Povray error!
  
  Error generated by error() command
  
  povray:
      error("Povray error!");
  Try "trace errors" to inspect local variables after errors.
  povend:
      povray(file,w,h,aspect,exit); 
\end{euleroutput}
\begin{eulerprompt}
>povstart(angle=25°,height=10°); 
>writeln(povsurface("pow(x,2)+pow(y,2)*pow(z,2)-1",povlook(blue),povbox(-2,2,"")));
>povend();
\end{eulerprompt}
\eulerimg{27}{images/Plot3D_Fadhila Asmaul Karimah-081.png}
\begin{eulerprompt}
>povstart(angle=70°,height=50°,zoom=4);
\end{eulerprompt}
\begin{eulercomment}
Buatlah permukaan implisit. Perhatikan sintaks yang berbeda dalam
ekspresi ini.
\end{eulercomment}
\begin{eulerprompt}
>writeln(povsurface("pow(x,2)*y-pow(y,3)-pow(z,2)",povlook(green))); ...
>writeAxes(); ...
>povend();
\end{eulerprompt}
\eulerimg{27}{images/Plot3D_Fadhila Asmaul Karimah-082.png}
\eulerheading{Objek Jaringan}
\begin{eulercomment}
Dalam contoh ini, kami akan menunjukkan bagaimana cara membuat objek
jala, dan menggambarnya dengan informasi tambahan.

Kami ingin memaksimalkan xy dengan kondisi x+y=1 dan menunjukkan
sentuhan tangensial dari garis level.
\end{eulercomment}
\begin{eulerprompt}
>povstart(angle=-10°,center=[0.5,0.5,0.5],zoom=7);
\end{eulerprompt}
\begin{eulercomment}
Kita tidak dapat menyimpan objek dalam bentuk string seperti
sebelumnya, karena terlalu besar. Jadi, kita mendefinisikan objek
dalam sebuah file Povray menggunakan #declare. Fungsi povtriangle()
melakukan ini secara otomatis. Ini dapat menerima vektor normal
seperti pov3d().

Berikut ini mendefinisikan objek mesh, dan langsung menulisnya ke
dalam file.
\end{eulercomment}
\begin{eulerprompt}
>x=0:0.02:1; y=x'; z=x*y; vx=-y; vy=-x; vz=1;
>mesh=povtriangles(x,y,z,"",vx,vy,vz);
\end{eulerprompt}
\begin{eulercomment}
Sekarang kita akan mendefinisikan dua cakram, yang akan dipotong
dengan permukaan.
\end{eulercomment}
\begin{eulerprompt}
>cl=povdisc([0.5,0.5,0],[1,1,0],2); ...
>ll=povdisc([0,0,1/4],[0,0,1],2);
\end{eulerprompt}
\begin{eulercomment}
Ketik permukaannya dikurangi dua cakram.
\end{eulercomment}
\begin{eulerprompt}
>writeln(povdifference(mesh,povunion([cl,ll]),povlook(green)));
\end{eulerprompt}
\begin{eulercomment}
Tulis dua perpotongan.
\end{eulercomment}
\begin{eulerprompt}
>writeln(povintersection([mesh,cl],povlook(red))); ...
>writeln(povintersection([mesh,ll],povlook(gray)));
\end{eulerprompt}
\begin{eulercomment}
Tulis sebuah titik maksimum.
\end{eulercomment}
\begin{eulerprompt}
>writeln(povpoint([1/2,1/2,1/4],povlook(gray),size=2*defaultpointsize));
\end{eulerprompt}
\begin{eulercomment}
Tambahkan sumbu dan selesaikan
\end{eulercomment}
\begin{eulerprompt}
>writeAxes(0,1,0,1,0,1,d=0.015); ...
>povend();
\end{eulerprompt}
\eulerimg{27}{images/Plot3D_Fadhila Asmaul Karimah-083.png}
\eulerheading{Anaglif dalam Povray}
\begin{eulercomment}
Untuk menghasilkan anaglif untuk kacamata merah/cyan, Povray harus
dijalankan dua kali dari posisi kamera yang berbeda. Ini menghasilkan
dua file Povray dan dua file PNG, yang dimuat dengan fungsi
loadanaglyph().

Tentu saja, Anda memerlukan kacamata merah/cyan untuk melihat
contoh-contoh berikut dengan benar.

Fungsi pov3d() memiliki sakelar sederhana untuk menghasilkan anaglif.
\end{eulercomment}
\begin{eulerprompt}
>pov3d("-exp(-x^2-y^2)/2",r=2,height=45°,>anaglyph, ...
>  center=[0,0,0.5],zoom=3.5);
\end{eulerprompt}
\eulerimg{27}{images/Plot3D_Fadhila Asmaul Karimah-084.png}
\begin{eulercomment}
Jika Anda membuat sebuah adegan dengan objek-objek, Anda perlu
menempatkan pembuatan adegan tersebut ke dalam sebuah fungsi, dan
menjalankannya dua kali dengan nilai yang berbeda untuk parameter
anaglyph.
\end{eulercomment}
\begin{eulerprompt}
>function myscene ...
\end{eulerprompt}
\begin{eulerudf}
    s=povsphere(povc,1);
    cl=povcylinder(-povz,povz,0.5);
    clx=povobject(cl,rotate=xrotate(90°));
    cly=povobject(cl,rotate=yrotate(90°));
    c=povbox([-1,-1,0],1);
    un=povunion([cl,clx,cly,c]);
    obj=povdifference(s,un,povlook(red));
    writeln(obj);
    writeAxes();
  endfunction
\end{eulerudf}
\begin{eulercomment}
Fungsi povanaglyph() melakukan semua ini. Parameter-parameternya mirip
dengan yang ada dalam povstart() dan povend() yang digabungkan.
\end{eulercomment}
\begin{eulerprompt}
>povanaglyph("myscene",zoom=4.5);
\end{eulerprompt}
\eulerimg{27}{images/Plot3D_Fadhila Asmaul Karimah-085.png}
\eulerheading{* Mendefinisikan Objek Sendiri}
\begin{eulercomment}
Antarmuka povray pada Euler berisi banyak objek. Namun, Anda tidak
terbatas pada objek-objek tersebut. Anda dapat membuat objek sendiri,
yang menggabungkan objek-objek lain atau benar-benar objek baru.

Kami akan mendemonstrasikan sebuah torus. Perintah Povray untuk ini
adalah "torus". Jadi, kami akan mengembalikan sebuah string dengan
perintah ini beserta parameter-parameternya. Perhatikan bahwa torus
selalu berada di pusat asal.
\end{eulercomment}
\begin{eulerprompt}
>function povdonat (r1,r2,look="") ...
\end{eulerprompt}
\begin{eulerudf}
    return "torus \{"+r1+","+r2+look+"\}";
  endfunction
\end{eulerudf}
\begin{eulercomment}
Ini adalah torus pertama kami.
\end{eulercomment}
\begin{eulerprompt}
>t1=povdonat(0.8,0.2)
\end{eulerprompt}
\begin{euleroutput}
  torus \{0.8,0.2\}
\end{euleroutput}
\begin{eulercomment}
Mari kita gunakan objek ini untuk membuat torus kedua, yang telah
diterjemahkan dan diputar.
\end{eulercomment}
\begin{eulerprompt}
>t2=povobject(t1,rotate=xrotate(90°),translate=[0.8,0,0])
\end{eulerprompt}
\begin{euleroutput}
  object \{ torus \{0.8,0.2\}
   rotate 90 *x 
   translate <0.8,0,0>
   \}
\end{euleroutput}
\begin{eulercomment}
Sekarang kita menempatkan objek-objek ini ke dalam sebuah adegan.
Untuk tampilannya, kita menggunakan Phong Shading.
\end{eulercomment}
\begin{eulerprompt}
>povstart(center=[0.4,0,0],angle=0°,zoom=3.8,aspect=1.5); ...
>writeln(povobject(t1,povlook(green,phong=1))); ...
>writeln(povobject(t2,povlook(green,phong=1))); ...
\end{eulerprompt}
\begin{eulerttcomment}
 >povend();
\end{eulerttcomment}
\begin{eulercomment}
memanggil program Povray. Namun, dalam kasus kesalahan, program ini
tidak menampilkan pesan kesalahan. Oleh karena itu, Anda sebaiknya
menggunakan

\end{eulercomment}
\begin{eulerttcomment}
 >povend(<exit);
\end{eulerttcomment}
\begin{eulercomment}

jika ada yang tidak berfungsi. Ini akan membuat jendela Povray tetap
terbuka.
\end{eulercomment}
\begin{eulerprompt}
>povend(h=320,w=480);
\end{eulerprompt}
\eulerimg{18}{images/Plot3D_Fadhila Asmaul Karimah-086.png}
\begin{eulercomment}
Berikut contoh yang lebih rinci. Kami menyelesaikan

\end{eulercomment}
\begin{eulerformula}
\[
Ax \le b, \quad x \ge 0, \quad c.x \to \text{Max.}
\]
\end{eulerformula}
\begin{eulercomment}
dan menunjukkan titik-titik yang layak dan optimum dalam plot 3D.
\end{eulercomment}
\begin{eulerprompt}
>A=[10,8,4;5,6,8;6,3,2;9,5,6];
>b=[10,10,10,10]';
>c=[1,1,1];
\end{eulerprompt}
\begin{eulercomment}
Pertama, mari kita periksa, apakah contoh ini memiliki solusi sama
sekali.
\end{eulercomment}
\begin{eulerprompt}
>x=simplex(A,b,c,>max,>check)'
\end{eulerprompt}
\begin{euleroutput}
  [0,  1,  0.5]
\end{euleroutput}
\begin{eulercomment}
Ya, itu sudah ada.

Selanjutnya, kita akan mendefinisikan dua objek. Yang pertama adalah
bidang

\end{eulercomment}
\begin{eulerformula}
\[
a \cdot x \le b
\]
\end{eulerformula}
\begin{eulerprompt}
>function oneplane (a,b,look="") ...
\end{eulerprompt}
\begin{eulerudf}
    return povplane(a,b,look)
  endfunction
\end{eulerudf}
\begin{eulercomment}
Kemudian kita mendefinisikan irisan dari semua ruang setengah dan
sebuah kubus.
\end{eulercomment}
\begin{eulerprompt}
>function adm (A, b, r, look="") ...
\end{eulerprompt}
\begin{eulerudf}
    ol=[];
    loop 1 to rows(A); ol=ol|oneplane(A[#],b[#]); end;
    ol=ol|povbox([0,0,0],[r,r,r]);
    return povintersection(ol,look);
  endfunction
\end{eulerudf}
\begin{eulercomment}
Sekarang kita dapat menggambar adegan tersebut.
\end{eulercomment}
\begin{eulerprompt}
>povstart(angle=120°,center=[0.5,0.5,0.5],zoom=3.5); ...
>writeln(adm(A,b,2,povlook(green,0.4))); ...
>writeAxes(0,1.3,0,1.6,0,1.5); ...
\end{eulerprompt}
\begin{eulercomment}
Berikut adalah lingkaran di sekitar titik optimum.
\end{eulercomment}
\begin{eulerprompt}
>writeln(povintersection([povsphere(x,0.5),povplane(c,c.x')], ...
>  povlook(red,0.9)));
\end{eulerprompt}
\begin{eulercomment}
Dan kesalahan ke arah optimal.
\end{eulercomment}
\begin{eulerprompt}
>writeln(povarrow(x,c*0.5,povlook(red)));
\end{eulerprompt}
\begin{eulercomment}
Kami menambahkan teks ke layar. Teks hanyalah objek 3D. Kita perlu
menempatkan dan memutarnya sesuai dengan pandangan kita.
\end{eulercomment}
\begin{eulerprompt}
>writeln(povtext("Linear Problem",[0,0.2,1.3],size=0.05,rotate=5°)); ...
>povend();
\end{eulerprompt}
\eulerimg{27}{images/Plot3D_Fadhila Asmaul Karimah-089.png}
\eulerheading{Lebih Banyak Contoh}
\begin{eulercomment}
Anda dapat menemukan beberapa contoh Povray di euler di file berikut.

See: Examples/Dandelin Spheres\\
See: Examples/Donat Math\\
See: Examples/Trefoil Knot\\
See: Examples/Optimization by Affine Scaling

\begin{eulercomment}
\eulerheading{Contoh Soal}
\begin{eulercomment}
1. Sketsakan plot 3D untuk fungsi berikut

\end{eulercomment}
\begin{eulerformula}
\[
z= \sqrt{6-x^2-y^2}
\]
\end{eulerformula}
\begin{eulerprompt}
>plot3d("exp((6-x^2-y^2)^1/2)",r=5,n=30,fscale=2,scale=1.2,frame=3):
\end{eulerprompt}
\eulerimg{27}{images/Plot3D_Fadhila Asmaul Karimah-091.png}
\begin{eulercomment}
2. Sketsakan plot kontur untuk fungsi berikut

\end{eulercomment}
\begin{eulerformula}
\[
z=-10\sqrt|xy|
\]
\end{eulerformula}
\begin{eulerprompt}
>plot3d("exp(-10*abs(x*y)^1/2)",r=2,n=100,level="thin", ...
> >contour,>spectral,angle=30°,height=20°):
\end{eulerprompt}
\eulerimg{27}{images/Plot3D_Fadhila Asmaul Karimah-093.png}
\begin{eulercomment}
3. Sketsakan plot 3D berikut

\end{eulercomment}
\begin{eulerformula}
\[
\frac{9x^2-36y^2}{10}
\]
\end{eulerformula}
\begin{eulerprompt}
>load povray;
>defaultpovray="C:\(\backslash\)Program Files\(\backslash\)POV-Ray\(\backslash\)v3.7\(\backslash\)bin\(\backslash\)pvengine.exe"
\end{eulerprompt}
\begin{euleroutput}
  C:\(\backslash\)Program Files\(\backslash\)POV-Ray\(\backslash\)v3.7\(\backslash\)bin\(\backslash\)pvengine.exe
\end{euleroutput}
\begin{eulerprompt}
>pov3d("(9*x^2-36*y^2)/10",zoom=3);
\end{eulerprompt}
\eulerimg{27}{images/Plot3D_Fadhila Asmaul Karimah-095.png}
\begin{eulercomment}
4. Sketsakan peta kontur untuk fungsi berikut

\end{eulercomment}
\begin{eulerformula}
\[
z=\frac{y}{1+x^2+y^2}
\]
\end{eulerformula}
\begin{eulerprompt}
>plot3d("exp(y/(1+x^2+y^2))",r=4,n=100,level="thin", ...
> >contour,>spectral,angle=45°,height=30°):
\end{eulerprompt}
\eulerimg{27}{images/Plot3D_Fadhila Asmaul Karimah-097.png}
\end{eulernotebook}
\end{document}

\newpage
\chapter{Pekan 9-10: Menggunakan EMT untuk Kalkulus}
\input{Pekan 9-10/EMT4Kalkulus_Fadhila Asmaul Karimah}
\newpage
\chapter{Pekan 11-12: Menggunakan EMT untuk Geometri}
\input{Pekan 11-12/EMT4Geometry_Fadhila Asmaul Karimah}
\newpage
\chapter{Pekan 13-14: Menggunakan EMT untuk Statistika}
\documentclass{article}

\usepackage{eumat}

\begin{document}
\begin{eulernotebook}
\eulerheading{EMT untuk Statistika}
\begin{eulercomment}
Dalam buku catatan ini, kami menunjukkan plot statistik utama, uji,
dan distribusi dalam Euler.

Mari kita mulai dengan beberapa statistik deskriptif. Ini bukan
pengantar statistika. Jadi, Anda mungkin perlu beberapa latar belakang
untuk memahami detailnya.

Anggaplah pengukuran berikut. Kami ingin menghitung nilai rata-rata
dan deviasi standar yang diukur.
\end{eulercomment}
\begin{eulerprompt}
>M=[1000,1004,998,997,1002,1001,998,1004,998,997]; ...
>median(M), mean(M), dev(M),
\end{eulerprompt}
\begin{euleroutput}
  999
  999.9
  2.72641400622
\end{euleroutput}
\begin{eulercomment}
Kami dapat membuat plot kotak-dan-rambu untuk data tersebut. Dalam
kasus kami, tidak ada outlier.
\end{eulercomment}
\begin{eulerprompt}
>aspect(1.75); boxplot(M):
\end{eulerprompt}
\eulerimg{15}{images/EMT4Statistika_Fadhila Asmaul Karimah-001.png}
\begin{eulercomment}
Kami menghitung probabilitas bahwa nilai lebih besar dari 1005, dengan
mengasumsikan nilai yang diukur berasal dari distribusi normal.

Semua fungsi untuk distribusi dalam Euler diakhiri dengan ...dis dan
menghitung distribusi probabilitas kumulatif (CPF).

\end{eulercomment}
\begin{eulerformula}
\[
\text{normaldis(x,m,d)}=\int_{-\infty}^x \frac{1}{d\sqrt{2\pi}}e^{-\frac{1}{2}(\frac{t-m}{d})^2}\ dt.
\]
\end{eulerformula}
\begin{eulercomment}
Kami mencetak hasilnya dalam \% dengan akurasi 2 digit menggunakan
fungsi print.
\end{eulercomment}
\begin{eulerprompt}
>print((1-normaldis(1005,mean(M),dev(M)))*100,2,unit=" %")
\end{eulerprompt}
\begin{euleroutput}
        3.07 %
\end{euleroutput}
\begin{eulercomment}
Untuk contoh berikutnya, kita asumsikan jumlah pria dalam rentang
ukuran tertentu.
\end{eulercomment}
\begin{eulerprompt}
>r=155.5:4:187.5; v=[22,71,136,169,139,71,32,8];
\end{eulerprompt}
\begin{eulercomment}
Berikut adalah plot distribusinya.
\end{eulercomment}
\begin{eulerprompt}
>plot2d(r,v,a=150,b=200,c=0,d=190,bar=1,style="\(\backslash\)/"):
\end{eulerprompt}
\eulerimg{15}{images/EMT4Statistika_Fadhila Asmaul Karimah-003.png}
\begin{eulercomment}
Kita bisa memasukkan data mentah seperti itu ke dalam tabel.

Tabel adalah metode untuk menyimpan data statistik. Tabel kita harus
berisi tiga kolom: Awal rentang, akhir rentang, jumlah pria dalam
rentang tersebut.

Tabel dapat dicetak dengan header. Kami menggunakan vektor string
untuk mengatur header.
\end{eulercomment}
\begin{eulerprompt}
>T:=r[1:8]' | r[2:9]' | v'; writetable(T,labc=["BB","BA","Frek"])
\end{eulerprompt}
\begin{euleroutput}
          BB        BA      Frek
       155.5     159.5        22
       159.5     163.5        71
       163.5     167.5       136
       167.5     171.5       169
       171.5     175.5       139
       175.5     179.5        71
       179.5     183.5        32
       183.5     187.5         8
\end{euleroutput}
\begin{eulercomment}
Jika kita membutuhkan nilai rata-rata dan statistik lainnya dari
ukuran, kita perlu menghitung titik tengah dari rentang tersebut. Kita
dapat menggunakan dua kolom pertama dari tabel kita untuk ini.

Simbol "\textbar{}" digunakan untuk memisahkan kolom, fungsi "writetable"
digunakan untuk menulis tabel, dengan opsi "labc" untuk menentukan
header kolom.
\end{eulercomment}
\begin{eulerprompt}
>(T[,1]+T[,2])/2 // the midpoint of each interval
\end{eulerprompt}
\begin{euleroutput}
          157.5 
          161.5 
          165.5 
          169.5 
          173.5 
          177.5 
          181.5 
          185.5 
\end{euleroutput}
\begin{eulercomment}
Tetapi lebih mudah, untuk melipat rentang dengan vektor [1/2,1/2].
\end{eulercomment}
\begin{eulerprompt}
>M=fold(r,[0.5,0.5])
\end{eulerprompt}
\begin{euleroutput}
  [157.5,  161.5,  165.5,  169.5,  173.5,  177.5,  181.5,  185.5]
\end{euleroutput}
\begin{eulercomment}
Sekarang kita bisa menghitung rata-rata dan deviasi dari sampel dengan
frekuensi yang diberikan.
\end{eulercomment}
\begin{eulerprompt}
>\{m,d\}=meandev(M,v); m, d,
\end{eulerprompt}
\begin{euleroutput}
  169.901234568
  5.98912964449
\end{euleroutput}
\begin{eulercomment}
Mari tambahkan distribusi normal dari nilai-nilai ke plot batang di
atas. Rumus distribusi normal dengan rata-rata m dan deviasi standar d
adalah:

\end{eulercomment}
\begin{eulerformula}
\[
y=\frac{1}{d\sqrt{2\pi}}e^{\frac{-(x-m)^2}{2d^2}}.
\]
\end{eulerformula}
\begin{eulercomment}
Karena nilainya antara 0 dan 1, untuk memplotnya pada plot batang, itu
harus dikalikan dengan 4 kali total jumlah data.
\end{eulercomment}
\begin{eulerprompt}
>plot2d("qnormal(x,m,d)*sum(v)*4", ...
>  xmin=min(r),xmax=max(r),thickness=3,add=1):
\end{eulerprompt}
\eulerimg{15}{images/EMT4Statistika_Fadhila Asmaul Karimah-005.png}
\eulerheading{Tabel}
\begin{eulercomment}
Di direktori notebook ini, Anda akan menemukan file dengan tabel. Data
ini mewakili hasil survei. Berikut adalah empat baris pertama dari
file tersebut. Data berasal dari buku online Jerman "Einführung in die
Statistik mit R" oleh A. Handl.
\end{eulercomment}
\begin{eulerprompt}
>printfile("table.dat",4);
\end{eulerprompt}
\begin{euleroutput}
  Person Sex Age Titanic Evaluation Tip Problem
  1 m 30 n . 1.80 n
  2 f 23 y g 1.80 n
  3 f 26 y g 1.80 y
\end{euleroutput}
\begin{eulercomment}
Tabel ini berisi 7 kolom angka atau token (string). Kami ingin membaca
tabel dari file tersebut. Pertama, kami menggunakan terjemahan kami
sendiri untuk token-token tersebut.

Untuk ini, kami mendefinisikan set token. Fungsi strtokens() mengambil
vektor string token dari string yang diberikan.
\end{eulercomment}
\begin{eulerprompt}
>mf:=["m","f"]; yn:=["y","n"]; ev:=strtokens("g vg m b vb");
\end{eulerprompt}
\begin{eulercomment}
Sekarang kami membaca tabel dengan terjemahan ini.

Argumen tok2, tok4, dll. adalah terjemahan dari kolom-kolom tabel.
Argumen-argumen ini tidak ada dalam daftar parameter readtable(), jadi
Anda perlu menyediakannya dengan ":=".
\end{eulercomment}
\begin{eulerprompt}
>\{MT,hd\}=readtable("table.dat",tok2:=mf,tok4:=yn,tok5:=ev,tok7:=yn);
>load over statistics;
\end{eulerprompt}
\begin{eulercomment}
Untuk mencetak, kami perlu menentukan set token yang sama. Kami
mencetak hanya empat baris pertama.
\end{eulercomment}
\begin{eulerprompt}
>writetable(MT[1:10],labc=hd,wc=5,tok2:=mf,tok4:=yn,tok5:=ev,tok7:=yn);
\end{eulerprompt}
\begin{euleroutput}
   Person  Sex  Age Titanic Evaluation  Tip Problem
        1    m   30       n          .  1.8       n
        2    f   23       y          g  1.8       n
        3    f   26       y          g  1.8       y
        4    m   33       n          .  2.8       n
        5    m   37       n          .  1.8       n
        6    m   28       y          g  2.8       y
        7    f   31       y         vg  2.8       n
        8    m   23       n          .  0.8       n
        9    f   24       y         vg  1.8       y
       10    m   26       n          .  1.8       n
\end{euleroutput}
\begin{eulercomment}
Titik "." mewakili nilai yang tidak tersedia.

Jika kami tidak ingin menentukan token untuk terjemahan sebelumnya,
kami hanya perlu menentukan kolom-kolom yang berisi token dan bukan
angka.
\end{eulercomment}
\begin{eulerprompt}
>ctok=[2,4,5,7]; \{MT,hd,tok\}=readtable("table.dat",ctok=ctok);
\end{eulerprompt}
\begin{eulercomment}
Fungsi readtable() sekarang mengembalikan set token.
\end{eulercomment}
\begin{eulerprompt}
>tok
\end{eulerprompt}
\begin{euleroutput}
  m
  n
  f
  y
  g
  vg
\end{euleroutput}
\begin{eulercomment}
Tabel ini berisi entri dari file dengan token diterjemahkan ke angka.

String khusus NA="." diinterpretasikan sebagai "Tidak Tersedia" dan
mendapatkan NAN (bukan angka) dalam tabel. Terjemahan ini dapat diubah
dengan parameter NA dan NAval.
\end{eulercomment}
\begin{eulerprompt}
>MT[1]
\end{eulerprompt}
\begin{euleroutput}
  [1,  1,  30,  2,  NAN,  1.8,  2]
\end{euleroutput}
\begin{eulercomment}
Berikut adalah isi tabel dengan angka yang belum diterjemahkan.
\end{eulercomment}
\begin{eulerprompt}
>writetable(MT,wc=5)
\end{eulerprompt}
\begin{euleroutput}
      1    1   30    2    .  1.8    2
      2    3   23    4    5  1.8    2
      3    3   26    4    5  1.8    4
      4    1   33    2    .  2.8    2
      5    1   37    2    .  1.8    2
      6    1   28    4    5  2.8    4
      7    3   31    4    6  2.8    2
      8    1   23    2    .  0.8    2
      9    3   24    4    6  1.8    4
     10    1   26    2    .  1.8    2
     11    3   23    4    6  1.8    4
     12    1   32    4    5  1.8    2
     13    1   29    4    6  1.8    4
     14    3   25    4    5  1.8    4
     15    3   31    4    5  0.8    2
     16    1   26    4    5  2.8    2
     17    1   37    2    .  3.8    2
     18    1   38    4    5    .    2
     19    3   29    2    .  3.8    2
     20    3   28    4    6  1.8    2
     21    3   28    4    1  2.8    4
     22    3   28    4    6  1.8    4
     23    3   38    4    5  2.8    2
     24    3   27    4    1  1.8    4
     25    1   27    2    .  2.8    4
\end{euleroutput}
\begin{eulercomment}
Untuk kenyamanan, Anda dapat menempatkan output readtable() ke dalam
daftar.
\end{eulercomment}
\begin{eulerprompt}
>Table=\{\{readtable("table.dat",ctok=ctok)\}\};
\end{eulerprompt}
\begin{eulercomment}
Dengan menggunakan kolom-kolom token yang sama dan token yang dibaca
dari file, kita dapat mencetak tabel. Kita dapat menentukan ctok, tok,
dll., atau menggunakan daftar Tabel.
\end{eulercomment}
\begin{eulerprompt}
>writetable(Table,ctok=ctok,wc=5);
\end{eulerprompt}
\begin{euleroutput}
   Person  Sex  Age Titanic Evaluation  Tip Problem
        1    m   30       n          .  1.8       n
        2    f   23       y          g  1.8       n
        3    f   26       y          g  1.8       y
        4    m   33       n          .  2.8       n
        5    m   37       n          .  1.8       n
        6    m   28       y          g  2.8       y
        7    f   31       y         vg  2.8       n
        8    m   23       n          .  0.8       n
        9    f   24       y         vg  1.8       y
       10    m   26       n          .  1.8       n
       11    f   23       y         vg  1.8       y
       12    m   32       y          g  1.8       n
       13    m   29       y         vg  1.8       y
       14    f   25       y          g  1.8       y
       15    f   31       y          g  0.8       n
       16    m   26       y          g  2.8       n
       17    m   37       n          .  3.8       n
       18    m   38       y          g    .       n
       19    f   29       n          .  3.8       n
       20    f   28       y         vg  1.8       n
       21    f   28       y          m  2.8       y
       22    f   28       y         vg  1.8       y
       23    f   38       y          g  2.8       n
       24    f   27       y          m  1.8       y
       25    m   27       n          .  2.8       y
\end{euleroutput}
\begin{eulercomment}
Fungsi tablecol() mengembalikan nilai-nilai kolom tabel, melewati
baris apa pun dengan nilai NAN (".") dalam file, dan indeks kolom yang
berisi nilai-nilai ini.
\end{eulercomment}
\begin{eulerprompt}
>\{c,i\}=tablecol(MT,[5,6]);
\end{eulerprompt}
\begin{eulercomment}
Kami dapat menggunakan ini untuk mengekstrak kolom dari tabel untuk
tabel baru.
\end{eulercomment}
\begin{eulerprompt}
>j=[1,5,6]; writetable(MT[i,j],labc=hd[j],ctok=[2],tok=tok)
\end{eulerprompt}
\begin{euleroutput}
      Person Evaluation       Tip
           2          g       1.8
           3          g       1.8
           6          g       2.8
           7         vg       2.8
           9         vg       1.8
          11         vg       1.8
          12          g       1.8
          13         vg       1.8
          14          g       1.8
          15          g       0.8
          16          g       2.8
          20         vg       1.8
          21          m       2.8
          22         vg       1.8
          23          g       2.8
          24          m       1.8
\end{euleroutput}
\begin{eulercomment}
Tentu saja, kita perlu mengekstrak tabel itu sendiri dari daftar Tabel
dalam hal ini.
\end{eulercomment}
\begin{eulerprompt}
>MT=Table[1];
\end{eulerprompt}
\begin{eulercomment}
Tentu saja, kita juga dapat menggunakannya untuk menentukan nilai
rata-rata dari suatu kolom atau nilai statistik lainnya.
\end{eulercomment}
\begin{eulerprompt}
>mean(tablecol(MT,6))
\end{eulerprompt}
\begin{euleroutput}
  2.175
\end{euleroutput}
\begin{eulercomment}
Fungsi getstatistics() mengembalikan elemen-elemen dalam vektor, dan
jumlahnya. Kami menerapkannya pada nilai "m" dan "f" dalam kolom kedua
tabel kami.
\end{eulercomment}
\begin{eulerprompt}
>\{xu,count\}=getstatistics(tablecol(MT,2)); xu, count,
\end{eulerprompt}
\begin{euleroutput}
  [1,  3]
  [12,  13]
\end{euleroutput}
\begin{eulercomment}
Kita dapat mencetak hasilnya dalam tabel baru.
\end{eulercomment}
\begin{eulerprompt}
>writetable(count',labr=tok[xu])
\end{eulerprompt}
\begin{euleroutput}
           m        12
           f        13
\end{euleroutput}
\begin{eulercomment}
Fungsi selecttable() mengembalikan tabel baru dengan nilai dalam satu
kolom yang dipilih dari vektor indeks. Pertama, kita mencari indeks
dua nilai kita dalam tabel token.
\end{eulercomment}
\begin{eulerprompt}
>v:=indexof(tok,["g","vg"])
\end{eulerprompt}
\begin{euleroutput}
  [5,  6]
\end{euleroutput}
\begin{eulercomment}
Sekarang kita dapat memilih baris tabel yang memiliki salah satu nilai
dalam vektor v pada baris ke-5 mereka.
\end{eulercomment}
\begin{eulerprompt}
>MT1:=MT[selectrows(MT,5,v)]; i:=sortedrows(MT1,5);
\end{eulerprompt}
\begin{eulercomment}
Sekarang kita dapat mencetak tabel, dengan nilai yang diekstrak dan
diurutkan di kolom ke-5.
\end{eulercomment}
\begin{eulerprompt}
>writetable(MT1[i],labc=hd,ctok=ctok,tok=tok,wc=7);
\end{eulerprompt}
\begin{euleroutput}
   Person    Sex    Age Titanic Evaluation    Tip Problem
        2      f     23       y          g    1.8       n
        3      f     26       y          g    1.8       y
        6      m     28       y          g    2.8       y
       18      m     38       y          g      .       n
       16      m     26       y          g    2.8       n
       15      f     31       y          g    0.8       n
       12      m     32       y          g    1.8       n
       23      f     38       y          g    2.8       n
       14      f     25       y          g    1.8       y
        9      f     24       y         vg    1.8       y
        7      f     31       y         vg    2.8       n
       20      f     28       y         vg    1.8       n
       22      f     28       y         vg    1.8       y
       13      m     29       y         vg    1.8       y
       11      f     23       y         vg    1.8       y
\end{euleroutput}
\begin{eulercomment}
Untuk statistik berikutnya, kita ingin menghubungkan dua kolom tabel.
Jadi kita ekstrak kolom 2 dan 4 dan mengurutkan tabel.
\end{eulercomment}
\begin{eulerprompt}
>i=sortedrows(MT,[2,4]);  ...
>  writetable(tablecol(MT[i],[2,4])',ctok=[1,2],tok=tok)
\end{eulerprompt}
\begin{euleroutput}
           m         n
           m         n
           m         n
           m         n
           m         n
           m         n
           m         n
           m         y
           m         y
           m         y
           m         y
           m         y
           f         n
           f         y
           f         y
           f         y
           f         y
           f         y
           f         y
           f         y
           f         y
           f         y
           f         y
           f         y
           f         y
\end{euleroutput}
\begin{eulercomment}
Dengan getstatistics(), kita juga dapat menghubungkan jumlah dalam dua
kolom tabel satu sama lain.
\end{eulercomment}
\begin{eulerprompt}
>MT24=tablecol(MT,[2,4]); ...
>\{xu1,xu2,count\}=getstatistics(MT24[1],MT24[2]); ...
>writetable(count,labr=tok[xu1],labc=tok[xu2])
\end{eulerprompt}
\begin{euleroutput}
                     n         y
           m         7         5
           f         1        12
\end{euleroutput}
\begin{eulercomment}
Sebuah tabel dapat ditulis ke file.
\end{eulercomment}
\begin{eulerprompt}
>filename="test.dat"; ...
>writetable(count,labr=tok[xu1],labc=tok[xu2],file=filename);
\end{eulerprompt}
\begin{eulercomment}
Kemudian kita dapat membaca tabel dari file.
\end{eulercomment}
\begin{eulerprompt}
>\{MT2,hd,tok2,hdr\}=readtable(filename,>clabs,>rlabs); ...
>writetable(MT2,labr=hdr,labc=hd)
\end{eulerprompt}
\begin{euleroutput}
                     n         y
           m         7         5
           f         1        12
\end{euleroutput}
\begin{eulercomment}
Dan menghapus file.
\end{eulercomment}
\begin{eulerprompt}
>fileremove(filename);
\end{eulerprompt}
\eulerheading{Distribusi}
\begin{eulercomment}
Dengan plot2d, ada metode yang sangat mudah untuk membuat plot
distribusi data eksperimental.
\end{eulercomment}
\begin{eulerprompt}
>p=normal(1,1000); //1000 random normal-distributed sample p
>plot2d(p,distribution=20,style="\(\backslash\)/"); // plot the random sample p
>plot2d("qnormal(x,0,1)",add=1): // add the standard normal distribution plot
\end{eulerprompt}
\eulerimg{27}{images/EMT4Statistika_Fadhila Asmaul Karimah-006.png}
\begin{eulercomment}
Perhatikan perbedaan antara diagram batang (sampel) dan kurva normal
(distribusi sebenarnya). Ketik kembali tiga perintah untuk melihat
hasil sampling lainnya.
\end{eulercomment}
\begin{eulercomment}
Berikut adalah perbandingan dari 10 simulasi nilai terdistribusi
normal sebanyak 1000 menggunakan diagram kotak (box plot). Plot ini
menunjukkan median, kuartil 25\% dan 75\%, nilai minimal dan maksimal,
serta nilai-nilai outliers.
\end{eulercomment}
\begin{eulerprompt}
>p=normal(10,1000); boxplot(p):
\end{eulerprompt}
\eulerimg{27}{images/EMT4Statistika_Fadhila Asmaul Karimah-007.png}
\begin{eulercomment}
Untuk menghasilkan bilangan bulat acak, Euler memiliki intrandom. Mari
kita simulasi lemparan dadu dan plot distribusinya.

Kami menggunakan fungsi getmultiplicities(v, x), yang menghitung
seberapa sering elemen-elemen v muncul dalam x. Kemudian kami plot
hasilnya menggunakan columnsplot().
\end{eulercomment}
\begin{eulerprompt}
>k=intrandom(1,6000,6);  ...
>columnsplot(getmultiplicities(1:6,k));  ...
>ygrid(1000,color=red):
\end{eulerprompt}
\eulerimg{27}{images/EMT4Statistika_Fadhila Asmaul Karimah-008.png}
\begin{eulercomment}
Sementara intrandom(n, m, k) mengembalikan bilangan bulat
terdistribusi merata dari 1 hingga k, mungkin juga menggunakan
distribusi bilangan bulat lainnya dengan randpint().

Dalam contoh berikut, probabilitas untuk 1, 2, 3 adalah masing-masing
0,4, 0,1, 0,5.
\end{eulercomment}
\begin{eulerprompt}
>randpint(1,1000,[0.4,0.1,0.5]); getmultiplicities(1:3,%)
\end{eulerprompt}
\begin{euleroutput}
  [378,  102,  520]
\end{euleroutput}
\begin{eulercomment}
Euler dapat menghasilkan nilai acak dari lebih banyak distribusi.
Lihat referensi untuk informasi lebih lanjut.

Sebagai contoh, kita mencoba distribusi eksponensial. Variabel acak
kontinu X dikatakan memiliki distribusi eksponensial jika PDF-nya
diberikan oleh

\end{eulercomment}
\begin{eulerformula}
\[
f_X(x)=\lambda e^{-\lambda x},\quad x>0,\quad \lambda>0,
\]
\end{eulerformula}
\begin{eulercomment}
dengan parameter

\end{eulercomment}
\begin{eulerformula}
\[
\lambda=\frac{1}{\mu},\quad \mu \text{ adalah rata-rata, dan dilambangkan dengan } X \sim \text{Exponential}(\lambda).
\]
\end{eulerformula}
\begin{eulerprompt}
>plot2d(randexponential(1,1000,2),>distribution):
\end{eulerprompt}
\eulerimg{27}{images/EMT4Statistika_Fadhila Asmaul Karimah-011.png}
\begin{eulercomment}
Untuk banyak distribusi, Euler dapat menghitung fungsi distribusi dan
inversnya.
\end{eulercomment}
\begin{eulerprompt}
>plot2d("normaldis",-4,4): 
\end{eulerprompt}
\eulerimg{27}{images/EMT4Statistika_Fadhila Asmaul Karimah-012.png}
\begin{eulercomment}
Berikut adalah satu cara untuk membuat plot kuantil.
\end{eulercomment}
\begin{eulerprompt}
>plot2d("qnormal(x,1,1.5)",-4,6);  ...
>plot2d("qnormal(x,1,1.5)",a=2,b=5,>add,>filled):
\end{eulerprompt}
\eulerimg{27}{images/EMT4Statistika_Fadhila Asmaul Karimah-013.png}
\begin{eulerformula}
\[
\text{normaldis(x,m,d)}=\int_{-\infty}^x \frac{1}{d\sqrt{2\pi}}e^{-\frac{1}{2}(\frac{t-m}{d})^2}\ dt.
\]
\end{eulerformula}
\begin{eulercomment}
Probabilitas/peluang berada di area hijau adalah sebagai berikut
\end{eulercomment}
\begin{eulerprompt}
>normaldis(5,1,1.5)-normaldis(2,1,1.5)
\end{eulerprompt}
\begin{euleroutput}
  0.248662156979
\end{euleroutput}
\begin{eulercomment}
Probabilitas berada di area hijau dapat dihitung numerik dengan
integral berikut.

\end{eulercomment}
\begin{eulerformula}
\[
\int_2^5 \frac{1}{1.5\sqrt{2\pi}}e^{-\frac{1}{2}(\frac{x-1}{1.5})^2}\ dx.
\]
\end{eulerformula}
\begin{eulerprompt}
>gauss("qnormal(x,1,1.5)",2,5)
\end{eulerprompt}
\begin{euleroutput}
  0.248662156979
\end{euleroutput}
\begin{eulercomment}
Mari kita bandingkan distribusi binomial dengan distribusi normal yang
memiliki mean dan deviasi yang sama. Fungsi invbindis() menyelesaikan
interpolasi linear antara nilai bulat.
\end{eulercomment}
\begin{eulerprompt}
>invbindis(0.95,1000,0.5), invnormaldis(0.95,500,0.5*sqrt(1000))
\end{eulerprompt}
\begin{euleroutput}
  525.516721219
  526.007419394
\end{euleroutput}
\begin{eulercomment}
Fungsi qdis() adalah densitas distribusi chi-square. Seperti biasa,
Euler memetakan vektor ke fungsi ini. Dengan demikian, kita dapat
dengan mudah membuat plot semua distribusi chi-square dengan derajat 5
hingga 30.
\end{eulercomment}
\begin{eulerprompt}
>plot2d("qchidis(x,(5:5:50)')",0,50):
\end{eulerprompt}
\eulerimg{27}{images/EMT4Statistika_Fadhila Asmaul Karimah-016.png}
\begin{eulercomment}
Euler memiliki fungsi akurat untuk mengevaluasi distribusi. Mari kita
periksa chidis() dengan sebuah integral.

Penamaannya mencoba konsisten. Misalnya,

- distribusi chi-square adalah chidis(),\\
- fungsi inversnya adalah invchidis(),\\
- densitasnya adalah qchidis().

Komplemen dari distribusi (upper tail) adalah chicdis().
\end{eulercomment}
\begin{eulerprompt}
>chidis(1.5,2), integrate("qchidis(x,2)",0,1.5)
\end{eulerprompt}
\begin{euleroutput}
  0.527633447259
  0.527633447259
\end{euleroutput}
\eulerheading{Distribusi Diskrit}
\begin{eulercomment}
Untuk mendefinisikan distribusi diskrit Anda sendiri, Anda dapat
menggunakan metode berikut.

Pertama, kita atur fungsi distribusinya.
\end{eulercomment}
\begin{eulerprompt}
>wd = 0|((1:6)+[-0.01,0.01,0,0,0,0])/6
\end{eulerprompt}
\begin{euleroutput}
  [0,  0.165,  0.335,  0.5,  0.666667,  0.833333,  1]
\end{euleroutput}
\begin{eulercomment}
Artinya, dengan probabilitas wd[i+1]-wd[i], kita menghasilkan nilai
acak i.

Ini hampir merupakan distribusi seragam. Mari kita tentukan pembangkit
bilangan acak untuk ini. Fungsi find(v, x) menemukan nilai x dalam
vektor v. Ini berfungsi juga untuk vektor x.
\end{eulercomment}
\begin{eulerprompt}
>function wrongdice (n,m) := find(wd,random(n,m))
\end{eulerprompt}
\begin{eulercomment}
Kesalahan ini begitu halus sehingga kita hanya melihatnya dengan
iterasi yang sangat banyak.
\end{eulercomment}
\begin{eulerprompt}
>columnsplot(getmultiplicities(1:6,wrongdice(1,1000000))):
\end{eulerprompt}
\eulerimg{15}{images/EMT4Statistika_Fadhila Asmaul Karimah-017.png}
\begin{eulercomment}
Berikut adalah fungsi sederhana untuk memeriksa distribusi seragam
dari nilai 1...K dalam v. Kami menerima hasilnya, jika untuk semua
frekuensi

\end{eulercomment}
\begin{eulerformula}
\[
\left|f_i-\frac{1}{K}\right| < \frac{\delta}{\sqrt{n}}.
\]
\end{eulerformula}
\begin{eulerprompt}
>function checkrandom (v, delta=1) ...
\end{eulerprompt}
\begin{eulerudf}
    K=max(v); n=cols(v);
    fr=getfrequencies(v,1:K);
    return max(fr/n-1/K)<delta/sqrt(n);
    endfunction
\end{eulerudf}
\begin{eulercomment}
Memang, fungsi menolak distribusi seragam.
\end{eulercomment}
\begin{eulerprompt}
>checkrandom(wrongdice(1,1000000))
\end{eulerprompt}
\begin{euleroutput}
  0
\end{euleroutput}
\begin{eulercomment}
Dan itu menerima pembangkit acak bawaan.
\end{eulercomment}
\begin{eulerprompt}
>checkrandom(intrandom(1,1000000,6))
\end{eulerprompt}
\begin{euleroutput}
  1
\end{euleroutput}
\begin{eulercomment}
Kita dapat menghitung distribusi binomial. Pertama, ada binomialsum(),
yang mengembalikan probabilitas i atau kurang hasil dari uji coba n.
\end{eulercomment}
\begin{eulerprompt}
>bindis(410,1000,0.4)
\end{eulerprompt}
\begin{euleroutput}
  0.751401349654
\end{euleroutput}
\begin{eulercomment}
Fungsi invers Beta digunakan untuk menghitung interval kepercayaan
Clopper-Pearson untuk parameter p. Tingkat defaultnya adalah alpha.

Arti dari interval ini adalah bahwa jika p berada di luar interval,
hasil yang diamati sebanyak 410 dari 1000 adalah langka.
\end{eulercomment}
\begin{eulerprompt}
>clopperpearson(410,1000)
\end{eulerprompt}
\begin{euleroutput}
  [0.37932,  0.441212]
\end{euleroutput}
\begin{eulercomment}
Perintah-perintah berikut adalah cara langsung untuk mendapatkan hasil
di atas. Tetapi untuk n yang besar, penjumlahan langsung tidak akurat
dan lambat.
\end{eulercomment}
\begin{eulerprompt}
>p=0.4; i=0:410; n=1000; sum(bin(n,i)*p^i*(1-p)^(n-i))
\end{eulerprompt}
\begin{euleroutput}
  0.751401349655
\end{euleroutput}
\begin{eulercomment}
Sekadar informasi, invbinsum() menghitung invers dari binomialsum().
\end{eulercomment}
\begin{eulerprompt}
>invbindis(0.75,1000,0.4)
\end{eulerprompt}
\begin{euleroutput}
  409.932733047
\end{euleroutput}
\begin{eulercomment}
Di Bridge, kita mengasumsikan 5 kartu luar biasa (dari 52) dalam dua
tangan (26 kartu). Mari kita hitung probabilitas distribusi yang lebih
buruk daripada 3:2 (misalnya, 0:5, 1:4, 4:1, atau 5:0).
\end{eulercomment}
\begin{eulerprompt}
>2*hypergeomsum(1,5,13,26)
\end{eulerprompt}
\begin{euleroutput}
  0.321739130435
\end{euleroutput}
\begin{eulercomment}
Ada juga simulasi distribusi multinomial.
\end{eulercomment}
\begin{eulerprompt}
>randmultinomial(10,1000,[0.4,0.1,0.5])
\end{eulerprompt}
\begin{euleroutput}
            381           100           519 
            376            91           533 
            417            80           503 
            440            94           466 
            406           112           482 
            408            94           498 
            395           107           498 
            399            96           505 
            428            87           485 
            400            99           501 
\end{euleroutput}
\eulerheading{Plotting Data}
\begin{eulercomment}
Untuk membuat grafik data, kita akan mencoba hasil pemilihan umum di
Jerman sejak tahun 1990, diukur dalam jumlah seats.
\end{eulercomment}
\begin{eulerprompt}
>BW := [ ...
>1990,662,319,239,79,8,17; ...
>1994,672,294,252,47,49,30; ...
>1998,669,245,298,43,47,36; ...
>2002,603,248,251,47,55,2; ...
>2005,614,226,222,61,51,54; ...
>2009,622,239,146,93,68,76; ...
>2013,631,311,193,0,63,64];
\end{eulerprompt}
\begin{eulercomment}
Untuk partai-partai, kita menggunakan serangkaian nama.
\end{eulercomment}
\begin{eulerprompt}
>P:=["CDU/CSU","SPD","FDP","Gr","Li"];
\end{eulerprompt}
\begin{eulercomment}
Mari kita cetak persentasenya dengan rapi.

Pertama, kita ekstrak kolom-kolom yang diperlukan. Kolom 3 hingga 7
adalah kursi masing-masing partai, dan kolom 2 adalah total jumlah
kursi. Kolom lainnya adalah tahun pemilihan.
\end{eulercomment}
\begin{eulerprompt}
>BT:=BW[,3:7]; BT:=BT/sum(BT); YT:=BW[,1]';
\end{eulerprompt}
\begin{eulercomment}
Kemudian kita cetak statistik dalam bentuk tabel. Kita gunakan
nama-nama sebagai header kolom, dan tahun sebagai header untuk baris.
Lebar default untuk kolom adalah wc=10, tetapi kita lebih suka output
yang lebih padat. Kolom akan diperluas untuk label-label kolom, jika
diperlukan.
\end{eulercomment}
\begin{eulerprompt}
>writetable(BT*100,wc=6,dc=0,>fixed,labc=P,labr=YT)
\end{eulerprompt}
\begin{euleroutput}
         CDU/CSU   SPD   FDP    Gr    Li
    1990      48    36    12     1     3
    1994      44    38     7     7     4
    1998      37    45     6     7     5
    2002      41    42     8     9     0
    2005      37    36    10     8     9
    2009      38    23    15    11    12
    2013      49    31     0    10    10
\end{euleroutput}
\begin{eulercomment}
Perkalian matriks berikutnya menghasilkan jumlah persentase dari dua
partai besar, menunjukkan bahwa partai-partai kecil telah mendapatkan
kursi di parlemen hingga tahun 2009.
\end{eulercomment}
\begin{eulerprompt}
>BT1:=(BT.[1;1;0;0;0])'*100
\end{eulerprompt}
\begin{euleroutput}
  [84.29,  81.25,  81.1659,  82.7529,  72.9642,  61.8971,  79.8732]
\end{euleroutput}
\begin{eulercomment}
Ada juga plot statistik sederhana. Kita menggunakannya untuk
menampilkan garis dan titik secara bersamaan. Alternatifnya adalah
memanggil plot2d dua kali dengan \textgreater{}add.
\end{eulercomment}
\begin{eulerprompt}
>statplot(YT,BT1,"b"):
\end{eulerprompt}
\eulerimg{15}{images/EMT4Statistika_Fadhila Asmaul Karimah-019.png}
\begin{eulercomment}
Tentukan beberapa warna untuk setiap bagian.
\end{eulercomment}
\begin{eulerprompt}
>CP:=[rgb(0.5,0.5,0.5),red,yellow,green,rgb(0.8,0,0)];
\end{eulerprompt}
\begin{eulercomment}
Sekarang kita bisa membuat grafik hasil pemilihan tahun 2009 dan
perubahannya ke dalam satu plot menggunakan figure. Kita bisa
menambahkan vektor kolom ke setiap plot.
\end{eulercomment}
\begin{eulerprompt}
>figure(2,1);  ...
>figure(1); columnsplot(BW[6,3:7],P,color=CP); ...
>figure(2); columnsplot(BW[6,3:7]-BW[5,3:7],P,color=CP);  ...
>figure(0):
\end{eulerprompt}
\eulerimg{15}{images/EMT4Statistika_Fadhila Asmaul Karimah-020.png}
\begin{eulercomment}
Grafik data menggabungkan baris data statistik dalam satu plot.
\end{eulercomment}
\begin{eulerprompt}
>J:=BW[,1]'; DP:=BW[,3:7]'; ...
>dataplot(YT,BT',color=CP);  ...
>labelbox(P,colors=CP,styles="[]",>points,w=0.2,x=0.3,y=0.4):
\end{eulerprompt}
\eulerimg{15}{images/EMT4Statistika_Fadhila Asmaul Karimah-021.png}
\begin{eulercomment}
Grafik kolom 3D menunjukkan baris data statistik dalam bentuk kolom.
Kita menyediakan label untuk baris dan kolom. Sudut adalah sudut
pandangnya.
\end{eulercomment}
\begin{eulerprompt}
>columnsplot3d(BT,scols=P,srows=YT, ...
>  angle=30°,ccols=CP):
\end{eulerprompt}
\eulerimg{15}{images/EMT4Statistika_Fadhila Asmaul Karimah-022.png}
\begin{eulercomment}
Representasi lainnya adalah plot mozaik. Perhatikan bahwa kolom-kolom
dari plot ini mewakili kolom-kolom matriks di sini. Karena panjang
label CDU/CSU, kita mengambil jendela yang lebih kecil dari biasanya.
\end{eulercomment}
\begin{eulerprompt}
>shrinkwindow(>smaller);  ...
>mosaicplot(BT',srows=YT,scols=P,color=CP,style="#"); ...
>shrinkwindow():
\end{eulerprompt}
\eulerimg{15}{images/EMT4Statistika_Fadhila Asmaul Karimah-023.png}
\begin{eulercomment}
Kita juga bisa membuat diagram lingkaran. Karena hitam dan kuning
membentuk koalisi, kita menyusun ulang elemen-elemen tersebut.
\end{eulercomment}
\begin{eulerprompt}
>i=[1,3,5,4,2]; piechart(BW[6,3:7][i],color=CP[i],lab=P[i]):
\end{eulerprompt}
\eulerimg{15}{images/EMT4Statistika_Fadhila Asmaul Karimah-024.png}
\begin{eulercomment}
Berikut adalah jenis plot lainnya.
\end{eulercomment}
\begin{eulerprompt}
>starplot(normal(1,10)+4,lab=1:10,>rays):
\end{eulerprompt}
\eulerimg{15}{images/EMT4Statistika_Fadhila Asmaul Karimah-025.png}
\begin{eulercomment}
Beberapa plot dalam plot2d bagus untuk statistika. Berikut adalah plot
impuls dari data acak, terdistribusi secara merata di [0,1].
\end{eulercomment}
\begin{eulerprompt}
>plot2d(makeimpulse(1:10,random(1,10)),>bar):
\end{eulerprompt}
\eulerimg{15}{images/EMT4Statistika_Fadhila Asmaul Karimah-026.png}
\begin{eulercomment}
Namun untuk data yang terdistribusi secara eksponensial, kita mungkin
memerlukan plot logaritmik.
\end{eulercomment}
\begin{eulerprompt}
>logimpulseplot(1:10,-log(random(1,10))*10):
\end{eulerprompt}
\eulerimg{15}{images/EMT4Statistika_Fadhila Asmaul Karimah-027.png}
\begin{eulercomment}
ungsi columnsplot() lebih mudah digunakan, karena hanya membutuhkan
vektor nilai. Selain itu, dapat mengatur label sesuai keinginan kita,
seperti yang sudah kita tunjukkan dalam tutorial ini.

Berikut adalah aplikasi lain, di mana kita menghitung karakter dalam
sebuah kalimat dan membuat plot statistik.
\end{eulercomment}
\begin{eulerprompt}
>v=strtochar("the quick brown fox jumps over the lazy dog"); ...
>w=ascii("a"):ascii("z"); x=getmultiplicities(w,v); ...
>cw=[]; for k=w; cw=cw|char(k); end; ...
>columnsplot(x,lab=cw,width=0.05):
\end{eulerprompt}
\eulerimg{15}{images/EMT4Statistika_Fadhila Asmaul Karimah-028.png}
\begin{eulercomment}
Juga mungkin untuk secara manual menetapkan sumbu-sumbu.
\end{eulercomment}
\begin{eulerprompt}
>n=10; p=0.4; i=0:n; x=bin(n,i)*p^i*(1-p)^(n-i); ...
>columnsplot(x,lab=i,width=0.05,<frame,<grid); ...
>yaxis(0,0:0.1:1,style="->",>left); xaxis(0,style="."); ...
>label("p",0,0.25), label("i",11,0); ...
>textbox(["Binomial distribution","with p=0.4"]):
\end{eulerprompt}
\eulerimg{15}{images/EMT4Statistika_Fadhila Asmaul Karimah-029.png}
\begin{eulercomment}
Berikut adalah cara membuat plot frekuensi angka dalam vektor.

Kita membuat vektor angka acak bulat 1 hingga 6.
\end{eulercomment}
\begin{eulerprompt}
>v:=intrandom(1,10,10)
\end{eulerprompt}
\begin{euleroutput}
  [8,  5,  8,  8,  6,  8,  8,  3,  5,  5]
\end{euleroutput}
\begin{eulercomment}
Kemudian ekstrak angka-angka unik dalam v.
\end{eulercomment}
\begin{eulerprompt}
>vu:=unique(v)
\end{eulerprompt}
\begin{euleroutput}
  [3,  5,  6,  8]
\end{euleroutput}
\begin{eulercomment}
Dan plot frekuensinya dalam plot kolom.
\end{eulercomment}
\begin{eulerprompt}
>columnsplot(getmultiplicities(vu,v),lab=vu,style="/"):
\end{eulerprompt}
\eulerimg{15}{images/EMT4Statistika_Fadhila Asmaul Karimah-030.png}
\begin{eulercomment}
Kita ingin menunjukkan fungsi untuk distribusi empiris nilai.
\end{eulercomment}
\begin{eulerprompt}
>x=normal(1,20);
\end{eulerprompt}
\begin{eulercomment}
Fungsi empdist(x,vs) memerlukan larik nilai yang telah diurutkan. Jadi
kita harus mengurutkan x sebelum kita bisa menggunakannya.
\end{eulercomment}
\begin{eulerprompt}
>xs=sort(x);
\end{eulerprompt}
\begin{eulercomment}
Kemudian kita membuat plot distribusi empiris dan beberapa batang
densitas dalam satu plot. Alih-alih plot batang untuk distribusi, kita
kali ini menggunakan plot gigi gergaji.
\end{eulercomment}
\begin{eulerprompt}
>figure(2,1); ...
>figure(1); plot2d("empdist",-4,4;xs); ...
>figure(2); plot2d(histo(x,v=-4:0.2:4,<bar));  ...
>figure(0):
\end{eulerprompt}
\eulerimg{15}{images/EMT4Statistika_Fadhila Asmaul Karimah-031.png}
\begin{eulercomment}
Plot titik sangat mudah dilakukan dalam Euler dengan plot titik biasa.
Grafik berikut menunjukkan bahwa X dan X+Y jelas positif terkorelasi.
\end{eulercomment}
\begin{eulerprompt}
>x=normal(1,100); plot2d(x,x+rotright(x),>points,style=".."):
\end{eulerprompt}
\eulerimg{15}{images/EMT4Statistika_Fadhila Asmaul Karimah-032.png}
\begin{eulercomment}
Seringkali, kita ingin membandingkan dua sampel dari distribusi yang
berbeda. Ini dapat dilakukan dengan plot kuantil-kuantil.

Untuk uji, kita mencoba distribusi t-student dan distribusi
eksponensial.
\end{eulercomment}
\begin{eulerprompt}
>x=randt(1,1000,5); y=randnormal(1,1000,mean(x),dev(x)); ...
>plot2d("x",r=6,style="--",yl="normal",xl="student-t",>vertical); ...
>plot2d(sort(x),sort(y),>points,color=red,style="x",>add):
\end{eulerprompt}
\eulerimg{15}{images/EMT4Statistika_Fadhila Asmaul Karimah-033.png}
\begin{eulercomment}
Plot ini jelas menunjukkan bahwa nilai yang terdistribusi normal
cenderung lebih kecil di ujung-ujung ekstrem.

Jika kita memiliki dua distribusi yang berbeda ukuran, kita dapat
memperluas yang lebih kecil atau menyusutkan yang lebih besar. Fungsi
berikut baik untuk keduanya. Ini mengambil nilai median dengan
persentase antara 0 dan 1.
\end{eulercomment}
\begin{eulerprompt}
>function medianexpand (x,n) := median(x,p=linspace(0,1,n-1));
\end{eulerprompt}
\begin{eulercomment}
Mari kita bandingkan dua distribusi yang sama.
\end{eulercomment}
\begin{eulerprompt}
>x=random(1000); y=random(400); ...
>plot2d("x",0,1,style="--"); ...
>plot2d(sort(medianexpand(x,400)),sort(y),>points,color=red,style="x",>add):
\end{eulerprompt}
\eulerimg{15}{images/EMT4Statistika_Fadhila Asmaul Karimah-034.png}
\eulerheading{Regresi dan Korelasi}
\begin{eulercomment}
Regresi linear dapat dilakukan dengan menggunakan fungsi polyfit()
atau berbagai fungsi fit lainnya.

Untuk memulainya, kita dapat menemukan garis regresi untuk data
univariat dengan polyfit(x, y, 1).
\end{eulercomment}
\begin{eulerprompt}
>x=1:10; y=[2,3,1,5,6,3,7,8,9,8]; writetable(x'|y',labc=["x","y"])
\end{eulerprompt}
\begin{euleroutput}
           x         y
           1         2
           2         3
           3         1
           4         5
           5         6
           6         3
           7         7
           8         8
           9         9
          10         8
\end{euleroutput}
\begin{eulercomment}
Kita ingin membandingkan hasil regresi tanpa bobot dan dengan bobot.
Pertama, kita tentukan koefisien regresi linear.
\end{eulercomment}
\begin{eulerprompt}
>p=polyfit(x,y,1)
\end{eulerprompt}
\begin{euleroutput}
  [0.733333,  0.812121]
\end{euleroutput}
\begin{eulercomment}
Sekarang, kita tentukan koefisien dengan bobot yang menekankan pada
nilai terakhir.
\end{eulercomment}
\begin{eulerprompt}
>w &= "exp(-(x-10)^2/10)"; pw=polyfit(x,y,1,w=w(x))
\end{eulerprompt}
\begin{euleroutput}
  [4.71566,  0.38319]
\end{euleroutput}
\begin{eulercomment}
Semua data dimasukkan ke dalam satu plot untuk titik-titik dan garis
regresi, serta untuk bobot yang digunakan.
\end{eulercomment}
\begin{eulerprompt}
>figure(2,1);  ...
>figure(1); statplot(x,y,"b",xl="Regression"); ...
>  plot2d("evalpoly(x,p)",>add,color=blue,style="--"); ...
>  plot2d("evalpoly(x,pw)",5,10,>add,color=red,style="--"); ...
>figure(2); plot2d(w,1,10,>filled,style="/",fillcolor=red,xl=w); ...
>figure(0):
\end{eulerprompt}
\eulerimg{15}{images/EMT4Statistika_Fadhila Asmaul Karimah-035.png}
\begin{eulercomment}
Sebagai contoh lain, kita membaca hasil survei mahasiswa, umur mereka,
umur orang tua, dan jumlah saudara dari sebuah file.

Tabel ini berisi "m" dan "f" di kolom kedua. Kita menggunakan variabel
tok2 untuk menentukan terjemahan yang tepat daripada membiarkan
readtable() mengumpulkan terjemahan.
\end{eulercomment}
\begin{eulerprompt}
>\{MS,hd\}:=readtable("table1.dat",tok2:=["m","f"]);  ...
>writetable(MS,labc=hd,tok2:=["m","f"]);
\end{eulerprompt}
\begin{euleroutput}
      Person       Sex       Age    Mother    Father  Siblings
           1         m        29        58        61         1
           2         f        26        53        54         2
           3         m        24        49        55         1
           4         f        25        56        63         3
           5         f        25        49        53         0
           6         f        23        55        55         2
           7         m        23        48        54         2
           8         m        27        56        58         1
           9         m        25        57        59         1
          10         m        24        50        54         1
          11         f        26        61        65         1
          12         m        24        50        52         1
          13         m        29        54        56         1
          14         m        28        48        51         2
          15         f        23        52        52         1
          16         m        24        45        57         1
          17         f        24        59        63         0
          18         f        23        52        55         1
          19         m        24        54        61         2
          20         f        23        54        55         1
\end{euleroutput}
\begin{eulercomment}
Bagaimana umur bergantung satu sama lain? Kesimpulan awal dapat
dilihat dari scatterplot berpasangan.
\end{eulercomment}
\begin{eulerprompt}
>scatterplots(tablecol(MS,3:5),hd[3:5]):
\end{eulerprompt}
\eulerimg{15}{images/EMT4Statistika_Fadhila Asmaul Karimah-036.png}
\begin{eulercomment}
Jelas bahwa umur ayah dan ibu saling bergantung. Mari tentukan dan
gambarkan garis regresi.
\end{eulercomment}
\begin{eulerprompt}
>cs:=MS[,4:5]'; ps:=polyfit(cs[1],cs[2],1)
\end{eulerprompt}
\begin{euleroutput}
  [17.3789,  0.740964]
\end{euleroutput}
\begin{eulercomment}
Ini jelas adalah model yang salah. Garis regresi seharusnya adalah
s=17+0.74t, di mana t adalah umur ibu dan s adalah umur ayah.
Perbedaan umur mungkin sedikit bergantung pada umur, tetapi tidak
begitu besar.

Sebaliknya, kita menduga ada fungsi seperti s=a+t. Maka a adalah
rata-rata dari s-t. Ini adalah perbedaan umur rata-rata antara ayah
dan ibu.
\end{eulercomment}
\begin{eulerprompt}
>da:=mean(cs[2]-cs[1])
\end{eulerprompt}
\begin{euleroutput}
  3.65
\end{euleroutput}
\begin{eulercomment}
Mari gambarkan ini dalam satu scatter plot.
\end{eulercomment}
\begin{eulerprompt}
>plot2d(cs[1],cs[2],>points);  ...
>plot2d("evalpoly(x,ps)",color=red,style=".",>add);  ...
>plot2d("x+da",color=blue,>add):
\end{eulerprompt}
\eulerimg{15}{images/EMT4Statistika_Fadhila Asmaul Karimah-037.png}
\begin{eulercomment}
Berikut adalah box plot dari kedua umur. Ini hanya menunjukkan bahwa
umur mereka berbeda.
\end{eulercomment}
\begin{eulerprompt}
>boxplot(cs,["mothers","fathers"]):
\end{eulerprompt}
\eulerimg{15}{images/EMT4Statistika_Fadhila Asmaul Karimah-038.png}
\begin{eulercomment}
Menariknya, perbedaan median tidak sebesar perbedaan rata-rata.
\end{eulercomment}
\begin{eulerprompt}
>median(cs[2])-median(cs[1])
\end{eulerprompt}
\begin{euleroutput}
  1.5
\end{euleroutput}
\begin{eulercomment}
Koefisien korelasi menunjukkan korelasi positif.
\end{eulercomment}
\begin{eulerprompt}
>correl(cs[1],cs[2])
\end{eulerprompt}
\begin{euleroutput}
  0.7588307236
\end{euleroutput}
\begin{eulercomment}
Korelasi peringkat adalah ukuran untuk urutan yang sama dalam kedua
vektor. Ini juga cukup positif.
\end{eulercomment}
\begin{eulerprompt}
>rankcorrel(cs[1],cs[2])
\end{eulerprompt}
\begin{euleroutput}
  0.758925292358
\end{euleroutput}
\eulerheading{Membuat Fungsi Baru}
\begin{eulercomment}
Tentu saja, bahasa EMT dapat digunakan untuk memprogram fungsi-fungsi
baru. Sebagai contoh, kita mendefinisikan fungsi skewness.

\end{eulercomment}
\begin{eulerformula}
\[
\text{sk}(x) = \dfrac{\sqrt{n} \sum_i (x_i-m)^3}{\left(\sum_i (x_i-m)^2\right)^{3/2}}
\]
\end{eulerformula}
\begin{eulercomment}
di mana m adalah rata-rata dari x.
\end{eulercomment}
\begin{eulerprompt}
>function skew (x:vector) ...
\end{eulerprompt}
\begin{eulerudf}
  m=mean(x);
  return sqrt(cols(x))*sum((x-m)^3)/(sum((x-m)^2))^(3/2);
  endfunction
\end{eulerudf}
\begin{eulercomment}
Seperti yang Anda lihat, kita dapat dengan mudah menggunakan bahasa
matriks untuk mendapatkan implementasi yang sangat singkat dan
efisien. Mari coba fungsi ini.
\end{eulercomment}
\begin{eulerprompt}
>data=normal(20); skew(normal(10))
\end{eulerprompt}
\begin{euleroutput}
  -0.198710316203
\end{euleroutput}
\begin{eulercomment}
Berikut adalah fungsi lain, yang disebut koefisien skewness Pearson.
\end{eulercomment}
\begin{eulerprompt}
>function skew1 (x) := 3*(mean(x)-median(x))/dev(x)
>skew1(data)
\end{eulerprompt}
\begin{euleroutput}
  -0.0801873249135
\end{euleroutput}
\eulerheading{Simulasi Monte Carlo }
\begin{eulercomment}
Euler dapat digunakan untuk mensimulasikan peristiwa acak. Kita sudah
melihat contoh sederhana di atas. Berikut adalah contoh lain, yang
mensimulasikan 1000 kali lemparan dadu 3 kali, dan meminta distribusi
jumlahnya.
\end{eulercomment}
\begin{eulerprompt}
>ds:=sum(intrandom(1000,3,6))';  fs=getmultiplicities(3:18,ds)
\end{eulerprompt}
\begin{euleroutput}
  [5,  17,  35,  44,  75,  97,  114,  116,  143,  116,  104,  53,  40,
  22,  13,  6]
\end{euleroutput}
\begin{eulercomment}
Sekarang kita bisa memplot ini.
\end{eulercomment}
\begin{eulerprompt}
>columnsplot(fs,lab=3:18):
\end{eulerprompt}
\eulerimg{15}{images/EMT4Statistika_Fadhila Asmaul Karimah-040.png}
\begin{eulercomment}
Untuk menentukan distribusi yang diharapkan tidak semudah itu. Kami
menggunakan rekursi canggih untuk ini.

Fungsi berikut menghitung jumlah cara di mana angka k dapat
direpresentasikan sebagai jumlah n angka dalam rentang 1 hingga m. Ini
bekerja secara rekursif dengan cara yang jelas.
\end{eulercomment}
\begin{eulerprompt}
>function map countways (k; n, m) ...
\end{eulerprompt}
\begin{eulerudf}
    if n==1 then return k>=1 && k<=m
    else
      sum=0; 
      loop 1 to m; sum=sum+countways(k-#,n-1,m); end;
      return sum;
    end;
  endfunction
\end{eulerudf}
\begin{eulercomment}
Berikut adalah hasilnya untuk tiga lemparan dadu.
\end{eulercomment}
\begin{eulerprompt}
>countways(5:25,5,5)
\end{eulerprompt}
\begin{euleroutput}
  [1,  5,  15,  35,  70,  121,  185,  255,  320,  365,  381,  365,  320,
  255,  185,  121,  70,  35,  15,  5,  1]
\end{euleroutput}
\begin{eulerprompt}
>cw=countways(3:18,3,6)
\end{eulerprompt}
\begin{euleroutput}
  [1,  3,  6,  10,  15,  21,  25,  27,  27,  25,  21,  15,  10,  6,  3,
  1]
\end{euleroutput}
\begin{eulercomment}
Kita tambahkan nilai yang diharapkan ke plot.
\end{eulercomment}
\begin{eulerprompt}
>plot2d(cw/6^3*1000,>add); plot2d(cw/6^3*1000,>points,>add):
\end{eulerprompt}
\eulerimg{15}{images/EMT4Statistika_Fadhila Asmaul Karimah-041.png}
\begin{eulercomment}
Untuk simulasi lain, deviasi dari nilai rata-rata dari n variabel acak
yang terdistribusi normal 0-1 adalah 1/sqrt(n).
\end{eulercomment}
\begin{eulerprompt}
>longformat; 1/sqrt(10)
\end{eulerprompt}
\begin{euleroutput}
  0.316227766017
\end{euleroutput}
\begin{eulercomment}
Mari kita periksa ini dengan simulasi. Kami menghasilkan 10000 kali 10
vektor acak.
\end{eulercomment}
\begin{eulerprompt}
>M=normal(10000,10); dev(mean(M)')
\end{eulerprompt}
\begin{euleroutput}
  0.319493614817
\end{euleroutput}
\begin{eulerprompt}
>plot2d(mean(M)',>distribution):
\end{eulerprompt}
\eulerimg{15}{images/EMT4Statistika_Fadhila Asmaul Karimah-042.png}
\begin{eulercomment}
Median dari 10 angka acak yang terdistribusi normal 0-1 memiliki
deviasi yang lebih besar.
\end{eulercomment}
\begin{eulerprompt}
>dev(median(M)')
\end{eulerprompt}
\begin{euleroutput}
  0.374460271535
\end{euleroutput}
\begin{eulercomment}
Karena kita dapat dengan mudah menghasilkan perjalanan acak, kita
dapat mensimulasikan proses Wiener. Kita ambil 1000 langkah dari 1000
proses. Kami kemudian memplot deviasi standar dan rata-rata langkah
ke-n dari proses-proses ini bersama dengan nilai yang diharapkan dalam
warna merah.
\end{eulercomment}
\begin{eulerprompt}
>n=1000; m=1000; M=cumsum(normal(n,m)/sqrt(m)); ...
>t=(1:n)/n; figure(2,1); ...
>figure(1); plot2d(t,mean(M')'); plot2d(t,0,color=red,>add); ...
>figure(2); plot2d(t,dev(M')'); plot2d(t,sqrt(t),color=red,>add); ...
>figure(0):
\end{eulerprompt}
\eulerimg{15}{images/EMT4Statistika_Fadhila Asmaul Karimah-043.png}
\eulerheading{Tests/Uji}
\begin{eulercomment}
Tests/uji adalah alat penting dalam statistika. Di Euler, banyak uji
diimplementasikan. Semua uji ini mengembalikan kesalahan yang kita
terima jika kita menolak hipotesis nol.

Sebagai contoh, kita menguji lemparan dadu untuk distribusi seragam.
Pada 600 lemparan, kita mendapatkan nilai-nilai berikut, yang kita
masukkan ke uji chi-square.
\end{eulercomment}
\begin{eulerprompt}
>chitest([90,103,114,101,103,89],dup(100,6)')
\end{eulerprompt}
\begin{euleroutput}
  0.498830517952
\end{euleroutput}
\begin{eulercomment}
Uji chi-square juga memiliki mode yang menggunakan simulasi Monte
Carlo untuk menguji statistik. Hasilnya seharusnya hampir sama.
Parameter \textgreater{}p mengartikan vektor y sebagai vektor probabilitas.
\end{eulercomment}
\begin{eulerprompt}
>chitest([90,103,114,101,103,89],dup(1/6,6)',>p,>montecarlo)
\end{eulerprompt}
\begin{euleroutput}
  0.526
\end{euleroutput}
\begin{eulercomment}
Kesalahan ini terlalu besar. Jadi kita tidak bisa menolak distribusi
seragam. Ini tidak membuktikan bahwa dadu kita adil. Tetapi kita tidak
bisa menolak hipotesis kita.

Selanjutnya kita menghasilkan 1000 lemparan dadu menggunakan generator
angka acak, dan melakukan uji yang sama.
\end{eulercomment}
\begin{eulerprompt}
>n=1000; t=random([1,n*6]); chitest(count(t*6,6),dup(n,6)')
\end{eulerprompt}
\begin{euleroutput}
  0.528028118442
\end{euleroutput}
\begin{eulercomment}
Mari uji untuk nilai rata-rata 100 dengan uji t.
\end{eulercomment}
\begin{eulerprompt}
>s=200+normal([1,100])*10; ...
>ttest(mean(s),dev(s),100,200)
\end{eulerprompt}
\begin{euleroutput}
  0.0218365848476
\end{euleroutput}
\begin{eulercomment}
Fungsi ttest() membutuhkan nilai rata-rata, deviasi, jumlah data, dan
nilai rata-rata yang akan diuji.

Sekarang mari kita periksa dua pengukuran untuk nilai rata-rata yang
sama. Kita menolak hipotesis bahwa mereka memiliki nilai rata-rata
yang sama, jika hasilnya \textless{}0,05.
\end{eulercomment}
\begin{eulerprompt}
>tcomparedata(normal(1,10),normal(1,10))
\end{eulerprompt}
\begin{euleroutput}
  0.38722000942
\end{euleroutput}
\begin{eulercomment}
Jika kita menambahkan bias ke salah satu distribusi, kita mendapatkan
lebih banyak penolakan. Ulangi simulasi ini beberapa kali untuk
melihat efeknya.
\end{eulercomment}
\begin{eulerprompt}
>tcomparedata(normal(1,10),normal(1,10)+2)
\end{eulerprompt}
\begin{euleroutput}
  5.60009101758e-07
\end{euleroutput}
\begin{eulercomment}
Pada contoh berikutnya, kita menghasilkan 20 lemparan dadu acak 100
kali dan menghitung jumlahnya. Harus ada 20/6=3.3 angka satu
rata-rata.
\end{eulercomment}
\begin{eulerprompt}
>R=random(100,20); R=sum(R*6<=1)'; mean(R)
\end{eulerprompt}
\begin{euleroutput}
  3.28
\end{euleroutput}
\begin{eulercomment}
Kemudian kita membandingkan jumlah angka satu dengan distribusi
binomial. Pertama kita plot distribusi angka satu.
\end{eulercomment}
\begin{eulerprompt}
>plot2d(R,distribution=max(R)+1,even=1,style="\(\backslash\)/"):
\end{eulerprompt}
\eulerimg{15}{images/EMT4Statistika_Fadhila Asmaul Karimah-044.png}
\begin{eulerprompt}
>t=count(R,21);
\end{eulerprompt}
\begin{eulercomment}
Kemudian kita menghitung nilai yang diharapkan.
\end{eulercomment}
\begin{eulerprompt}
>n=0:20; b=bin(20,n)*(1/6)^n*(5/6)^(20-n)*100;
\end{eulerprompt}
\begin{eulercomment}
Kita harus mengumpulkan beberapa angka untuk mendapatkan kategori yang
cukup besar.
\end{eulercomment}
\begin{eulerprompt}
>t1=sum(t[1:2])|t[3:7]|sum(t[8:21]); ...
>b1=sum(b[1:2])|b[3:7]|sum(b[8:21]);
\end{eulerprompt}
\begin{eulercomment}
Uji chi-square menolak hipotesis bahwa distribusi kita adalah
distribusi binomial, jika hasilnya \textless{}0,05.
\end{eulercomment}
\begin{eulerprompt}
>chitest(t1,b1)
\end{eulerprompt}
\begin{euleroutput}
  0.53921579764
\end{euleroutput}
\begin{eulercomment}
Contoh berikutnya berisi hasil dua kelompok orang (pria dan wanita,
misalnya) yang memilih salah satu dari enam partai.
\end{eulercomment}
\begin{eulerprompt}
>A=[23,37,43,52,64,74;27,39,41,49,63,76];  ...
>  writetable(A,wc=6,labr=["m","f"],labc=1:6)
\end{eulerprompt}
\begin{euleroutput}
             1     2     3     4     5     6
       m    23    37    43    52    64    74
       f    27    39    41    49    63    76
\end{euleroutput}
\begin{eulercomment}
Kita ingin menguji kemandirian suara dari jenis kelamin. Uji tabel
chi\textasciicircum{}2 melakukan ini. Hasilnya terlalu besar untuk menolak kemandirian.
Jadi kita tidak bisa mengatakan, apakah pemilihan tergantung pada
jenis kelamin dari data ini.
\end{eulercomment}
\begin{eulerprompt}
>tabletest(A)
\end{eulerprompt}
\begin{euleroutput}
  0.990701632326
\end{euleroutput}
\begin{eulercomment}
Berikut adalah tabel yang diharapkan, jika kita mengasumsikan
frekuensi pemilihan yang diamati.
\end{eulercomment}
\begin{eulerprompt}
>writetable(expectedtable(A),wc=6,dc=1,labr=["m","f"],labc=1:6)
\end{eulerprompt}
\begin{euleroutput}
             1     2     3     4     5     6
       m  24.9  37.9  41.9  50.3  63.3  74.7
       f  25.1  38.1  42.1  50.7  63.7  75.3
\end{euleroutput}
\begin{eulercomment}
Kita dapat menghitung koefisien kontingensi yang sudah dikoreksi.
Karena sangat dekat dengan 0, kita menyimpulkan bahwa pemilihan tidak
bergantung pada jenis kelamin.
\end{eulercomment}
\begin{eulerprompt}
>contingency(A)
\end{eulerprompt}
\begin{euleroutput}
  0.0427225484717
\end{euleroutput}
\begin{eulercomment}
\begin{eulercomment}
\eulerheading{Beberapa Uji Lainnya}
\begin{eulercomment}
Selanjutnya kita menggunakan analisis varian (uji F) untuk menguji
tiga sampel data yang terdistribusi normal untuk nilai rata-rata yang
sama. Metode ini disebut ANOVA (analisis varian). Di Euler, fungsi
varanalysis() digunakan.
\end{eulercomment}
\begin{eulerprompt}
>x1=[109,111,98,119,91,118,109,99,115,109,94]; mean(x1),
\end{eulerprompt}
\begin{euleroutput}
  106.545454545
\end{euleroutput}
\begin{eulerprompt}
>x2=[120,124,115,139,114,110,113,120,117]; mean(x2),
\end{eulerprompt}
\begin{euleroutput}
  119.111111111
\end{euleroutput}
\begin{eulerprompt}
>x3=[120,112,115,110,105,134,105,130,121,111]; mean(x3)
\end{eulerprompt}
\begin{euleroutput}
  116.3
\end{euleroutput}
\begin{eulerprompt}
>varanalysis(x1,x2,x3)
\end{eulerprompt}
\begin{euleroutput}
  0.0138048221371
\end{euleroutput}
\begin{eulercomment}
Ini berarti kita menolak hipotesis nilai rata-rata yang sama. Kita
melakukannya dengan tingkat kesalahan 1,3\%.

Ada juga uji median, yang menolak sampel data dengan distribusi
rata-rata yang berbeda dengan menguji median dari sampel yang
digabung.
\end{eulercomment}
\begin{eulerprompt}
>a=[56,66,68,49,61,53,45,58,54];
>b=[72,81,51,73,69,78,59,67,65,71,68,71];
>mediantest(a,b)
\end{eulerprompt}
\begin{euleroutput}
  0.0241724220052
\end{euleroutput}
\begin{eulercomment}
Uji kesetaraan lainnya adalah uji peringkat. Ini jauh lebih tajam
daripada uji median.
\end{eulercomment}
\begin{eulerprompt}
>ranktest(a,b)
\end{eulerprompt}
\begin{euleroutput}
  0.00199969612469
\end{euleroutput}
\begin{eulercomment}
Pada contoh berikut, kedua distribusi memiliki nilai rata-rata yang
sama.
\end{eulercomment}
\begin{eulerprompt}
>ranktest(random(1,100),random(1,50)*3-1)
\end{eulerprompt}
\begin{euleroutput}
  0.129608141484
\end{euleroutput}
\begin{eulercomment}
Mari kita coba mensimulasikan dua perlakuan a dan b yang diberikan
kepada orang yang berbeda.
\end{eulercomment}
\begin{eulerprompt}
>a=[8.0,7.4,5.9,9.4,8.6,8.2,7.6,8.1,6.2,8.9];
>b=[6.8,7.1,6.8,8.3,7.9,7.2,7.4,6.8,6.8,8.1];
\end{eulerprompt}
\begin{eulercomment}
Uji signum memutuskan apakah a lebih baik daripada b.
\end{eulercomment}
\begin{eulerprompt}
>signtest(a,b)
\end{eulerprompt}
\begin{euleroutput}
  0.0546875
\end{euleroutput}
\begin{eulercomment}
Ini terlalu banyak kesalahan. Kita tidak bisa menolak bahwa a sebaik
b.

Uji Wilcoxon lebih tajam daripada uji ini, tetapi bergantung pada
nilai kuantitatif dari perbedaan.
\end{eulercomment}
\begin{eulerprompt}
>wilcoxon(a,b)
\end{eulerprompt}
\begin{euleroutput}
  0.0296680599405
\end{euleroutput}
\begin{eulercomment}
Mari kita coba dua uji lagi menggunakan rangkaian yang dihasilkan.
\end{eulercomment}
\begin{eulerprompt}
>wilcoxon(normal(1,20),normal(1,20)-1)
\end{eulerprompt}
\begin{euleroutput}
  0.0068706451766
\end{euleroutput}
\begin{eulerprompt}
>wilcoxon(normal(1,20),normal(1,20))
\end{eulerprompt}
\begin{euleroutput}
  0.275145971064
\end{euleroutput}
\eulerheading{Angka Random }
\begin{eulercomment}
Berikut adalah uji coba untuk generator angka acak. Euler menggunakan
generator yang sangat baik, jadi kita tidak perlu mengharapkan masalah
apa pun.

Pertama, kita menghasilkan sepuluh juta angka acak dalam rentang
[0,1].
\end{eulercomment}
\begin{eulerprompt}
>n:=10000000; r:=random(1,n);
\end{eulerprompt}
\begin{eulercomment}
Selanjutnya, kita menghitung jarak antara dua angka kurang dari 0,05.
\end{eulercomment}
\begin{eulerprompt}
>a:=0.05; d:=differences(nonzeros(r<a));
\end{eulerprompt}
\begin{eulercomment}
Terakhir, kita plot jumlah kali masing-masing jarak terjadi dan
membandingkannya dengan nilai yang diharapkan.
\end{eulercomment}
\begin{eulerprompt}
>m=getmultiplicities(1:100,d); plot2d(m); ...
>  plot2d("n*(1-a)^(x-1)*a^2",color=red,>add):
\end{eulerprompt}
\eulerimg{15}{images/EMT4Statistika_Fadhila Asmaul Karimah-045.png}
\begin{eulercomment}
Hapus data.
\end{eulercomment}
\begin{eulerprompt}
>remvalue n;
\end{eulerprompt}
\begin{eulercomment}
Pengantar untuk Pengguna Proyek R\\
Jelas, EMT tidak bersaing dengan R sebagai paket statistik. Namun, ada
banyak prosedur statistik dan fungsi yang tersedia di EMT juga. Jadi,
EMT mungkin dapat memenuhi kebutuhan dasar. Pada dasarnya, EMT
dilengkapi dengan paket numerik dan sistem aljabar komputer.

Buku catatan ini untuk Anda jika Anda akrab dengan R, tetapi perlu
mengetahui perbedaan sintaks EMT dan R. Kami mencoba memberikan
gambaran tentang hal-hal yang jelas dan kurang jelas yang perlu Anda
ketahui.

Selain itu, kami melihat cara pertukaran data antara kedua sistem
tersebut.
\end{eulercomment}
\begin{eulercomment}
Perhatikan bahwa ini masih dalam proses.
\end{eulercomment}
\eulerheading{Syntax Dasar}
\begin{eulercomment}
Hal pertama yang Anda pelajari dalam R adalah membuat vektor. Di EMT,
perbedaan utama adalah operator ":" dapat mengambil langkah. Selain
itu, ia memiliki daya ikat rendah.
\end{eulercomment}
\begin{eulerprompt}
>n=10; 0:n/20:n-1
\end{eulerprompt}
\begin{euleroutput}
  [0,  0.5,  1,  1.5,  2,  2.5,  3,  3.5,  4,  4.5,  5,  5.5,  6,  6.5,
  7,  7.5,  8,  8.5,  9]
\end{euleroutput}
\begin{eulercomment}
Fungsi c() tidak ada. Mungkin untuk menggunakan vektor untuk
menggabungkan hal-hal.

Contoh berikut, seperti banyak contoh lainnya, berasal dari
"Pengenalan ke R" yang disertakan dalam proyek R. Jika Anda membaca
PDF ini, Anda akan menemukan bahwa saya mengikuti jalurnya dalam
tutorial ini.
\end{eulercomment}
\begin{eulerprompt}
>x=[10.4, 5.6, 3.1, 6.4, 21.7]; [x,0,x]
\end{eulerprompt}
\begin{euleroutput}
  [10.4,  5.6,  3.1,  6.4,  21.7,  0,  10.4,  5.6,  3.1,  6.4,  21.7]
\end{euleroutput}
\begin{eulercomment}
Operator kolon dengan langkah ukuran EMT digantikan oleh fungsi seq()
di R. Kita dapat menulis fungsi ini di EMT.
\end{eulercomment}
\begin{eulerprompt}
>function seq(a,b,c) := a:b:c; ...
>seq(0,-0.1,-1)
\end{eulerprompt}
\begin{euleroutput}
  [0,  -0.1,  -0.2,  -0.3,  -0.4,  -0.5,  -0.6,  -0.7,  -0.8,  -0.9,  -1]
\end{euleroutput}
\begin{eulercomment}
Fungsi rep() dari R tidak ada di EMT. Untuk input vektor, bisa ditulis
seperti berikut.
\end{eulercomment}
\begin{eulerprompt}
>function rep(x:vector,n:index) := flatten(dup(x,n)); ...
>rep(x,2)
\end{eulerprompt}
\begin{euleroutput}
  [10.4,  5.6,  3.1,  6.4,  21.7,  10.4,  5.6,  3.1,  6.4,  21.7]
\end{euleroutput}
\begin{eulercomment}
Perhatikan bahwa "=" atau ":=" digunakan untuk penugasan. Operator
"-\textgreater{}" digunakan untuk unit di EMT.
\end{eulercomment}
\begin{eulerprompt}
>125km -> " miles"
\end{eulerprompt}
\begin{euleroutput}
  77.6713990297 miles
\end{euleroutput}
\begin{eulercomment}
Operator "\textless{}-" untuk penugasan memang menyesatkan, dan bukan ide bagus
dari R. Berikut adalah perbandingan a dan -4 di EMT.
\end{eulercomment}
\begin{eulerprompt}
>a=2; a<-4
\end{eulerprompt}
\begin{euleroutput}
  0
\end{euleroutput}
\begin{eulercomment}
Di R, "a\textless{}-4\textless{}3" berhasil, tetapi "a\textless{}-4\textless{}-3" tidak. Saya memiliki
ambiguitas serupa di EMT juga, tetapi mencoba untuk menghilangkannya
sedikit demi sedikit.

EMT dan R memiliki vektor tipe boolean. Tetapi di EMT, angka 0 dan 1
digunakan untuk mewakili false dan true. Di R, nilai true dan false
masih dapat digunakan dalam aritmatika biasa seperti di EMT.
\end{eulercomment}
\begin{eulerprompt}
>x<5, %*x
\end{eulerprompt}
\begin{euleroutput}
  [0,  0,  1,  0,  0]
  [0,  0,  3.1,  0,  0]
\end{euleroutput}
\begin{eulercomment}
EMT menghasilkan kesalahan atau menghasilkan NAN tergantung pada flag
"errors".
\end{eulercomment}
\begin{eulerprompt}
>errors off; 0/0, isNAN(sqrt(-1)), errors on;
\end{eulerprompt}
\begin{euleroutput}
  NAN
  1
\end{euleroutput}
\begin{eulercomment}
String sama di R dan EMT. Keduanya berada dalam locale saat ini, bukan
dalam Unicode.

Di R, ada paket untuk Unicode. Di EMT, sebuah string dapat menjadi
string Unicode. Sebuah string unicode dapat diterjemahkan ke encoding
lokal dan sebaliknya. Selain itu, u"..." dapat berisi entitas HTML.
\end{eulercomment}
\begin{eulerprompt}
>u"&#169; Ren&eacut; Grothmann"
\end{eulerprompt}
\begin{euleroutput}
  © René Grothmann
\end{euleroutput}
\begin{eulercomment}
Berikut mungkin atau mungkin tidak ditampilkan dengan benar di sistem
Anda sebagai A dengan titik dan garis di atasnya. Itu tergantung pada
font yang Anda gunakan.
\end{eulercomment}
\begin{eulerprompt}
>chartoutf([480])
\end{eulerprompt}
\begin{euleroutput}
  Ǡ
\end{euleroutput}
\begin{eulercomment}
Penggabungan string dilakukan dengan "+" atau "\textbar{}". Ini dapat mencakup
angka, yang akan mencetak dalam format saat ini.
\end{eulercomment}
\begin{eulerprompt}
>"pi = "+pi
\end{eulerprompt}
\begin{euleroutput}
  pi = 3.14159265359
\end{euleroutput}
\eulerheading{Indeks}
\begin{eulercomment}
Sebagian besar waktu, ini akan berfungsi seperti di R.

Tetapi EMT akan mengartikan indeks negatif dari belakang vektor,
sementara R mengartikan x[n] sebagai x tanpa elemen ke-n.
\end{eulercomment}
\begin{eulerprompt}
>x, x[1:3], x[-2]
\end{eulerprompt}
\begin{euleroutput}
  [10.4,  5.6,  3.1,  6.4,  21.7]
  [10.4,  5.6,  3.1]
  6.4
\end{euleroutput}
\begin{eulercomment}
Perilaku R dapat dicapai di EMT dengan drop().
\end{eulercomment}
\begin{eulerprompt}
>drop(x,2)
\end{eulerprompt}
\begin{euleroutput}
  [10.4,  3.1,  6.4,  21.7]
\end{euleroutput}
\begin{eulercomment}
Vektor logika tidak dianggap berbeda sebagai indeks di EMT, berbeda
dengan R. Anda perlu mengekstrak elemen non-nol terlebih dahulu di
EMT.
\end{eulercomment}
\begin{eulerprompt}
>x, x>5, x[nonzeros(x>5)]
\end{eulerprompt}
\begin{euleroutput}
  [10.4,  5.6,  3.1,  6.4,  21.7]
  [1,  1,  0,  1,  1]
  [10.4,  5.6,  6.4,  21.7]
\end{euleroutput}
\begin{eulercomment}
Seperti halnya di R, vektor indeks dapat berisi pengulangan.
\end{eulercomment}
\begin{eulerprompt}
>x[[1,2,2,1]]
\end{eulerprompt}
\begin{euleroutput}
  [10.4,  5.6,  5.6,  10.4]
\end{euleroutput}
\begin{eulercomment}
Tetapi nama untuk indeks tidak mungkin di EMT. Untuk paket statistik,
ini mungkin sering diperlukan untuk mempermudah akses ke elemen
vektor.

Untuk meniru perilaku ini, kita dapat mendefinisikan fungsi seperti
berikut.
\end{eulercomment}
\begin{eulerprompt}
>function sel (v,i,s) := v[indexof(s,i)]; ...
>s=["first","second","third","fourth"]; sel(x,["first","third"],s)
\end{eulerprompt}
\begin{euleroutput}
  
  Trying to overwrite protected function sel!
  Error in:
  function sel (v,i,s) := v[indexof(s,i)]; ... ...
               ^
  
  Trying to overwrite protected function sel!
  Error in:
  function sel (v,i,s) := v[indexof(s,i)]; ... ...
               ^
  [10.4,  3.1]
\end{euleroutput}
\eulerheading{Jenis Data}
\begin{eulercomment}
EMT memiliki lebih banyak jenis data yang tetap dibandingkan dengan R.
Jelas, dalam R terdapat vektor yang terus berkembang. Anda dapat
menetapkan vektor numerik kosong v dan memberikan nilai pada elemen
v[17]. Hal ini tidak mungkin dilakukan dalam EMT.

Berikut adalah sedikit tidak efisien.
\end{eulercomment}
\begin{eulerprompt}
>v=[]; for i=1 to 10000; v=v|i; end;
\end{eulerprompt}
\begin{eulercomment}
EMT sekarang akan membuat vektor dengan v dan i ditambahkan ke
tumpukan dan menyalin vektor tersebut kembali ke variabel global v.

Lebih efisien jika sebelumnya telah menentukan vektor tersebut.
\end{eulercomment}
\begin{eulerprompt}
>v=zeros(10000); for i=1 to 10000; v[i]=i; end;
\end{eulerprompt}
\begin{eulercomment}
Untuk mengubah jenis data di EMT, Anda dapat menggunakan fungsi
seperti complex().
\end{eulercomment}
\begin{eulerprompt}
>complex(1:4)
\end{eulerprompt}
\begin{euleroutput}
  [ 1+0i ,  2+0i ,  3+0i ,  4+0i  ]
\end{euleroutput}
\begin{eulercomment}
Konversi ke string hanya mungkin untuk jenis data dasar. Format saat
ini digunakan untuk penggabungan string sederhana. Namun, ada fungsi
seperti print() atau frac().

Untuk vektor, Anda dapat dengan mudah menulis fungsi sendiri.
\end{eulercomment}
\begin{eulerprompt}
>function tostr (v) ...
\end{eulerprompt}
\begin{eulerudf}
  s="[";
  loop 1 to length(v);
     s=s+print(v[#],2,0);
     if #<length(v) then s=s+","; endif;
  end;
  return s+"]";
  endfunction
\end{eulerudf}
\begin{eulerprompt}
>tostr(linspace(0,1,10))
\end{eulerprompt}
\begin{euleroutput}
  [0.00,0.10,0.20,0.30,0.40,0.50,0.60,0.70,0.80,0.90,1.00]
\end{euleroutput}
\begin{eulercomment}
Untuk berkomunikasi dengan Maxima, ada fungsi convertmxm(), yang juga
dapat digunakan untuk memformat vektor untuk output.
\end{eulercomment}
\begin{eulerprompt}
>convertmxm(1:10)
\end{eulerprompt}
\begin{euleroutput}
  [1,2,3,4,5,6,7,8,9,10]
\end{euleroutput}
\begin{eulercomment}
Untuk Latex, perintah tex dapat digunakan untuk mendapatkan perintah
Latex.
\end{eulercomment}
\begin{eulerprompt}
>tex(&[1,2,3])
\end{eulerprompt}
\begin{euleroutput}
  \(\backslash\)left[ 1 , 2 , 3 \(\backslash\)right] 
\end{euleroutput}
\eulerheading{Faktor dan Tabel}
\begin{eulercomment}
Dalam pengantar ke R, ada contoh dengan faktor yang disebut.

Berikut adalah daftar wilayah dari 30 negara bagian.
\end{eulercomment}
\begin{eulerprompt}
>austates = ["tas", "sa", "qld", "nsw", "nsw", "nt", "wa", "wa", ...
>"qld", "vic", "nsw", "vic", "qld", "qld", "sa", "tas", ...
>"sa", "nt", "wa", "vic", "qld", "nsw", "nsw", "wa", ...
>"sa", "act", "nsw", "vic", "vic", "act"];
\end{eulerprompt}
\begin{eulercomment}
Anggap saja kita memiliki pendapatan yang sesuai di setiap negara
bagian.
\end{eulercomment}
\begin{eulerprompt}
>incomes = [60, 49, 40, 61, 64, 60, 59, 54, 62, 69, 70, 42, 56, ...
>61, 61, 61, 58, 51, 48, 65, 49, 49, 41, 48, 52, 46, ...
>59, 46, 58, 43];
\end{eulerprompt}
\begin{eulercomment}
Sekarang, kita ingin menghitung rata-rata pendapatan di
wilayah-wilayah tersebut. Sebagai program statistik, R memiliki
factor() dan tapply() untuk hal ini.

EMT dapat melakukan ini dengan menemukan indeks wilayah di daftar unik
wilayah.
\end{eulercomment}
\begin{eulerprompt}
>auterr=sort(unique(austates)); f=indexofsorted(auterr,austates)
\end{eulerprompt}
\begin{euleroutput}
  [6,  5,  4,  2,  2,  3,  8,  8,  4,  7,  2,  7,  4,  4,  5,  6,  5,  3,
  8,  7,  4,  2,  2,  8,  5,  1,  2,  7,  7,  1]
\end{euleroutput}
\begin{eulercomment}
Pada titik itu, kita dapat menulis fungsi loop sendiri untuk melakukan
hal-hal untuk satu faktor saja.

Atau kita dapat meniru fungsi tapply() dengan cara berikut.
\end{eulercomment}
\begin{eulerprompt}
>function map tappl (i; f$:call, cat, x) ...
\end{eulerprompt}
\begin{eulerudf}
  u=sort(unique(cat));
  f=indexof(u,cat);
  return f$(x[nonzeros(f==indexof(u,i))]);
  endfunction
\end{eulerudf}
\begin{eulercomment}
Ini agak tidak efisien, karena menghitung wilayah unik untuk setiap i,
tetapi ini berhasil.
\end{eulercomment}
\begin{eulerprompt}
>tappl(auterr,"mean",austates,incomes)
\end{eulerprompt}
\begin{euleroutput}
  [44.5,  57.3333333333,  55.5,  53.6,  55,  60.5,  56,  52.25]
\end{euleroutput}
\begin{eulercomment}
Perhatikan bahwa ini berfungsi untuk setiap vektor wilayah.
\end{eulercomment}
\begin{eulerprompt}
>tappl(["act","nsw"],"mean",austates,incomes)
\end{eulerprompt}
\begin{euleroutput}
  [44.5,  57.3333333333]
\end{euleroutput}
\begin{eulercomment}
Sekarang, paket statistik EMT menentukan tabel seperti halnya di R.
Fungsi readtable() dan writetable() dapat digunakan untuk input dan
output.

Jadi kita dapat mencetak pendapatan rata-rata negara bagian di wilayah
secara ramah.
\end{eulercomment}
\begin{eulerprompt}
>writetable(tappl(auterr,"mean",austates,incomes),labc=auterr,wc=7)
\end{eulerprompt}
\begin{euleroutput}
      act    nsw     nt    qld     sa    tas    vic     wa
     44.5  57.33   55.5   53.6     55   60.5     56  52.25
\end{euleroutput}
\begin{eulercomment}
Kita juga dapat mencoba meniru perilaku R sepenuhnya.

Faktor seharusnya jelas disimpan dalam koleksi dengan jenis dan
kategori (negara bagian dan wilayah dalam contoh kita). Untuk EMT,
kita menambahkan indeks yang telah dihitung sebelumnya.
\end{eulercomment}
\begin{eulerprompt}
>function makef (t) ...
\end{eulerprompt}
\begin{eulerudf}
  ## Factor data
  ## Returns a collection with data t, unique data, indices.
  ## See: tapply
  u=sort(unique(t));
  return \{\{t,u,indexofsorted(u,t)\}\};
  endfunction
\end{eulerudf}
\begin{eulerprompt}
>statef=makef(austates);
\end{eulerprompt}
\begin{eulercomment}
Sekarang elemen ketiga dalam koleksi akan berisi indeks.
\end{eulercomment}
\begin{eulerprompt}
>statef[3]
\end{eulerprompt}
\begin{euleroutput}
  [6,  5,  4,  2,  2,  3,  8,  8,  4,  7,  2,  7,  4,  4,  5,  6,  5,  3,
  8,  7,  4,  2,  2,  8,  5,  1,  2,  7,  7,  1]
\end{euleroutput}
\begin{eulercomment}
Sekarang kita dapat meniru tapply() dengan cara berikut. Ini akan
mengembalikan tabel sebagai koleksi data tabel dan judul kolom.
\end{eulercomment}
\begin{eulerprompt}
>function tapply (t:vector,tf,f$:call) ...
\end{eulerprompt}
\begin{eulerudf}
  ## Makes a table of data and factors
  ## tf : output of makef()
  ## See: makef
  uf=tf[2]; f=tf[3]; x=zeros(length(uf));
  for i=1 to length(uf);
     ind=nonzeros(f==i);
     if length(ind)==0 then x[i]=NAN;
     else x[i]=f$(t[ind]);
     endif;
  end;
  return \{\{x,uf\}\};
  endfunction
\end{eulerudf}
\begin{eulercomment}
Kami tidak menambahkan banyak pemeriksaan tipe di sini. Satu-satunya
tindakan pencegahan berkaitan dengan kategori (faktor) tanpa data.
Tetapi Anda harus memeriksa panjang t yang benar dan kebenaran koleksi
tf.

Tabel ini dapat dicetak sebagai tabel dengan writetable().
\end{eulercomment}
\begin{eulerprompt}
>writetable(tapply(incomes,statef,"mean"),wc=7)
\end{eulerprompt}
\begin{euleroutput}
      act    nsw     nt    qld     sa    tas    vic     wa
     44.5  57.33   55.5   53.6     55   60.5     56  52.25
\end{euleroutput}
\eulerheading{Array}
\begin{eulercomment}
EMT hanya memiliki dua dimensi untuk array. Jenis data ini disebut
matriks. Mudah untuk menulis fungsi untuk dimensi yang lebih tinggi
atau perpustakaan C untuk ini, bagaimanapun.

R memiliki lebih dari dua dimensi. Dalam R, array adalah vektor dengan
bidang dimensi.

Dalam EMT, vektor adalah matriks dengan satu baris. Ini dapat diubah
menjadi matriks dengan redim().
\end{eulercomment}
\begin{eulerprompt}
>shortformat; X=redim(1:20,4,5)
\end{eulerprompt}
\begin{euleroutput}
          1         2         3         4         5 
          6         7         8         9        10 
         11        12        13        14        15 
         16        17        18        19        20 
\end{euleroutput}
\begin{eulercomment}
Ekstraksi baris dan kolom, atau sub-matriks, mirip dengan di R.
\end{eulercomment}
\begin{eulerprompt}
>X[,2:3]
\end{eulerprompt}
\begin{euleroutput}
          2         3 
          7         8 
         12        13 
         17        18 
\end{euleroutput}
\begin{eulercomment}
Namun, di R, mungkin untuk menetapkan daftar indeks tertentu dari
vektor ke suatu nilai. Hal yang sama hanya mungkin di EMT dengan loop.
\end{eulercomment}
\begin{eulerprompt}
>function setmatrixvalue (M, i, j, v) ...
\end{eulerprompt}
\begin{eulerudf}
  loop 1 to max(length(i),length(j),length(v))
     M[i\{#\},j\{#\}] = v\{#\};
  end;
  endfunction
\end{eulerudf}
\begin{eulercomment}
Kami menunjukkan ini untuk menunjukkan bahwa matriks disalin dengan
referensi di EMT. Jika Anda tidak ingin mengubah matriks asli M, Anda
perlu menyalinnya di dalam fungsi.
\end{eulercomment}
\begin{eulerprompt}
>setmatrixvalue(X,1:3,3:-1:1,0); X,
\end{eulerprompt}
\begin{euleroutput}
          1         2         0         4         5 
          6         0         8         9        10 
          0        12        13        14        15 
         16        17        18        19        20 
\end{euleroutput}
\begin{eulercomment}
Produk luar di EMT hanya dapat dilakukan antara vektor. Ini otomatis
karena bahasa matriks. Satu vektor perlu menjadi vektor kolom dan yang
lainnya vektor baris.
\end{eulercomment}
\begin{eulerprompt}
>(1:5)*(1:5)'
\end{eulerprompt}
\begin{euleroutput}
          1         2         3         4         5 
          2         4         6         8        10 
          3         6         9        12        15 
          4         8        12        16        20 
          5        10        15        20        25 
\end{euleroutput}
\begin{eulercomment}
Dalam pengantar PDF untuk R ada contoh, yang menghitung distribusi
ab-cd untuk a,b,c,d yang dipilih dari 0 hingga n secara acak. Solusi
di R adalah membuat matriks 4 dimensi dan menjalankan fungsi table()
di atasnya.

Tentu saja, ini bisa dicapai dengan loop. Tetapi loop tidak efektif di
EMT atau R. Di EMT, kita bisa menulis loop di C dan itu akan menjadi
solusi tercepat.

Tetapi kita ingin meniru perilaku R. Untuk ini, kita perlu meratakan
perkalian ab dan membuat matriks ab-cd.
\end{eulercomment}
\begin{eulerprompt}
>a=0:6; b=a'; p=flatten(a*b); q=flatten(p-p'); ...
>u=sort(unique(q)); f=getmultiplicities(u,q); ...
>statplot(u,f,"h"):
\end{eulerprompt}
\eulerimg{15}{images/EMT4Statistika_Fadhila Asmaul Karimah-046.png}
\begin{eulercomment}
Selain perkalian yang tepat, EMT dapat menghitung frekuensi dalam
vektor.
\end{eulercomment}
\begin{eulerprompt}
>getfrequencies(q,-50:10:50)
\end{eulerprompt}
\begin{euleroutput}
  [0,  23,  132,  316,  602,  801,  333,  141,  53,  0]
\end{euleroutput}
\begin{eulercomment}
Cara paling mudah untuk memplot ini sebagai distribusi adalah sebagai
berikut.
\end{eulercomment}
\begin{eulerprompt}
>plot2d(q,distribution=11):
\end{eulerprompt}
\eulerimg{15}{images/EMT4Statistika_Fadhila Asmaul Karimah-047.png}
\begin{eulercomment}
Tetapi juga mungkin untuk mendahului perhitungan jumlah dalam interval
yang dipilih sebelumnya. Tentu saja, ini menggunakan getfrequencies()
secara internal.

Karena fungsi histo() mengembalikan frekuensi, kita perlu menskalakan
ini sehingga integral di bawah diagram batang menjadi 1.
\end{eulercomment}
\begin{eulerprompt}
>\{x,y\}=histo(q,v=-55:10:55); y=y/sum(y)/differences(x); ...
>plot2d(x,y,>bar,style="/"):
\end{eulerprompt}
\eulerimg{15}{images/EMT4Statistika_Fadhila Asmaul Karimah-048.png}
\eulerheading{Daftar}
\begin{eulercomment}
EMT memiliki dua jenis daftar. Satu adalah daftar global yang dapat
diubah, dan yang lainnya adalah tipe daftar yang tidak dapat diubah.
Kami tidak peduli tentang daftar global di sini.

Tipe daftar yang tidak dapat diubah disebut koleksi dalam EMT. Ini
berperilaku seperti struktur dalam C, tetapi elemennya hanya dinomori
dan tidak dinamai.
\end{eulercomment}
\begin{eulerprompt}
>L=\{\{"Fred","Flintstone",40,[1990,1992]\}\}
\end{eulerprompt}
\begin{euleroutput}
  Fred
  Flintstone
  40
  [1990,  1992]
\end{euleroutput}
\begin{eulercomment}
Saat ini elemen tidak memiliki nama, meskipun nama dapat diatur untuk
tujuan khusus. Mereka diakses dengan nomor.
\end{eulercomment}
\begin{eulerprompt}
>(L[4])[2]
\end{eulerprompt}
\begin{euleroutput}
  1992
\end{euleroutput}
\begin{eulercomment}
\begin{eulercomment}
\eulerheading{Input dan Output File (Membaca dan Menulis Data)}
\begin{eulercomment}
Anda seringkali ingin mengimpor matriks data dari sumber lain ke EMT.
Tutorial ini memberi tahu Anda tentang banyak cara untuk mencapainya.
Fungsi sederhana adalah writematrix() dan readmatrix().

Mari kita tunjukkan bagaimana cara membaca dan menulis vektor bilangan
riil ke dalam file.
\end{eulercomment}
\begin{eulerprompt}
>a=random(1,100); mean(a), dev(a),
\end{eulerprompt}
\begin{euleroutput}
  0.49815
  0.28037
\end{euleroutput}
\begin{eulercomment}
Untuk menulis data ke dalam berkas, kita menggunakan fungsi
writematrix().

Karena pengantar ini kemungkinan besar berada dalam direktori di mana
pengguna tidak memiliki akses penulisan, kita menulis data ke
direktori rumah pengguna. Untuk notebook sendiri, ini tidak perlu
dilakukan, karena berkas data akan ditulis ke dalam direktori yang
sama.
\end{eulercomment}
\begin{eulerprompt}
>filename="test.dat";
\end{eulerprompt}
\begin{eulercomment}
Sekarang kita menulis vektor kolom a' ke dalam berkas. Ini
menghasilkan satu angka dalam setiap baris file.
\end{eulercomment}
\begin{eulerprompt}
>writematrix(a',filename);
\end{eulerprompt}
\begin{eulercomment}
Untuk membaca data, kita menggunakan readmatrix().
\end{eulercomment}
\begin{eulerprompt}
>a=readmatrix(filename)';
\end{eulerprompt}
\begin{eulercomment}
Dan hapus file tersebut.
\end{eulercomment}
\begin{eulerprompt}
>fileremove(filename);
>mean(a), dev(a),
\end{eulerprompt}
\begin{euleroutput}
  0.49815
  0.28037
\end{euleroutput}
\begin{eulercomment}
Fungsi writematrix() atau writetable() dapat dikonfigurasi untuk
bahasa lain.

Contohnya, jika Anda memiliki sistem berbahasa Indonesia (titik
desimal dengan koma), Excel Anda memerlukan nilai dengan koma desimal
yang dipisahkan oleh titik koma dalam berkas csv (defaultnya adalah
nilai yang dipisahkan oleh koma). Berkas "test.csv" berikut seharusnya
muncul di folder saat ini.
\end{eulercomment}
\begin{eulerprompt}
>filename="test.csv"; ...
>writematrix(random(5,3),file=filename,separator=",");
\end{eulerprompt}
\begin{eulercomment}
Anda sekarang dapat membuka berkas ini dengan Excel berbahasa
Indonesia secara langsung.
\end{eulercomment}
\begin{eulerprompt}
>fileremove(filename);
\end{eulerprompt}
\begin{eulercomment}
Kadang-kadang kita memiliki string dengan token seperti berikut.
\end{eulercomment}
\begin{eulerprompt}
>s1:="f m m f m m m f f f m m f";  ...
>s2:="f f f m m f f";
\end{eulerprompt}
\begin{eulercomment}
Untuk melakukan tokenisasi ini, kita tentukan vektor token.
\end{eulercomment}
\begin{eulerprompt}
>tok:=["f","m"]
\end{eulerprompt}
\begin{euleroutput}
  f
  m
\end{euleroutput}
\begin{eulercomment}
Kemudian kita dapat menghitung berapa kali setiap token muncul dalam
string, dan menempatkan hasilnya ke dalam tabel.
\end{eulercomment}
\begin{eulerprompt}
>M:=getmultiplicities(tok,strtokens(s1))_ ...
>  getmultiplicities(tok,strtokens(s2));
\end{eulerprompt}
\begin{eulercomment}
Tulis tabel dengan header token.
\end{eulercomment}
\begin{eulerprompt}
>writetable(M,labc=tok,labr=1:2,wc=8)
\end{eulerprompt}
\begin{euleroutput}
                 f       m
         1       6       7
         2       5       2
\end{euleroutput}
\begin{eulercomment}
Untuk statistik, EMT dapat membaca dan menulis tabel.
\end{eulercomment}
\begin{eulerprompt}
>file="test.dat"; open(file,"w"); ...
>writeln("A,B,C"); writematrix(random(3,3)); ...
>close();
\end{eulerprompt}
\begin{eulercomment}
Berkas terlihat seperti ini.
\end{eulercomment}
\begin{eulerprompt}
>printfile(file)
\end{eulerprompt}
\begin{euleroutput}
  A,B,C
  0.7003664386138074,0.1875530821001213,0.3262339279660414
  0.5926249243193858,0.1522927283984059,0.368140583062521
  0.8065535209872989,0.7265910840408142,0.7332619844597152
  
\end{euleroutput}
\begin{eulercomment}
Fungsi readtable() dalam bentuk paling sederhana dapat membaca ini dan
mengembalikan kumpulan nilai dan baris judul.
\end{eulercomment}
\begin{eulerprompt}
>L=readtable(file,>list);
\end{eulerprompt}
\begin{eulercomment}
Kumpulan ini dapat dicetak dengan writetable() ke notebook, atau ke
dalam berkas.
\end{eulercomment}
\begin{eulerprompt}
>writetable(L,wc=10,dc=5)
\end{eulerprompt}
\begin{euleroutput}
           A         B         C
     0.70037   0.18755   0.32623
     0.59262   0.15229   0.36814
     0.80655   0.72659   0.73326
\end{euleroutput}
\begin{eulercomment}
Matriks nilai adalah elemen pertama dari L. Perhatikan bahwa mean() di
EMT menghitung nilai rata-rata dari baris matriks.
\end{eulercomment}
\begin{eulerprompt}
>mean(L[1])
\end{eulerprompt}
\begin{euleroutput}
    0.40472 
    0.37102 
    0.75547 
\end{euleroutput}
\eulerheading{File CSV}
\begin{eulercomment}
Pertama, mari kita menulis matriks ke dalam file. Untuk keluaran, kita
menghasilkan file di direktori kerja saat ini.
\end{eulercomment}
\begin{eulerprompt}
>file="test.csv";  ...
>M=random(3,3); writematrix(M,file);
\end{eulerprompt}
\begin{eulercomment}
Berikut adalah isi file ini.
\end{eulercomment}
\begin{eulerprompt}
>printfile(file)
\end{eulerprompt}
\begin{euleroutput}
  0.8221197733097619,0.821531098722547,0.7771240608094004
  0.8482947121863489,0.3237767724883862,0.6501422353377985
  0.1482301827518109,0.3297459716109594,0.6261901074210923
  
\end{euleroutput}
\begin{eulercomment}
CSV ini dapat dibuka pada sistem berbahasa Inggris ke dalam Excel
dengan double click. Jika Anda mendapatkan file seperti itu pada
sistem berbahasa Jerman, Anda perlu mengimpor data ke dalam Excel
dengan memperhatikan titik desimal.

Tetapi titik desimal adalah format default untuk EMT juga. Anda dapat
membaca matriks dari file dengan readmatrix().
\end{eulercomment}
\begin{eulerprompt}
>readmatrix(file)
\end{eulerprompt}
\begin{euleroutput}
    0.82212   0.82153   0.77712 
    0.84829   0.32378   0.65014 
    0.14823   0.32975   0.62619 
\end{euleroutput}
\begin{eulercomment}
Mungkin menulis beberapa matriks ke dalam satu berkas. Perintah open()
dapat membuka file untuk penulisan dengan parameter "w". Defaultnya
adalah "r" untuk membaca.
\end{eulercomment}
\begin{eulerprompt}
>open(file,"w"); writematrix(M); writematrix(M'); close();
\end{eulerprompt}
\begin{eulercomment}
Matriks-matriks ini dipisahkan oleh baris kosong. Untuk membaca
matriks, buka berkas dan panggil readmatrix() beberapa kali.
\end{eulercomment}
\begin{eulerprompt}
>open(file); A=readmatrix(); B=readmatrix(); A==B, close();
\end{eulerprompt}
\begin{euleroutput}
          1         0         0 
          0         1         0 
          0         0         1 
\end{euleroutput}
\begin{eulercomment}
Di Excel atau spreadsheet serupa, Anda dapat mengekspor matriks
sebagai CSV (comma separated values). Pada Excel 2007, gunakan "save
as" dan "other formats", lalu pilih "CSV". Pastikan tabel saat ini
hanya berisi data yang ingin Anda ekspor.

Berikut adalah contohnya.
\end{eulercomment}
\begin{eulerprompt}
>printfile("excel-data.csv")
\end{eulerprompt}
\begin{euleroutput}
  0;1000;1000
  1;1051,271096;1072,508181
  2;1105,170918;1150,273799
  3;1161,834243;1233,67806
  4;1221,402758;1323,129812
  5;1284,025417;1419,067549
  6;1349,858808;1521,961556
  7;1419,067549;1632,31622
  8;1491,824698;1750,6725
  9;1568,312185;1877,610579
  10;1648,721271;2013,752707
\end{euleroutput}
\begin{eulercomment}
Seperti yang Anda lihat, sistem Jerman saya menggunakan titik koma
sebagai pemisah dan koma desimal. Anda dapat mengubahnya dalam
pengaturan sistem atau di Excel, tetapi itu tidak diperlukan untuk
membaca matriks ke dalam EMT.

Cara termudah untuk membaca ini ke dalam Euler adalah dengan
readmatrix(). Semua koma digantikan oleh titik dengan parameter
\textgreater{}comma. Untuk CSV berbahasa Inggris, cukup hilangkan parameter ini.
\end{eulercomment}
\begin{eulerprompt}
>M=readmatrix("excel-data.csv",>comma)
\end{eulerprompt}
\begin{euleroutput}
          0      1000      1000 
          1    1051.3    1072.5 
          2    1105.2    1150.3 
          3    1161.8    1233.7 
          4    1221.4    1323.1 
          5      1284    1419.1 
          6    1349.9      1522 
          7    1419.1    1632.3 
          8    1491.8    1750.7 
          9    1568.3    1877.6 
         10    1648.7    2013.8 
\end{euleroutput}
\begin{eulercomment}
Mari kita gambarkan plot
\end{eulercomment}
\begin{eulerprompt}
>plot2d(M'[1],M'[2:3],>points,color=[red,green]'):
\end{eulerprompt}
\eulerimg{15}{images/EMT4Statistika_Fadhila Asmaul Karimah-049.png}
\begin{eulercomment}
Ada cara lebih mendasar untuk membaca data dari file. Anda dapat
membuka berkas dan membaca angka-angka baris per baris. Fungsi
getvectorline() akan membaca angka-angka dari sebuah baris data.
Secara default, ia mengharapkan titik desimal. Tetapi juga dapat
menggunakan koma desimal, jika Anda memanggil setdecimaldot(",")
sebelum Anda menggunakan fungsi ini.

Fungsi berikut adalah contoh untuk ini. Ini akan berhenti di akhir
file atau baris kosong.
\end{eulercomment}
\begin{eulerprompt}
>function myload (file) ...
\end{eulerprompt}
\begin{eulerudf}
  open(file);
  M=[];
  repeat
     until eof();
     v=getvectorline(3);
     if length(v)>0 then M=M_v; else break; endif;
  end;
  return M;
  close(file);
  endfunction
\end{eulerudf}
\begin{eulerprompt}
>myload(file)
\end{eulerprompt}
\begin{euleroutput}
    0.82212   0.82153   0.77712 
    0.84829   0.32378   0.65014 
    0.14823   0.32975   0.62619 
\end{euleroutput}
\begin{eulercomment}
Juga mungkin untuk membaca semua angka dalam file itu dengan
getvector().
\end{eulercomment}
\begin{eulerprompt}
>open(file); v=getvector(10000); close(); redim(v[1:9],3,3)
\end{eulerprompt}
\begin{euleroutput}
    0.82212   0.82153   0.77712 
    0.84829   0.32378   0.65014 
    0.14823   0.32975   0.62619 
\end{euleroutput}
\begin{eulercomment}
Dengan demikian, sangat mudah untuk menyimpan vektor nilai, satu nilai
dalam setiap baris, dan membaca kembali vektor ini.
\end{eulercomment}
\begin{eulerprompt}
>v=random(1000); mean(v)
\end{eulerprompt}
\begin{euleroutput}
  0.50303
\end{euleroutput}
\begin{eulerprompt}
>writematrix(v',file); mean(readmatrix(file)')
\end{eulerprompt}
\begin{euleroutput}
  0.50303
\end{euleroutput}
\eulerheading{Menggunakan Tabel}
\begin{eulercomment}
Tabel dapat digunakan untuk membaca atau menulis data numerik. Sebagai
contoh, kita menulis tabel dengan judul baris dan kolom ke dalam file.
\end{eulercomment}
\begin{eulerprompt}
>file="test.tab"; M=random(3,3);  ...
>open(file,"w");  ...
>writetable(M,separator=",",labc=["one","two","three"]);  ...
>close(); ...
>printfile(file)
\end{eulerprompt}
\begin{euleroutput}
  one,two,three
        0.09,      0.39,      0.86
        0.39,      0.86,      0.71
         0.2,      0.02,      0.83
\end{euleroutput}
\begin{eulercomment}
Ini dapat diimpor ke Excel.

Untuk membaca file di EMT, kita menggunakan readtable().
\end{eulercomment}
\begin{eulerprompt}
>\{M,headings\}=readtable(file,>clabs); ...
>writetable(M,labc=headings)
\end{eulerprompt}
\begin{euleroutput}
         one       two     three
        0.09      0.39      0.86
        0.39      0.86      0.71
         0.2      0.02      0.83
\end{euleroutput}
\eulerheading{Menganalisis Sebuah Baris}
\begin{eulercomment}
Anda bahkan dapat mengevaluasi setiap baris secara manual. Misalkan,
kita memiliki baris dengan format berikut.
\end{eulercomment}
\begin{eulerprompt}
>line="2020-11-03,Tue,1'114.05"
\end{eulerprompt}
\begin{euleroutput}
  2020-11-03,Tue,1'114.05
\end{euleroutput}
\begin{eulercomment}
Pertama, kita bisa melakukan tokenisasi pada baris tersebut.
\end{eulercomment}
\begin{eulerprompt}
>vt=strtokens(line)
\end{eulerprompt}
\begin{euleroutput}
  2020-11-03
  Tue
  1'114.05
\end{euleroutput}
\begin{eulercomment}
Kemudian kita dapat mengevaluasi setiap elemen dari baris menggunakan
evaluasi yang sesuai.
\end{eulercomment}
\begin{eulerprompt}
>day(vt[1]),  ...
>indexof(["mon","tue","wed","thu","fri","sat","sun"],tolower(vt[2])),  ...
>strrepl(vt[3],"'","")()
\end{eulerprompt}
\begin{euleroutput}
  7.3816e+05
  2
  1114
\end{euleroutput}
\begin{eulercomment}
Dengan menggunakan ekspresi reguler, mungkin untuk mengekstrak hampir
semua informasi dari sebuah baris data.

Misalkan kita memiliki baris berikut dalam dokumen HTML.
\end{eulercomment}
\begin{eulerprompt}
>line="<tr><td>1145.45</td><td>5.6</td><td>-4.5</td><tr>"
\end{eulerprompt}
\begin{euleroutput}
  <tr><td>1145.45</td><td>5.6</td><td>-4.5</td><tr>
\end{euleroutput}
\begin{eulercomment}
Untuk mengekstrak ini, kita menggunakan ekspresi reguler yang mencari

- tanda kurung sudut penutup \textgreater{},\\
- string apa pun yang tidak mengandung tanda kurung dengan
sub-pencocokan "(...)",\\
- tanda kurung buka dan penutup menggunakan solusi terpendek,\\
- sekali lagi string apa pun yang tidak mengandung tanda kurung,\\
- dan tanda kurung buka \textless{}.

Ekspresi reguler agak sulit untuk dipelajari tetapi sangat kuat.
\end{eulercomment}
\begin{eulerprompt}
>\{pos,s,vt\}=strxfind(line,">([^<>]+)<.+?>([^<>]+)<");
\end{eulerprompt}
\begin{eulercomment}
Hasilnya adalah posisi match, string yang cocok, dan vektor string
untuk sub-match.
\end{eulercomment}
\begin{eulerprompt}
>for k=1:length(vt); vt[k](), end;
\end{eulerprompt}
\begin{euleroutput}
  1145.5
  5.6
\end{euleroutput}
\begin{eulercomment}
Berikut adalah fungsi yang membaca semua item numerik antara \textless{}td\textgreater{} dan
\textless{}/td\textgreater{}.
\end{eulercomment}
\begin{eulerprompt}
>function readtd (line) ...
\end{eulerprompt}
\begin{eulerudf}
  v=[]; cp=0;
  repeat
     \{pos,s,vt\}=strxfind(line,"<td.*?>(.+?)</td>",cp);
     until pos==0;
     if length(vt)>0 then v=v|vt[1]; endif;
     cp=pos+strlen(s);
  end;
  return v;
  endfunction
\end{eulerudf}
\begin{eulerprompt}
>readtd(line+"<td>non-numerical</td>")
\end{eulerprompt}
\begin{euleroutput}
  1145.45
  5.6
  -4.5
  non-numerical
\end{euleroutput}
\eulerheading{Membaca dari Web}
\begin{eulercomment}
Situs web atau file dengan URL dapat dibuka di EMT dan dapat dibaca
per baris.

Pada contoh ini, kita membaca versi terbaru dari situs EMT. Kita
menggunakan ekspresi reguler untuk mencari "Versi ..." dalam judul.
\end{eulercomment}
\begin{eulerprompt}
>function readversion () ...
\end{eulerprompt}
\begin{eulerudf}
  urlopen("http://www.euler-math-toolbox.de/Programs/Changes.html");
  repeat
    until urleof();
    s=urlgetline();
    k=strfind(s,"Version ",1);
    if k>0 then substring(s,k,strfind(s,"<",k)-1), break; endif;
  end;
  urlclose();
  endfunction
\end{eulerudf}
\begin{eulerprompt}
>readversion
\end{eulerprompt}
\begin{euleroutput}
  Version 2022-05-18
\end{euleroutput}
\eulerheading{Input dan Output Variabel}
\begin{eulercomment}
Anda dapat menulis variabel dalam bentuk definisi Euler ke file atau
ke baris perintah.
\end{eulercomment}
\begin{eulerprompt}
>writevar(pi,"mypi");
\end{eulerprompt}
\begin{euleroutput}
  mypi = 3.141592653589793;
\end{euleroutput}
\begin{eulercomment}
Untuk uji coba, kita membuat file Euler di direktori kerja EMT.
\end{eulercomment}
\begin{eulerprompt}
>file="test.e"; ...
>writevar(random(2,2),"M",file); ...
>printfile(file,3)
\end{eulerprompt}
\begin{euleroutput}
  M = [ ..
  0.5991820585590205, 0.7960280262224293;
  0.5167243983231363, 0.2996684599070898];
\end{euleroutput}
\begin{eulercomment}
Sekarang kita bisa memuat file tersebut. Ini akan mendefinisikan
matriks M.
\end{eulercomment}
\begin{eulerprompt}
>load(file); show M,
\end{eulerprompt}
\begin{euleroutput}
  M = 
    0.59918   0.79603 
    0.51672   0.29967 
\end{euleroutput}
\begin{eulercomment}
By the way, jika writevar() digunakan pada suatu variabel, itu akan
mencetak definisi variabel dengan nama variabel ini.
\end{eulercomment}
\begin{eulerprompt}
>writevar(M); writevar(inch$)
\end{eulerprompt}
\begin{euleroutput}
  M = [ ..
  0.5991820585590205, 0.7960280262224293;
  0.5167243983231363, 0.2996684599070898];
  inch$ = 0.0254;
\end{euleroutput}
\begin{eulercomment}
Kita juga bisa membuka file baru atau menambahkan ke file yang sudah
ada. Pada contoh ini, kita menambahkan ke file yang telah dibuat
sebelumnya.
\end{eulercomment}
\begin{eulerprompt}
>open(file,"a"); ...
>writevar(random(2,2),"M1"); ...
>writevar(random(3,1),"M2"); ...
>close();
>load(file); show M1; show M2;
\end{eulerprompt}
\begin{euleroutput}
  M1 = 
    0.30287   0.15372 
     0.7504   0.75401 
  M2 = 
    0.27213 
   0.053211 
    0.70249 
\end{euleroutput}
\begin{eulercomment}
Untuk menghapus file apa pun, gunakan fileremove().
\end{eulercomment}
\begin{eulerprompt}
>fileremove(file);
\end{eulerprompt}
\begin{eulercomment}
Vektor baris dalam sebuah file tidak memerlukan koma, jika setiap
angka berada di baris baru. Mari kita hasilkan file seperti itu,
menulis setiap baris satu per satu dengan writeln().
\end{eulercomment}
\begin{eulerprompt}
>open(file,"w"); writeln("M = ["); ...
>for i=1 to 5; writeln(""+random()); end; ...
>writeln("];"); close(); ...
>printfile(file)
\end{eulerprompt}
\begin{euleroutput}
  M = [
  0.344851384551
  0.0807510017715
  0.876519562911
  0.754157709472
  0.688392638934
  ];
\end{euleroutput}
\begin{eulerprompt}
>load(file); M
\end{eulerprompt}
\begin{euleroutput}
  [0.34485,  0.080751,  0.87652,  0.75416,  0.68839]
\end{euleroutput}
\begin{eulercomment}
\begin{eulercomment}
\eulerheading{Contoh Soal}
\begin{eulercomment}
1. Pada suatu kelas berisi 50 mahasiswa, didapatkan nilai ujian akhir
sebagai berikut:

60,50,60,75,60,55,80,60,50,90,\\
50,65,70,80,70,40,50,65,45,45,\\
40,45,60,70,70,80,90,75,60,80,\\
70,75,80,70,70,60,50,85,85,60,\\
40,45,50,70,90,70,60,75,65,60

Buatlah distribusi frekuensi berdasarkan data di atas!

Penyelesaian:

Data yang sudah diurutkan mnejadi:

40,40,40,45,45,45,45,50,50,50,\\
50,50,50,55,60,60,60,60,60,60,\\
60,60,60,60,65,65,65,70,70,70,\\
70,70,70,70,70,70,75,75,75,75,\\
80,80,80,80,80,85,85,90,90,90

- Menentukan range\\
\end{eulercomment}
\begin{eulerttcomment}
     range= 90-40
          = 50
\end{eulerttcomment}
\begin{eulercomment}

- Menentukan banyak kelas\\
\end{eulercomment}
\begin{eulerttcomment}
     banyak kelas= 1+3,3 log n    , n adalah banyak data
                 = 1+3,3 log 50
                 = 6,60
                 = 7
\end{eulerttcomment}
\begin{eulercomment}

- Menentukan panjang kelas\\
\end{eulercomment}
\begin{eulerformula}
\[
p= \frac{range}{banyak \quad kelas}
\]
\end{eulerformula}
\begin{eulerformula}
\[
p= \frac{50}{7}
\]
\end{eulerformula}
\begin{eulerformula}
\[
p= 7,14 = 8
\]
\end{eulerformula}
\begin{eulercomment}
- Menentukan batas bawah\\
\end{eulercomment}
\begin{eulerformula}
\[
40-0,5 = 39,5
\]
\end{eulerformula}
\begin{eulercomment}
- Menentukan batas atas\\
\end{eulercomment}
\begin{eulerformula}
\[
90+0,5 = 90,5
\]
\end{eulerformula}
\begin{eulerprompt}
>r=39.5:8:95.5; v=[7,7,10,12,4,7,3];
>T:=r[1:7]' | r[2:8]' | v'; writetable(T,labc=["TB","TA","Frekuensi"])
\end{eulerprompt}
\begin{euleroutput}
          TB        TA Frekuensi
        39.5      47.5         7
        47.5      55.5         7
        55.5      63.5        10
        63.5      71.5        12
        71.5      79.5         4
        79.5      87.5         7
        87.5      95.5         3
\end{euleroutput}
\begin{eulercomment}
Mencari titik tengah
\end{eulercomment}
\begin{eulerprompt}
>(T[,1]+T[,2])/2
\end{eulerprompt}
\begin{euleroutput}
       43.5 
       51.5 
       59.5 
       67.5 
       75.5 
       83.5 
       91.5 
\end{euleroutput}
\begin{eulercomment}
Sajian dalam bentuk diagram
\end{eulercomment}
\begin{eulerprompt}
>plot2d(r,v,a=40,b=100,c=0,d=20,bar=1,style="\(\backslash\)/"):
\end{eulerprompt}
\eulerimg{15}{images/EMT4Statistika_Fadhila Asmaul Karimah-055.png}
\begin{eulercomment}
Soal 2

Banyaknya jawaban yang salah pada suatu kuiz dengan soal benar-salah
dari 15 mahasiswa adalah: 2,1,3,3,2,3,6,4,3,4,5,2,1,4,2.

Tentukan rata-rata jawaban salah pada kuiz tersebut!

Penyelesaian:

Diketahui:\\
\end{eulercomment}
\begin{eulerformula}
\[
\sum x_i={2,1,3,3,2,3,6,4,3,4,5,2,1,4,2}=45
\]
\end{eulerformula}
\begin{eulerttcomment}
                 n = 15
\end{eulerttcomment}
\begin{eulerformula}
\[
\bar{X} = \frac{\sum x_i}{n}
\]
\end{eulerformula}
\begin{eulerformula}
\[
\bar{X} = \frac{45}{15}
\]
\end{eulerformula}
\begin{eulerformula}
\[
\bar{X} = 3
\]
\end{eulerformula}
\begin{eulerprompt}
>x=[2,1,3,3,2,3,6,4,3,4,5,2,1,4,2]; mean(x),
\end{eulerprompt}
\begin{euleroutput}
  3
\end{euleroutput}
\begin{eulercomment}
Jadi, rata-rata jawaban salah pada kuz mahasiswa sebanyak 3 soal





Soal 3

Data berikut merupakan nilai yang diperoleh mahasiswa saat mengikuti
kuis harian Statistika: 74,81,65,56,96,63,55,91,93,85,51,59,69.

Tentukan varians dari data tersebut!

Penyelesaian:

- Mengurutkan data
\end{eulercomment}
\begin{eulerprompt}
>data=[74,81,65,56,96,63,55,91,93,85,51,59,69];
>urut=sort(data)
\end{eulerprompt}
\begin{euleroutput}
  [51,  55,  56,  59,  63,  65,  69,  74,  81,  85,  91,  93,  96]
\end{euleroutput}
\begin{eulercomment}
- Menentukan rata-rata(mean)
\end{eulercomment}
\begin{eulerprompt}
>x=mean(urut)
\end{eulerprompt}
\begin{euleroutput}
  72.154
\end{euleroutput}
\begin{eulercomment}
- Menentukan hasil dari pengurangan antara data dengan mean
\end{eulercomment}
\begin{eulerprompt}
>dev = urut-x
\end{eulerprompt}
\begin{euleroutput}
  [-21.154,  -17.154,  -16.154,  -13.154,  -9.1538,  -7.1538,  -3.1538,
  1.8462,  8.8462,  12.846,  18.846,  20.846,  23.846]
\end{euleroutput}
\begin{eulercomment}
- Menentukan varians
\end{eulercomment}
\begin{eulerprompt}
>varians = mean(dev^2)
\end{eulerprompt}
\begin{euleroutput}
  225.05
\end{euleroutput}
\begin{eulercomment}
Jadi, varians dari data nilai kuis Statistika adalah 225,05
\end{eulercomment}
\end{eulernotebook}
\end{document}


\end{document}